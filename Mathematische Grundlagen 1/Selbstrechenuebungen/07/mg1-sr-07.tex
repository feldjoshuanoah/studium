\documentclass[german,12pt]{homework}

\usepackage[ngerman]{babel}
\usepackage[utf8]{inputenc}
\usepackage[T1]{fontenc}

\usepackage{tikz}

\newcommand{\ZZ}{\mathbb{Z}}
\newcommand{\RR}{\mathbb{R}}

\newcommand{\dotproduct}[2]{\left\langle{#1, #2}\right\rangle}

\newcommand{\dd}{\,\differ}
\DeclareMathOperator{\differ}{d}
\DeclareMathOperator{\vecspan}{span}

\DeclarePairedDelimiter{\norm}{\lVert}{\rVert}
\DeclarePairedDelimiter{\enbrace}{(}{)}
\DeclarePairedDelimiter{\benbrace}{[}{]}
\DeclarePairedDelimiter{\penbrace}{\{}{\}}

\title{Selbstrechenübung 7}
\author{Joshua Feld, 406718}
\institute{RWTH Aachen University\\Center for Computational Engineering Science}
\class{Mathematische Grundlagen I}
\professor{Prof. Dr. Torrilhon \& Prof. Dr. Stamm}

\begin{document}
    \maketitle

    \section*{Aufgabe 1. (Konvergenz von Reihen)}

    \begin{problem}
        Untersuchen Sie folgende Reihen auf Konvergenz und absolute Konvergenz:
        \begin{enumerate}
            \item \(\sum_{k = 1}^\infty\frac{\enbrace*{-1}^k}{\sqrt{k}}\)
            \item \(\sum_{k = 1}^\infty\frac{k!}{k^k}\)
            \item \(\sum_{k = 1}^\infty\frac{k^2}{2^k}\)
            \item \(\sum_{k = 1}^\infty\benbrace*{\begin{pmatrix}2k\\k
            \end{pmatrix}}^{-1}\)
        \end{enumerate}

        \textbf{Hinweis:} \quad \emph{Bei b) kann verwendet werden, dass
        \(\enbrace*{1 + \frac{1}{k}}^k\) gegen \(e\) konvergiert.}
    \end{problem}

    \subsection*{Lösung}

    \section*{Aufgabe 2. (Potenzreihe)}

    \begin{problem}
        Bestimmen Sie alle \(x \in \RR\), für welche die Potenzreihe
        \(\sum_{k = 1}^\infty\frac{1}{k^2}x^k\) konvergiert.

        Verwenden Sie für die Untersuchung einmal das Quotientenkriterium und
        einmal das Wurzelkriterium. Erhalten Sie jeweils den gleichen
        Konvergenzradius?
    \end{problem}

    \subsection*{Lösung}

    \section*{Aufgabe 3. (Lineare Abbildung)}

    \begin{problem}
        Eine lineare Abbildung \(f: \RR^2 \to \RR^3\) sei gegeben durch
        \[f\begin{pmatrix}1\\1\end{pmatrix} = \begin{pmatrix}1\\0\\2
        \end{pmatrix} \quad \text{und} \quad f\begin{pmatrix}1\\2\end{pmatrix}
        = \begin{pmatrix}0\\1\\-1\end{pmatrix}.\]
        \begin{enumerate}
            \item Worauf wird der Vektor \(\begin{pmatrix}5\\7\end{pmatrix}\)
            abgebildet?
            \item Geben Sie eine Darstellung von Kern und Bild der Abbildung
            \(f\) an und bestimmen Sie jeweils die Dimension.
        \end{enumerate}
        \textbf{Hinweis:} \quad \emph{Nutzen Sie die Linearität der Abbildung
        aus.}
    \end{problem}

    \subsection*{Lösung}

    \section*{Aufgabe 4. (Lineare Abbildungen)}

    \begin{problem}
        Prüfen Sie, ob die folgenden Abbildungen \(L_i: V \to W\), \(i = 1,
        \ldots, 4\) linear sind und bestimmen sie ggf. ihren Kern und Bild
        sowie deren Dimensionen.
        \begin{enumerate}
            \item \(L_1: x \mapsto 5x - 1\), \(V = W = \RR\)
            \item \(L_2: x \mapsto 0\), \(V = W = \RR\)
            \item \(L_3: \enbrace*{x, y} \mapsto \enbrace*{x + y, x - y}\), \(V
            = W = \RR^2\)
            \item \(L_4: \enbrace*{x, y, z} \mapsto \enbrace*{z, y, 0}\), \(V =
            W = \RR^3\)
        \end{enumerate}
    \end{problem}

    \subsection*{Lösung}
\end{document}
