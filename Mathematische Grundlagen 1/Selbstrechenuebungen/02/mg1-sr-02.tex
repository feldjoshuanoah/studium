\documentclass[german,12pt]{homework}

\usepackage[ngerman]{babel}
\usepackage[utf8]{inputenc}
\usepackage[T1]{fontenc}

\usepackage{booktabs}
\usepackage{tikz}
\usepackage{stmaryrd}

\newcommand{\NN}{\mathbb{N}}
\newcommand{\RR}{\mathbb{R}}

\DeclarePairedDelimiter{\enbrace}{(}{)}
\DeclarePairedDelimiter{\benbrace}{[}{]}
\DeclarePairedDelimiter{\penbrace}{\{}{\}}

\title{Selbstrechenübung 2}
\author{Joshua Feld, 406718}
\institute{RWTH Aachen University\\Center for Computational Engineering Science}
\class{Mathematische Grundlagen I}
\professor{Prof. Dr. Torrilhon \& Prof. Dr. Stamm}

\begin{document}
    \maketitle

    \section*{Aufgabe 1. (Natürliche Zahlen, Beweistechniken)}

    \begin{problem}
        Die Zahl \(39\) hat die interessante Eigenschaft, dass
        \[39 = 3 \cdot 9 + 3 + 9\]
        gilt. Zeigen Sie, dass eine solche Darstellung für alle natürlichen
        Zahlen größer als \(9\) gilt, deren Dezimaldarstellung auf \(9\) endet.
        Analysieren Sie Ihren Beweis. Haben Sie eine Folgerung oder eine
        Äquivalenz gezeigt? Wie müssen Sie Ihren Beweis gegebenenfalls
        erweitern um die Äquivalenz zu zeigen?
    \end{problem}

    \subsection*{Lösung} Jede natürliche Zahl, insbesondere alle Zahlen \(\ge
    9\), kann geschrieben werden als
    \[10a + b, \quad a, b \in \NN.\]
    Für die Zahlen, die mit \(9\) enden, soll entsprechend gelten
    \[10a + 9 = 9a + a + 9 \iff 10a = 10a\]
    und das ist immer wahr.  Wir haben hier gezeigt, dann jede Zahl, die auf
    \(9\) endet, diese Darstellung hat. Um auch die andere Richtung zu zeigen
    beginnen wir mit
    \[10a + b = ab + a + b \iff 10a = ab + a \iff 10 = b + 1 \iff 9 = b\]
    Für \(a = 0\) erhalten wir die Zahlen \(1, \ldots, 9\), d.h. für diese gilt
    \[10 \cdot 0 + b = 0 \cdot b + 0 + b.\]
    Deshalb waren sie in der Aufgabenstellung ausgenommen. Damit sind
    \(\implies\) und \(\impliedby\) und somit auch die Äquivalenz gezeigt.

    \section*{Aufgabe 2. (Abbildungen)}

    \begin{problem}
        Wir betrachten die Funktion \(f: X \to Y\), \(X, y \subset \RR\) mit
        Funktionsvorschrift \(f\enbrace*{x} = \frac{1}{x + 1}\).
        \begin{enumerate}
            \item Geben Sei die größte Menge \(X \subset \RR\) an, sodass
            \(f\enbrace*{x}\), \(x \in X\), eine reelle Zahl ist. Diese Menge
            nennen wir den maximalen Definitionsbereich. Der maximale
            Wertebereich \(Y \subset \RR\) ist dann gegeben durch \(Y =
            f\enbrace*{X}\). Begründen Sie Ihre Wahl.
            \item Für welchen Definitions-/Bildbereich ist \(f\) invertierbar?
            Bestimmen Sie die Umkehrabbildung \(f^{-1}\).
            \item Skizzieren Sie \(f\) und \(f^{-1}\). Was fällt Ihnen auf?
        \end{enumerate}
    \end{problem}

    \subsection*{Lösung}
    \begin{enumerate}
        \item Der Wert \(\frac{1}{-1 + 1}\) ist nicht definiert. Folglich
        ergibt sich für den maximalen Definitionsbereich
        \[X = \RR \setminus \penbrace*{-1}.\]
        Es gilt \(\lim_{x \to -1^+}f\enbrace*{x} = \infty\), \(\lim_{x \to -1^-}
        f\enbrace*{x} = -\infty\) und \(\lim_{x \to \infty}f\enbrace*{x} = 0\),
        aber \(0 = \frac{1}{x + 1} \iff 0 = 1\ \lightning\). Also ergibt sich
        für den maximalen Wertebereich
        \[Y = \RR \setminus \penbrace*{0}.\]
        \item \(f: \RR \setminus \penbrace*{-1} \to \RR \setminus
        \penbrace*{0}\) ist bijektiv und somit invertierbar.
        \begin{itemize}
            \item Injektivität: Seien \(x_1, x_2 \in \RR \setminus
            \penbrace*{-1}\). Dann gilt
            \[\frac{1}{x_1 + 1} = \frac{1}{x_2 + 1} \iff x_2 + 1 = x_1 + 1 \iff
            x_1 = x_2.\]
            \item Surjektivität: \emph{siehe oben}
        \end{itemize}
        Für die Umkehrfunktion lösen wir die Funktionsgleichung \(f\enbrace*{x}
        = y\) nach \(x\) auf:
        \[y = \frac{1}{x + 1} \iff x + 1 = \frac{1}{y} \iff x = \frac{1}{y} -
        1,\]
        also gilt
        \[f^{-1}: \RR \setminus \penbrace*{0} \to \RR \setminus \penbrace*{-1},
        y \mapsto f^{-1}\enbrace*{y} = \frac{1}{y} - 1.\]
        \item Es fällt auf, dass die Graphen von \(f\) und \(f^{-1}\) sich als
        Spiegelungen des jeweils anderen Graphen an der Winkelhalbierenden \(y
        = x\) ergeben.
        \begin{center}
            \begin{tikzpicture}[scale=.75]
                \begin{scope}[thick,font=\scriptsize]
                    \draw [->] (-5,0) -- (5,0) node [above left] {\(x\)};
                    \draw [->] (0,-5) -- (0,5) node [below right] {\(y\)};

                    \foreach \n in {-4,-3,...,-1,1,2,...,4}{
                        \draw (\n,3pt) -- (\n,-3pt) node [below] {\(\n\)};
                        \draw (3pt,\n) -- (-3pt,\n) node [left] {\(\n\)};
                    }

                    \draw[dotted] (-5,-5) -- (5,5);
                    \draw node at (3,4) {\(y = x\)};
                    \draw[domain=-5:-1.2, smooth, variable=\x] plot ({\x}, {1/(1 + \x)});
                    \draw[domain=-0.8:5, smooth, variable=\x] plot ({\x}, {1/(1 + \x)});
                    \draw node at (-1.5,4) {\(f\enbrace*{x}\)};
                    \draw[domain=-5:-0.25, smooth, variable=\x, dashed] plot ({\x}, {(1/\x) - 1});
                    \draw[domain=0.168:5, smooth, variable=\x, dashed] plot ({\x}, {(1/\x) - 1});
                    \draw node at (3.6,-1.2) {\(f^{-1}\enbrace*{x}\)};
                \end{scope}
            \end{tikzpicture}
        \end{center}
    \end{enumerate}

    \section*{Aufgabe 3. (Abbildungen)}

    \begin{problem}
        Geben Sie Abbildungen mit den folgenden Eigenschaften an und begründen
        Sie Ihre Wahl.
        \begin{enumerate}
            \item \(f_1: \RR \to \RR\) ist injektiv, aber nicht surjektiv.
            \item \(f_2: \RR \to \RR\) ist surjektiv, aber nicht injektiv.
            \item \(f_2: \RR \to \benbrace*{-1, 1}\) ist surjektiv, aber nicht
            injektiv.
            \item \(f_2: \benbrace*{-1, 1} \to \benbrace*{1, 10}\) ist bijektiv.
        \end{enumerate}
    \end{problem}

    \subsection*{Lösung}
    \begin{enumerate}
        \item z.B. \(f_1\enbrace*{x} = \arctan\enbrace*{x}\), denn
        \begin{itemize}
            \item Injektivität: Sei \(f_1\enbrace*{x} = f_1\enbrace*{y}\) und
            \(x \ne y\). Es gilt \(\arctan\enbrace*{x} \ne
            \arctan\enbrace*{y}\), da der Arkustangens streng monoton steigend
            ist.
            \item Surjektivität: Da der Wertebereich \(W_{f_1} = \enbrace*{-
            \frac{\pi}{2}, \frac{\pi}{2}}\) eine echte Teilmenge des
            Bildbereichs \(\RR\) ist, ist \(f_1\) nicht surjektiv (Wertebereich
            \(\ne\) Bildbereich).
        \end{itemize}
        \item z.B. \(f_2\enbrace*{x} = x\enbrace*{x - 1}\enbrace*{x + 1} = x^3
        - x\), denn
        \begin{itemize}
            \item Injektivität: Seien \(x = -1, y = 1 \in \RR\), dann gilt
            \(f_2\enbrace*{x} = 0 = f_2\enbrace*{y}\) aber \(x \ne y\).
            \item Surjektivität: Für alle \(y \in \RR\) existiert mindestens
            ein \(x \in \RR\) mit \(f_2\enbrace*{x} = y\). (\(f_2\enbrace*{x}\)
            geht gegen plus Unendlich für \(x\) gegen plus Unendlich und
            \(f_2\enbrace*{x}\) geht gegen minus Unendlich für \(x\) gegen
            minus Unendlich.)
        \end{itemize}
        \item z.B. \(f_3\enbrace*{x} = \sin\enbrace*{x}\), denn
        \begin{itemize}
            \item Injektivität: Seien \(x = 0, y = 2\pi \in \RR\), dann gilt
            \(\sin\enbrace*{x} = \sin\enbrace*{y}\) aber \(x \ne y\).
            \item Surjektivität: Da jeder Wert in \(\benbrace*{-1, 1}\)
            mindestens einmal (sogar unendlich oft) angenommen wird, ist
            \(f_3\) surjektiv.
        \end{itemize}
        \item z.B. \(f_4\enbrace*{x} = \frac{9}{2}x + \frac{11}{2}\). Diese
        Funktion kann wie folgt konstruiert werden. konstruiere
        \(f_4\enbrace*{x} = ax + b\) (Geradengleichung) mit \(a, b \in \RR\),
        die durch die Punkte \(\enbrace*{-1, 1}\) und \(\enbrace*{1, 10}\) geht:
        \[a = \frac{10 - 1}{1 - \enbrace*{-1}} = \frac{9}{2} \quad \text{und}
        \quad b = 1 + \frac{9}{2} = \frac{11}{2}.\]
    \end{enumerate}

    \section*{Aufgabe 4. (Vollständige Induktion)}

    \begin{problem}
        Zeigen Sie per vollständiger Induktion, dass für alle \(n \in \NN\)
        gilt:
        \[133\text{ ist Teiler von }11^{n + 1} + 12^{2n - 1}.\]
    \end{problem}

    \subsection*{Lösung}
    \begin{itemize}
        \item Induktionsverankerung: (\(n = 1\))
        \[11^{1 + 1} + 12^{2 \cdot 1 - 1} = 121 + 12 = 133.\]
        Offensichtlich ist \(133\) ein Teiler von \(133\).
        \item Induktionsvoraussetzung: Die Aussage gelte für ein beliebiges
        aber festes \(n \in \NN\).
        \item Induktionsschritt: (\(n \to n + 1\))
        \begin{align*}
            11^{\enbrace*{n + 1} + 1} + 12^{2\enbrace*{n + 1} - 1} &= 11^{n +
            2} + 12^{2n + 1}\\
            &= 11 \cdot 11^{n + 1} + 12^2 \cdot 12^{2n - 1}\\
            &= 11 \cdot 11^{n + 1} + \enbrace*{133 + 11} \cdot 12^{2n - 1}\\
            &= 11 \cdot \enbrace*{11^{n + 1} + 12^{2n - 1}} + 133 \cdot 12^{2n
            - 1}
        \end{align*}
        und nach Induktionsvoraussetzung ist \(133\) ein Teiler von \(11^{n +
        1} + 12^{2n - 1}\) und damit teilt \(133\) beide Summanden, also auch
        die Summe.
    \end{itemize}
    Nach dem Prinzip der vollständigen Induktion gilt die Aussage für alle \(n
    \in \NN\).
\end{document}
