\documentclass[german,12pt]{homework}

\usepackage[ngerman]{babel}
\usepackage[utf8]{inputenc}
\usepackage[T1]{fontenc}

\usepackage{tikz}

\newcommand{\ZZ}{\mathbb{Z}}
\newcommand{\RR}{\mathbb{R}}

\newcommand{\dotproduct}[2]{\left\langle{#1, #2}\right\rangle}

\newcommand{\dd}{\,\differ}
\DeclareMathOperator{\differ}{d}
\DeclareMathOperator{\vecspan}{span}

\DeclarePairedDelimiter{\norm}{\lVert}{\rVert}
\DeclarePairedDelimiter{\enbrace}{(}{)}
\DeclarePairedDelimiter{\benbrace}{[}{]}
\DeclarePairedDelimiter{\penbrace}{\{}{\}}

\title{Selbstrechenübung 6}
\author{Joshua Feld, 406718}
\institute{RWTH Aachen University\\Center for Computational Engineering Science}
\class{Mathematische Grundlagen I}
\professor{Prof. Dr. Torrilhon \& Prof. Dr. Stamm}

\begin{document}
    \maketitle

    \section*{Aufgabe 1. (Reihen)}

    \begin{problem}
        Beweisen Sie die Konvergenz der folgenden Reihe:
        \[\sum_{n = 1}^\infty{a_n}, \quad \text{mit }a_n = \frac{3^n}{5^n +
        1}.\]
        Gehen Sie dabei wie folgt vor:
        \begin{enumerate}
            \item Betrachten Sie die Folge der Partialsummen.
            \item Schätzen Sie geeignet nach oben ab.
            \item Erinnern Sie sich an die geometrische Summenformel.
        \end{enumerate}
    \end{problem}

    \subsection*{Lösung}
    \begin{enumerate}
        \item Wir definieren \(b_k = \sum_{n = 1}^ka_n\) als Folge von
        Teilsummen.
        \item Für \(a_n\) gilt
        \[a_n = \frac{3^n}{5^n + 1} \le \frac{3^n}{5^n} =
        \enbrace*{\frac{3}{5}}^n.\]
        Somit ist
        \[b_k \le \sum_{n = 1}^k\enbrace*{\frac{3}{5}}^n =: c_k.\]
        \item Die geometrische Summenformel ist \(\sum_{k = 0}^nq^k =
        \frac{1 - q^{n + 1}}{1 - q}\) für \(q \ne 1\). Das heißt
        \[c_k = \sum_{n = 1}^k\enbrace*{\frac{3}{5}}^n =
        \sum_{n = 0}^k\enbrace*{\frac{3}{5}}^n - \enbrace*{\frac{3}{5}}^0 =
        \frac{1 - \enbrace*{\frac{3}{5}}^{k + 1}}{1 - \frac{3}{5}} - 1.\]
        Da \(\enbrace*{\frac{3}{5}}^{k + 1} \to 0\) für \(k \to \infty\), gilt
        \(c_k \frac{1}{\frac{2}{5}} - 1 = \frac{3}{2}\) für \(k \to \infty\) und
        somit auch
        \[b_k \to b \le \frac{3}{2} \text{, mit }b = \sum_{n = 1}^{\infty}a_n\
        \enbrace*{k \to \infty}.\]
        Also konvergiert die Reihe.
    \end{enumerate}

    \section*{Aufgabe 2. (Konvergenz von Reihen)}

    \begin{problem}
        Wir approximieren ein rechtwinkliges Dreieck mit Kathetenlänge \(1\) in
        jedem Schritt durch weitere Hinzunahme von (kleineren) Quadraten, siehe
        Skizze:
        \begin{center}
            \begin{tikzpicture}[scale=.6]
                \foreach \n in {0,1,...,3}{
                    \draw (0 + \n * 6,0) -- (0 + \n * 6,5) -- (5 + \n * 6,0) -- cycle;
                }
                \filldraw[fill=gray,fill opacity=0.45] (6,0) -- (6,2.5) -- (8.5,2.5) -- (8.5,0) -- cycle;
                \filldraw[fill=gray,fill opacity=0.45] (12,0) -- (12,2.5) -- (14.5,2.5) -- (14.5,0) -- cycle;
                \filldraw[fill=gray,fill opacity=0.45] (18,0) -- (18,2.5) -- (20.5,2.5) -- (20.5,0) -- cycle;

                \filldraw[fill=gray,fill opacity=0.45] (12,2.5) -- (12,3.75) -- (13.25,3.75) -- (13.25,2.5) -- cycle;
                \filldraw[fill=gray,fill opacity=0.45] (18,2.5) -- (18,3.75) -- (19.25,3.75) -- (19.25,2.5) -- cycle;
                \filldraw[fill=gray,fill opacity=0.45] (14.5,0) -- (14.5,1.25) -- (15.75,1.25) -- (15.75,0) -- cycle;
                \filldraw[fill=gray,fill opacity=0.45] (20.5,0) -- (20.5,1.25) -- (21.75,1.25) -- (21.75,0) -- cycle;

                \filldraw[fill=gray,fill opacity=0.45] (18,3.75) -- (18,4.375) -- (18.625,4.375) -- (18.625,3.75) -- cycle;
                \filldraw[fill=gray,fill opacity=0.45] (19.25,2.5) -- (19.25,3.125) -- (19.875,3.125) -- (19.875,2.5) -- cycle;
                \filldraw[fill=gray,fill opacity=0.45] (20.5,1.25) -- (21.125,1.25) -- (21.125,1.875) -- (20.5,1.875) -- cycle;
                \filldraw[fill=gray,fill opacity=0.45] (21.75,0) -- (22.375,0) -- (22.375,0.625) -- (21.75,0.625) -- cycle;

                \draw node at (8.5,-0.75) {\(n = 0\)};
                \draw node at (14.5,-0.75) {\(n = 1\)};
                \draw node at (20.5,-0.75) {\(n = 2\)};
            \end{tikzpicture}
        \end{center}
        Zeigen Sie, dass die zu den Flächeninhalten der Quadrate gehörende
        Reihe gegen den Flächeninhalt des Dreiecks konvergiert.
    \end{problem}

    \subsection*{Lösung} Es gilt für die Reihe der aufsummierten Flächeninhalte
    der Quadrate:
    \[A = \sum_{n = 0}^{\infty}2^n \cdot \enbrace*{\frac{1}{2^{n + 1}}}^2
    = \sum_{n = 0}^\infty\frac{1}{2^{n + 2}} = \frac{1}{4} \cdot
    \sum_{n = 0}^\infty\enbrace*{\frac{1}{2}}^n = \frac{1}{4} \cdot\frac{1}
    {1 - \frac{1}{2}} = \frac{1}{4} \cdot 2 = \frac{1}{2}.\]
    Die Summe der Flächeninhalte der Quadrate konvergiert also gegen den
    Flächeninhalt des Dreiecks.

    \section*{Aufgabe 3. (Bestapproximation)}

    \begin{problem}
        Sei \(V = \RR^3\) und \(U = \vecspan\penbrace*{e_1, e_2} \subset V\),
        wobei \(e_i\) der \(i\)-te Einheitsvektor ist.
        \begin{enumerate}
            \item Für welchen Vektor \(v_0\) gilt
            \[\norm*{v_0 - v_1}_2 \le \norm*{v - v_1}_2 \quad \forall{v \in U}\]
            mit \(v = \begin{pmatrix}1\\2\\3\end{pmatrix}\)? Begründen Sie Ihre
            Antwort.
            \item Wie groß ist die Norm \(\norm*{v_0 - v_1}_2\)?
        \end{enumerate}
    \end{problem}

    \subsection*{Lösung}
    \begin{enumerate}
        \item Wir finden \(v_0\) als Bestapproximation von \(v_1\) durch
        Projektion auf \(U\), d.h.
        \begin{align*}
            v_0 &= \dotproduct{v_1}{e_1}e_1 + \dotproduct{v_1}{e_2}e_2\\
            &= \dotproduct{\begin{pmatrix}1\\2\\3\end{pmatrix}}{\begin{pmatrix}
            1\\0\\0\end{pmatrix}} \cdot \begin{pmatrix}1\\0\\0\end{pmatrix} +
            \dotproduct{\begin{pmatrix}1\\2\\3\end{pmatrix}}{\begin{pmatrix}
            0\\1\\0\end{pmatrix}} \cdot \begin{pmatrix}0\\1\\0\end{pmatrix}\\
            &= \begin{pmatrix}1\\0\\0\end{pmatrix} + 2 \cdot \begin{pmatrix}
            0\\1\\0\end{pmatrix} = \begin{pmatrix}1\\2\\0\end{pmatrix}.
        \end{align*}
        \item Wir berechnen sofort
        \[\norm*{v_0 - v_1}_2 = \norm*{\begin{pmatrix}1\\2\\0\end{pmatrix}
        - \begin{pmatrix}1\\2\\3\end{pmatrix}}_2 = \norm*{\begin{pmatrix}
        0\\0\\-3\end{pmatrix}}_2 = 3.\]
    \end{enumerate}

    \section*{Aufgabe 4. (Bestapproximation)}

    \begin{problem}
        Sei \(V = C\enbrace*{\benbrace*{0, 1}}\) der Vektorraum der stetigen,
        reellwertigen Funktion auf \(\benbrace*{0, 1}\), ausgestattet mit dem
        Skalarprodukt
        \[\dotproduct{f}{g} = \int_0^1f\enbrace*{x}g\enbrace*{x}\dd{x} \quad
        \text{für alle }f, g \in V.\]
        Bestimmen Sie die Bestapproximation \(u^* \in U\) des Unterraums
        \[U = \vecspan\penbrace*{1, x, x^2}\]
        an die Funktion \(f \in V\) mit \(f\enbrace*{x} = x^2 - \frac{1}{3}\).
        Bevor Sie rechnen: Welche Genauigkeit der Bestapproximation erwarten
        Sie hier?

        \textbf{Hinweis:} \quad \emph{Eine Orthonormalbasis von \(U\) ist
        gegeben durch
        \[u_0\enbrace*{x} = 1, \quad u_1\enbrace*{x} = 2\sqrt{3}\enbrace*{x -
        \frac{1}{2}}, \quad u_2\enbrace*{x} = 6\sqrt{5}\enbrace*{x^2 - x +
        \frac{1}{6}}.\]}
    \end{problem}

    \subsection*{Lösung} Da sich die zu approximierende Funktion aus Vektoren
    vom Unterraum \(U\) zusammensetzt, wird hier erwartet, dass die
    Bestapproximation zum genauen Ergebnis führt. Da die Bestapproximation
    \(u^*\) im Untervektorraum \(U\) liegt, lässt sich \(u^*\) durch eine
    Linearkombination des Basisvektoren darstellen:
    \[u^* = \alpha_0u_0 + \alpha_1u_1 + \alpha_2u_2.\]
    Die Bestapproximation \(u^*\) entspricht gerade der Funktion in \(U\), die
    zu der gegebenen Funktion \(f\) den kürzesten Abstand hat. Damit muss also
    gelten, dass die Verbindung von \(f\) und \(u^*\) orthogonal auf dem
    Untervektorraum \(U\) liegt. Für \(i = 0, 1, 2\):
    \begin{align*}
        \dotproduct{f - u^*}{u_i} = 0 &\iff \dotproduct{f}{u_i} =
        \dotproduct{u^*}{u_i}\\
        &\iff \dotproduct{f}{u_i} = \dotproduct{\sum_{j = 0}^2\alpha_ju_j}
        {u_i}\\
        &\iff \dotproduct{f}{u_i} = \sum_{j = 0}^2\alpha_j\dotproduct{u_j}
        {u_i}\\
        &\underbrace{\iff}_\text{orthog.} \dotproduct{f}{u_i} =
        \alpha_i\dotproduct{u_i}{u_i}\\
        &\underbrace{\iff}_\text{orthon.} \dotproduct{f}{u_i} = \alpha_i \cdot
        1.
    \end{align*}
    Hierbei haben wir die folgenden beiden Eigenschaften genutzt:
    \begin{itemize}
        \item Das Skalarprodukt von orthogonalen Vektoren ist gleich \(0\); \(\dotproduct{u_i}{u_j} = 0\) für \(i \ne j\).
        \item Das Skalarprodukt von orthonormalen Vektoren ist gleich \(1\);
        \(\dotproduct{u_i}{u_j} = 1\) für \(i = j\).
    \end{itemize}
    Damit liegen drei Gleichungen \(\dotproduct{f}{u_0} = \alpha_0\),
    \(\dotproduct{f}{u_1} = \alpha_1\) und \(\dotproduct{f}{u_2} = \alpha_2\)
    vor, um die drei Unbekannten \(\alpha_0\), \(\alpha_1\) und \(\alpha_2\) für
    \(u^*\) zu bestimmen:
    \begin{align*}
        \alpha_0 = \dotproduct{f}{u_0} &= \int_0^1\enbrace*{x^2 - \frac{1}{3}}
        \cdot 1\dd{x} = \frac{1}{3} - \frac{1}{3} = 0,\\
        \alpha_1 = \dotproduct{f}{u_1} &= \int_0^1\enbrace*{x^2 - \frac{1}{3}}
        \enbrace*{2\sqrt{3}\enbrace*{x - \frac{1}{2}}}\dd{x},\\
        &= \enbrace*{2\sqrt{3}}\int_0^1x^3 - \frac{1}{2}x^2 + \frac{1}{3}x -
        \frac{1}{6}\dd{x}\\
        &= \enbrace*{2\sqrt{3}}\enbrace*{\frac{1}{4} - \frac{1}{6} + \frac{1}{6}
        - \frac{1}{6}} = \enbrace*{2\sqrt{3}} \cdot \frac{1}{12} =
        \frac{1}{2\sqrt{3}}.\\
        \alpha_2 = \dotproduct{f}{u_2} &= \int_0^1\enbrace*{x^2 - \frac{1}{3}}
        \enbrace*{6\sqrt{5}\enbrace*{x^2- x + \frac{1}{6}}}\dd{x}\\
        &= \enbrace*{6\sqrt{5}}\int_0^1x^4 - x^3 + \frac{1}{6}x^2 -
        \frac{1}{3}x^2 + \frac{1}{3}x - \frac{1}{18}\dd{x}\\
        &= \enbrace*{6\sqrt{5}}\enbrace*{\frac{1}{5} - \frac{1}{4} + \frac{1}{18} - \frac{1}{9} + \frac{1}{6} - \frac{1}{18}}\\
        &= \enbrace*{6\sqrt{5}}\enbrace*{\frac{1}{5} - \frac{1}{4} - \frac{1}{9}
        + \frac{1}{6}}\\
        &= \enbrace*{6\sqrt{5}}\enbrace*{\frac{36}{180} - \frac{45}{180} -
        \frac{20}{180} + \frac{30}{180}}\\
        &= \enbrace*{6\sqrt{5}}\enbrace*{\frac{1}{\enbrace*{6\sqrt{5}}^2}}
        = \frac{1}{6\sqrt{5}}.
    \end{align*}
    Damit ist
    \begin{align*}
        u^* &= \sum_{j = 0}^2\alpha_j \cdot u_j\enbrace*{x} = \sum_{j = 0}^2
        \dotproduct{f}{u_j} \cdot u_j\\
        &= \frac{1}{2\sqrt{3}} \cdot 2\sqrt{3} \cdot \enbrace*{x - \frac{1}{2}}
        + \frac{1}{6\sqrt{5}} \cdot 6\sqrt{5} \cdot \enbrace*{x^2 - x +
        \frac{1}{6}}\\
        &= \enbrace*{x - \frac{1}{2}} + \enbrace*{x^2 - x + \frac{1}{6}}
        = x^2 - \frac{1}{3}.
    \end{align*}
    Somit ist die Vermutung einer Bestapproximation ohne Fehler bestätigt.
\end{document}
