\documentclass[german,12pt]{homework}

\usepackage[ngerman]{babel}
\usepackage[utf8]{inputenc}
\usepackage[T1]{fontenc}

\usepackage{tikz}

\newcommand{\ZZ}{\mathbb{Z}}
\newcommand{\RR}{\mathbb{R}}

\newcommand{\dotproduct}[2]{\left\langle{#1, #2}\right\rangle}

\newcommand{\dd}{\,\differ}
\DeclareMathOperator{\differ}{d}
\DeclareMathOperator{\vecspan}{span}

\DeclarePairedDelimiter{\norm}{\lVert}{\rVert}
\DeclarePairedDelimiter{\enbrace}{(}{)}
\DeclarePairedDelimiter{\benbrace}{[}{]}
\DeclarePairedDelimiter{\penbrace}{\{}{\}}

\title{Selbstrechenübung 6}
\author{Joshua Feld, 406718}
\institute{RWTH Aachen University\\Center for Computational Engineering Science}
\class{Mathematische Grundlagen I}
\professor{Prof. Dr. Torrilhon \& Prof. Dr. Stamm}

\begin{document}
    \maketitle

    \section*{Aufgabe 1. (Reihen)}

    \begin{problem}
        Beweisen Sie die Konvergenz der folgenden Reihe:
        \[\sum_{n = 1}^\infty{a_n}, \quad \text{mit }a_n = \frac{3^n}{5^n +
        1}.\]
        Gehen Sie dabei wie folgt vor:
        \begin{enumerate}
            \item Betrachten Sie die Folge der Partialsummen.
            \item Schätzen Sie geeignet nach oben ab.
            \item Erinnern Sie sich an die geometrische Summenformel.
        \end{enumerate}
    \end{problem}

    \subsection*{Lösung}

    \section*{Aufgabe 2. (Konvergenz von Reihen)}

    \begin{problem}
        Wir approximieren ein rechtwinkliges Dreieck mit Kathetenlänge \(1\) in
        jedem Schritt durch weitere Hinzunahme von (kleineren) Quadraten, siehe
        Skizze:
        \begin{center}
            \begin{tikzpicture}[scale=.6]
                \foreach \n in {0,1,...,3}{
                    \draw (0 + \n * 6,0) -- (0 + \n * 6,5) -- (5 + \n * 6,0) -- cycle;
                }
                \filldraw[fill=gray,fill opacity=0.45] (6,0) -- (6,2.5) -- (8.5,2.5) -- (8.5,0) -- cycle;
                \filldraw[fill=gray,fill opacity=0.45] (12,0) -- (12,2.5) -- (14.5,2.5) -- (14.5,0) -- cycle;
                \filldraw[fill=gray,fill opacity=0.45] (18,0) -- (18,2.5) -- (20.5,2.5) -- (20.5,0) -- cycle;

                \filldraw[fill=gray,fill opacity=0.45] (12,2.5) -- (12,3.75) -- (13.25,3.75) -- (13.25,2.5) -- cycle;
                \filldraw[fill=gray,fill opacity=0.45] (18,2.5) -- (18,3.75) -- (19.25,3.75) -- (19.25,2.5) -- cycle;
                \filldraw[fill=gray,fill opacity=0.45] (14.5,0) -- (14.5,1.25) -- (15.75,1.25) -- (15.75,0) -- cycle;
                \filldraw[fill=gray,fill opacity=0.45] (20.5,0) -- (20.5,1.25) -- (21.75,1.25) -- (21.75,0) -- cycle;

                \filldraw[fill=gray,fill opacity=0.45] (18,3.75) -- (18,4.375) -- (18.625,4.375) -- (18.625,3.75) -- cycle;
                \filldraw[fill=gray,fill opacity=0.45] (19.25,2.5) -- (19.25,3.125) -- (19.875,3.125) -- (19.875,2.5) -- cycle;
                \filldraw[fill=gray,fill opacity=0.45] (20.5,1.25) -- (21.125,1.25) -- (21.125,1.875) -- (20.5,1.875) -- cycle;
                \filldraw[fill=gray,fill opacity=0.45] (21.75,0) -- (22.375,0) -- (22.375,0.625) -- (21.75,0.625) -- cycle;

                \draw node at (8.5,-0.75) {\(n = 0\)};
                \draw node at (14.5,-0.75) {\(n = 1\)};
                \draw node at (20.5,-0.75) {\(n = 2\)};
            \end{tikzpicture}
        \end{center}
        Zeigen Sie, dass die zu den Flächeninhalten der Quadrate gehörende
        Reihe gegen den Flächeninhalt des Dreiecks konvergiert.
    \end{problem}

    \subsection*{Lösung}

    \section*{Aufgabe 3. (Bestapproximation)}

    \begin{problem}
        Sei \(V = \RR^3\) und \(U = \vecspan\penbrace*{e_1, e_2} \subset V\),
        wobei \(e_i\) der \(i\)-te Einheitsvektor ist.
        \begin{enumerate}
            \item Für welchen Vektor \(v_0\) gilt
            \[\norm*{v_0 - v_1}_2 \le \norm*{v - v_1}_2 \quad \forall{v \in U}\]
            mit \(v = \begin{pmatrix}1\\2\\3\end{pmatrix}\)? Begründen Sie Ihre
            Antwort.
            \item Wie groß ist die Norm \(\norm*{v_0 - v_1}_2\)?
        \end{enumerate}
    \end{problem}

    \subsection*{Lösung}

    \section*{Aufgabe 4. (Bestapproximation)}

    \begin{problem}
        Sei \(V = C\enbrace*{\benbrace*{0, 1}}\) der Vektorraum der stetigen,
        reellwertigen Funktion auf \(\benbrace*{0, 1}\), ausgestattet mit dem
        Skalarprodukt
        \[\dotproduct*{f}{g} = \int_0^1f\enbrace*{x}g\enbrace*{x}\dd{x} \quad
        \text{für alle }f, g \in V.\]
        Bestimmen Sie die Bestapproximation \(u^* \in U\) des Unterraums
        \[U = \vecspan\penbrace*{1, x, x^2}\]
        an die Funktion \(f \in V\) mit \(f\enbrace*{x} = x^2 - \frac{1}{3}\).
        Bevor Sie rechnen: Welche Genauigkeit der Bestapproximation erwarten
        Sie hier?

        \textbf{Hinweis:} \quad \emph{Eine Orthonormalbasis von \(U\) ist
        gegeben durch
        \[u_0\enbrace*{x} = 1, \quad u_1\enbrace*{x} = 2\sqrt{3}\enbrace*{x -
        \frac{1}{2}}, \quad u_2\enbrace*{x} = 6\sqrt{5}\enbrace*{x^2 - x +
        \frac{1}{6}}.\]}
    \end{problem}
\end{document}
