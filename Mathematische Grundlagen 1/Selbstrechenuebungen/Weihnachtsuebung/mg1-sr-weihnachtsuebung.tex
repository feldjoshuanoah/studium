\documentclass[german,12pt]{homework}

\usepackage[ngerman]{babel}
\usepackage[utf8]{inputenc}
\usepackage[T1]{fontenc}

\usepackage{tikz}

\newcommand{\NN}{\mathbb{N}}
\newcommand{\ZZ}{\mathbb{Z}}
\newcommand{\RR}{\mathbb{R}}
\newcommand{\CC}{\mathbb{C}}

\newcommand{\dotproduct}[2]{\left\langle{#1, #2}\right\rangle}

\newcommand{\dd}{\,\differ}
\DeclareMathOperator{\differ}{d}
\DeclareMathOperator{\vecspan}{span}

\DeclarePairedDelimiter{\absolute}{\lvert}{\rvert}
\DeclarePairedDelimiter{\norm}{\lVert}{\rVert}
\DeclarePairedDelimiter{\enbrace}{(}{)}
\DeclarePairedDelimiter{\benbrace}{[}{]}
\DeclarePairedDelimiter{\penbrace}{\{}{\}}

\title{Weihnachtsübung}
\author{Joshua Feld, 406718}
\institute{RWTH Aachen University\\Center for Computational Engineering Science}
\class{Mathematische Grundlagen I}
\professor{Prof. Dr. Torrilhon \& Prof. Dr. Stamm}

\begin{document}
    \maketitle
    \section*{Aufgabe 1. (Vollständige Induktion)}

    \begin{problem}
        Zeigen Sie, dass für alle \(n \in \NN\) gilt:
        \begin{enumerate}
            \item \(1 + q + q^2 + \ldots + q^n = \frac{1 - q^{n + 1}}{1 - q}, \quad q \ne 1\)
            \item \(\sum_{k = 1}^n\frac{1}{k^2} \le 2 - \frac{1}{n}\)
        \end{enumerate}
        \textbf{Hinweis:} \quad \emph{Übersetzen Sie Teil a) zunächst in die Summenschreibweise. Teil a) heißt auch die ``geometrische Summenformel''.}
    \end{problem}

    \subsection*{Lösung} \clearpage

    \section*{Aufgabe 2. (Einheitswurzeln, Nullstellen)}

    \begin{problem}
        \begin{enumerate}
            \item Bestimmen Sie die Polarkoordinatendarstellung der komplexen Zahlen \(-12 - 7i\), \(\overline{3 + 5i}\) und \(\enbrace*{1 - i}^{10}\), wobei jeweils das Argument in \(\left(-\pi, \pi\right]\) liegen soll.
            \item Bestimmen Sie alle \(z \in \CC\) für die gilt: \(z^4 = -\frac{1}{2} + \frac{\sqrt{3}}{2}i\).
            \item Das Polynom \(p\enbrace*{x} := 2x^4 - 4x^3 - 8x^2 - 16x - 64\)
            hat die Nullstellen \(x = -2\) und \(x = 4\). Ermitteln Sie alle weiteren Nullstellen von \(p\) in \(\CC\).
        \end{enumerate}
    \end{problem}

    \subsection*{Lösung} \clearpage

    \section*{Aufgabe 3. (Komplexe Folgen)}

    \begin{problem}
        Wir betrachten die Folgen \(\enbrace*{a_n}_{n \in \NN}\) und \(\enbrace*{b_n}_{n \in \NN}\) mit
        \[a_n = \enbrace*{\frac{1 + i\sqrt{3}}{2}}^n, \qquad
        b_n = \enbrace*{-\frac{1}{2}}^na_n.\]
        \begin{enumerate}
            \item Bestimmen Sie alle konvergenten Teilfolgen von \(\enbrace*{a_n}\) und \(\enbrace*{b_n}\) und deren Grenzwert.
            \item Bestimmen Sie \(\max\), \(\min\), \(\inf\) und \(\sup\) der
            Folgen \(\enbrace*{c_n^k}_{n \in \NN}, k = 1, 2\) mit \(c_n^1 = \absolute*{a_n}\) und \(c_n^2 = \absolute*{b_n}\).
            \item Skizzieren Sie \(a_n\) und \(b_n\) in der komplexen Ebene.
        \end{enumerate}
    \end{problem}

    \subsection*{Lösung} \clearpage

    \section*{Aufgabe 4. (Konvergente Reihen und Potenzreihen)}

    \begin{problem}
        \begin{enumerate}
            \item Untersuchen Sie die folgende Reihe auf Konvergenz und bestimmen Sie gegebenenfalls den Grenzwert:
            \[\sum_{k = 1}^\infty\frac{\enbrace*{2i}^k + 3^{k - 1}}{5^k}.\]
            \item Bestimmen Sie den Konvergenzradius der Potenzreihe \(\sum_{n = 0}^\infty\cos\enbrace*{n\pi}\enbrace*{\frac{x}{2}}^n\).
        \end{enumerate}
    \end{problem}

    \subsection*{Lösung} \clearpage

    \section*{Aufgabe 5. (Stetigkeit, gleichmäßige und Lipschitz-Stetigkeit)}

    \begin{problem}
        \begin{enumerate}
            \item Untersuchen Sie die folgenden Funktionen auf Stetigkeit, gleichmäßige Stetigkeit und Lipschitz-Stetigkeit:
            \begin{enumerate}[label=\roman*)]
                \item \(f\enbrace*{x} = x^2\) mit \(f: \RR \to \RR\),
                \item \(g\enbrace*{x} = x^2\) mit \(f: \benbrace*{0, 1} \to \RR\),
                \item \(h\enbrace*{x} = m \cdot x + b\) mit \(m, b \in \RR\).
            \end{enumerate}
            \textbf{Hinweis:} \emph{Benutzen Sie die \(\varepsilon\)-\(\delta\)-Definition der Stetigkeit nur in Teil (i), argumentieren Sie in den anderen beiden Fällen geschickter.}
            \item Benutzen Sie den Zwischenwertsatz um zu zeigen, dass \(k\enbrace*{x} = mx^2 + b\) immer mindestens eine Nullstelle besitzt, wenn \(m < 0\) und \(b > 0\).
        \end{enumerate}
    \end{problem}

    \subsection*{Lösung} \clearpage

    \section*{Aufgabe 6. (Lipschitz-Stetigkeit)}

    \begin{problem}
        \begin{enumerate}
            \item Zeigen Sie die Lipschitz-Stetigkeit der Funktion
            \[f: \benbrace*{1, 10} \to \RR, f\enbrace*{x} = \frac{3x^2}{x^2 + 1}\]
            und geben Sie eine Lipschitz-Konstante \(L > 0\) an.
            \item Beweisen Sie, dass aus der Lipschitz-Stetigkeit einer Funktion
            die gleichmäßige Stetigkeit folgt.
        \end{enumerate}
    \end{problem}

    \subsection*{Lösung} \clearpage

    \section*{Aufgabe 7. (Vektorraum und Skalarprodukt)}

    \begin{problem}
        Seien \(a, b \in \RR\) mit \(a < b\), \(I := \benbrace*{a, b}\) und
        \(k \in \NN_0\).
        \[C^k\enbrace*{I} := \penbrace*{\left.f: I \to \RR\right|\,f\text{ ist }k\text{-mal stetig differenzierbar auf }I}\]
        und
        \[\dotproduct{f}{g}_{H^k} := \sum_{l = 0}^k\int_a^bf^{\enbrace*{l}}\enbrace*{x} \cdot g^{\enbrace*{l}}\enbrace*{x}\dd{x} \quad {\forall}f, g \in C^k\enbrace*{U}.\]
        Zeigen Sie, dass \(C^k\enbrace*{I}\) unter der Vektoraddition
        \[\enbrace*{f + g}\enbrace*{x} := f\enbrace*{x} + g\enbrace*{x} \quad
        {\forall}x \in I, f, g \in C^k\enbrace*{I}\]
        und der Multiplikation mit Skalaren
        \[\enbrace*{\alpha \cdot f}\enbrace*{x} := \alpha \cdot f\enbrace*{x}
        \quad {\forall}\alpha \in \RR, x \in I, f \in C^k\enbrace*{I}\]
        ein \(\RR\)-Vektorraum ist. Zeigen Sie weiter, dass \(\dotproduct{f}{g}_{H^k}\) ein Skalarprodukt auf \(C^k\enbrace*{I}\) ist.
    \end{problem}

    \subsection*{Lösung} \clearpage

    \section*{Aufgabe 8. (Unterräume)}

    \begin{problem}
        Entscheiden und begründen Sie, ob \(U\) ein Unterraum des \(\RR\)-Vektorraums \(V\) ist:
        \begin{enumerate}
            \item \(V = \RR^2\), \quad \(U = \penbrace*{\enbrace*{x, y}^T \in \RR^2\,\left|\,\begin{pmatrix}x\\y\end{pmatrix} = \begin{pmatrix}t\\2t\end{pmatrix}\right.}\)
            \item \(V = \RR^3\), \quad \(U = \penbrace*{\left.\enbrace*{x, y, z}^T \in \RR^3\,\right|\,y = 2x + 1}\)
            \item \(V = \RR^3\), \quad \(U = \penbrace*{\left.\enbrace*{x, y, z}^T \in \RR^3\,\right|\,y + 2x + 2z = 0}\)
            \item \(V = \mathcal{F}\enbrace*{\benbrace*{a, b}, \RR}\), \quad Raum aller Funktionen \(f: \benbrace*{a, b} \to \RR\)
            \begin{enumerate}[label=\roman*.]
                \item \(U_1 = \penbrace*{\left.f: \benbrace*{a, b} \to \RR\,\right|\,f\text{ ist Polynom vom Grad }\le n}\)
                \item \(U_2 = C^k\enbrace*{\benbrace*{a, b}}\)
                \item \(U_3 = C^\infty\enbrace*{\benbrace*{a, b}}\)
            \end{enumerate}
            Welcher Unterraum ist dabei wieder ein Unterraum der jeweils anderen Teilräume?
        \end{enumerate}
    \end{problem}

    \subsection*{Lösung} \clearpage

    \section*{Aufgabe 9. (Lineare Unabhängigkeit)}

    \begin{problem}
        \begin{enumerate}
            \item Untersuchen Sie, ob \(\penbrace*{f_1, f_2, f_3} \subset \mathcal{F}\enbrace*{\RR, \RR}\) definiert durch
            \[f_1\enbrace*{t} = \cos\enbrace*{t}, \quad
            f_2\enbrace*{t} = \cos\enbrace*{2t}, \quad
            f_3\enbrace*{t} = \cos\enbrace*{3t}, \quad t \in \RR,\]
            linear unabhängig ist.
            \item Sei \(\penbrace*{x, y, z} \subset V\) linear unabhängig. Ist
            \(\penbrace*{x + y, x + z, y + z}\) auch linear abhängig?
            \item Für welche Werte \(\lambda \in \RR\) ist
            \(\penbrace*{\enbrace*{\lambda, 1, 0}^T, \enbrace*{1, \lambda, 1}^T, \enbrace*{0, 1, \lambda}^T}\) linear abhängig?
        \end{enumerate}
    \end{problem}

    \subsection*{Lösung} \clearpage

    \section*{Aufgabe 10. (Basis und Dimension)}

    \begin{problem}
        Bestimmen Sie die Dimension und eine Basis von
        \begin{enumerate}
            \item \(V = \penbrace*{\left.p \in \mathcal{P}_2\,\right|\,p\enbrace*{0} = 0}\),
            \item \(V = \vecspan\penbrace*{1 - 2x, 2x - x^2, 1 - x^2, 1 + x^2}
            \subset \mathcal{P}_2\).
        \end{enumerate}
    \end{problem}

    \subsection*{Lösung} \clearpage

    \section*{Aufgabe 11. (Bestapproximation)}

    \begin{problem}
        Seien
        \[v_1 = \begin{pmatrix}1\\2\\0\end{pmatrix}, \quad
        v_2 = \begin{pmatrix}4\\3\\2\end{pmatrix}, \quad
        v_3 = \begin{pmatrix}0\\-1\\4\end{pmatrix}.\]
        Vektoren  im \(\RR^3\) und der Teilraum \(U = \vecspan\penbrace*{v_1, v_2}\) gegeben. Bestimmen Sie die Bestapproximation von \(v_3\) durch ein Element \(u^*\) des Teilraums \(U\).
    \end{problem}

    \subsection*{Lösung} \clearpage

    \section*{Aufgabe 12. (Lineare Abbildungen)}

    \begin{problem}
        Gegeben seien die Abbildungen
        \[\varphi: \RR^2 \to \RR^3, \penbrace*{x, y} \mapsto \begin{pmatrix}x - y\\2x\\y + 5x\end{pmatrix}\]
        und
        \[\phi: \RR^3 \to \RR, \penbrace*{x, y, z} \mapsto x + y + z.\]
        \begin{enumerate}
            \item Zeigen Sie die Linearität der Abbildungen \(\varphi\) und \(\phi\).
            \item Identifizieren Sie jeweils die zugehörige Matrix.
            \item Bilden Sie die Verknüpfung \(\phi \circ \varphi\) und geben Sie die Matrix an.
            \item Diskutieren Sie die Verknüpfung \(\varphi \circ \phi\).
        \end{enumerate}
    \end{problem}

    \subsection*{Lösung} \clearpage
\end{document}
