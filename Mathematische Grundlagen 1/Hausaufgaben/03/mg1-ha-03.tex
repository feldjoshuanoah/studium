\documentclass[german,12pt]{homework}

\usepackage[ngerman]{babel}
\usepackage[utf8]{inputenc}
\usepackage[T1]{fontenc}

\newcommand{\NN}{\mathbb{N}}
\newcommand{\ZZ}{\mathbb{Z}}
\newcommand{\QQ}{\mathbb{Q}}
\newcommand{\RR}{\mathbb{R}}
\newcommand{\LL}{\mathbb{L}}

\newcommand{\dotproduct}[2]{\left\langle#1, #2\right\rangle}

\DeclarePairedDelimiter{\absolute}{\lvert}{\rvert}
\DeclarePairedDelimiter{\norm}{\lVert}{\rVert}
\DeclarePairedDelimiter{\enbrace}{(}{)}
\DeclarePairedDelimiter{\benbrace}{[}{]}
\DeclarePairedDelimiter{\penbrace}{\{}{\}}

\title{Hausaufgabenübung 3}
\author{Joshua Feld, 406718 \quad Jeff Vogel, 407758 \quad Henrik Herrmann, 421853}
\date{23. November, 2020}
\institute{RWTH Aachen University\\Center for Computational Engineering Science}
\class{Mathematische Grundlagen I}
\professor{Prof. Dr. Torrilhon \& Prof. Dr. Stamm}

\begin{document}
    \maketitle

    \section*{Aufgabe 1. (Menge reeller Zahlen)}

    \begin{problem}
        \begin{enumerate}
            \item Bestimmen Sie das Maximum, Minimum, Infimum und Supremum der
            folgenden Menge:
            \[M_1 = \penbrace*{\frac{1}{n} + \frac{1}{m} : n, m \in \NN}.\]
            \item Bestimmen Sie die Menge aller \(x \in \RR\), für die gilt:
            \[\frac{x - 5}{x + 5} > 0 \land \absolute*{15x - 2} \ge 30\]
        \end{enumerate}
    \end{problem}

    \subsection*{Lösung}
    \begin{enumerate}
        \item Ein Bruch \(\frac{1}{k}\) mit \(k \in \NN\) ist streng monoton
        fallend, d.h. am Beginn des Definitionsberechs ist das Maximum zu
        finden. Daraus folgt
        \[\frac{1}{n} + \frac{1}{m} \le \frac{1}{1} + \frac{1}{1} = 2.\]
        Da \(2 \in M_1\) gilt \(\max{M_1} = \sup{M_1} = 2\). Des Weiteren gilt
        \(\lim_{k \to \infty}\frac{1}{k} = 0\) mit \(k \in \NN\), also auch
        \[\lim_{n, m \to \infty}\enbrace*{\frac{1}{n} + \frac{1}{m}} = 0.\]
        Wir wissen jedoch, dass \(0 \not\in M_1\), denn sonst müsste
        \(\frac{1}{m}\) das additive Inverse von \(\frac{1}{n}\) sein, was aber
        einen Widerspruch darstellt, da dazu \(m = -n\) gelten müsste, jedoch
        liegen die negativen Zahlen nicht in \(\NN\). Folglich gilt \(\inf{M_1}
        = 0\), aber es existiert kein Minimum.
        \item Um die erste Ungleichung zu lösen bestimmen wir zunächst die
        Nullstellen der Terme \(x - 5\) und \(x + 5\). Offensichtlich sind die
        Nullstellen hier \(x = 5\) und \(x = -5\). Damit erhalten wir die
        folgenden möglichen Lösungsintervalle:
        \[\LL_1 = \enbrace*{-\infty, -5}, \quad \LL_2 = \enbrace*{-5, 5} \quad
        \text{und} \quad \LL_3 = \enbrace*{5, \infty}.\]
        Durch Einsetzen von Werten überprüfen wir, welche Intervalle zur Lösung
        gehören.
        \begin{align*}
            -6 \in \LL_1: \quad &\frac{\enbrace*{-6} - 5}{-6 + 5} = \frac{-11}{-
            1} = 11 > 0\\
            0 \in \LL_2: \quad &\frac{0 - 5}{0 + 5} = -\frac{5}{5} = -1 \le 0\\
            6 \in \LL_3: \quad &\frac{6 - 5}{6 + 5} = \frac{1}{11} > 0
        \end{align*}
        Zusammenfassend gilt also
        \[\LL_A = \LL_1 \cup \LL_3 = \penbrace*{\left.x \in \RR\ \right|\ x < -
        5 \lor x > 5}.\]
        Wir wollen nun noch die Menge aller \(x \in \RR\) bestimmen, für die
        \(\absolute*{15x - 2} \ge 30\). Es gilt
        \begin{align*}
            \absolute*{15x - 2} \ge 30 &\iff \enbrace*{15x - 2}^2 \ge 900\\
            &\iff 225x^2 - 60x + 4 \ge 900\\
            &\iff x^2 - \frac{4}{15}x - \frac{896}{225} \ge 0
        \end{align*}
        Wir wollen nun die Nullstellen der quadratischen Gleichung \(x^2 -
        \frac{4}{15}x - \frac{896}{225} = 0\) finden. Diese sind:
        \begin{align*}
            &x^2 - \frac{4}{15}x - \frac{896}{225} = 0\\
            \iff&x = \frac{2}{15} \pm \sqrt{\frac{4}{225} + \frac{896}{225}} =
            \frac{2}{15} \pm 2\\
            \iff&x = \frac{32}{15} \lor x = -\frac{28}{15}
        \end{align*}
        Damit haben wir nun die möglichen Lösungsintervalle der quadratischen
        Ungleichung \(x^2 - \frac{4}{15}x - \frac{896}{225} \ge 0\) gefunden.
        Diese sind
        \[\LL_1 = \left(-\infty, -\frac{28}{15}\right], \quad \LL_2 =
        \benbrace*{-\frac{28}{15}, \frac{32}{15}} \quad \text{und} \quad \LL_3
        = \left[\frac{32}{15}, \infty\right)\]
        Durch Einsetzen von Werten überprüfen wir, welche Intervalle zur Lösung
        gehören.
        \begin{align*}
            -2 \in \LL_1: \quad &\enbrace*{-2}^2 - \frac{4}{15} \cdot
            \enbrace*{-2} - \frac{896}{225} = \frac{124}{225} \ge 0\\
            0 \in \LL_2: \quad &0^2 - \frac{4}{15} \cdot 0 - \frac{896}{225} = -
            \frac{896}{225} < 0\\
            3 \in \LL_3: \quad &3^2 - \frac{4}{15} \cdot 3 - \frac{896}{225} =
            \frac{949}{225} \ge 0
        \end{align*}
        Zusammenfassend gilt also
        \[\LL_B = \LL_1 \cup \LL_3 = \penbrace*{x \in \RR\ \left|\ x \le -
        \frac{28}{15} \lor x \ge \frac{32}{15}\right.}.\]
        Für die Menge aller \(x \in \RR\) für die beide Ungleichungen gelten,
        ergibt sich
        \[\LL = \LL_A \cap \LL_B = \LL_A = \penbrace*{\left.x \in \RR\ \right|\
        x < -5 \lor x > 5}.\]
    \end{enumerate}

    \section*{Aufgabe 2. (Supremum, Infimum, Maximum, Minimum)}

    \begin{problem}
        Bestimmen Sie Supremum und Infimum der folgenden Mengen. In welchen
        Fällen handelt es sich um ein Maximum bzw. Minimum? Begründen Sie,
        warum es sich jeweils um ein Supremum/Infimum/Maximum/Minimum handelt.
        \begin{enumerate}
            \item \(M_1 = \left[a, b\right)\), \(M_2 = \left(a, b\right]\) für
            \(a, b \in \RR´, a < b\),
            \item \(M_3 = \penbrace*{x \in \RR : x = \frac{1}{z}\text{ für }z
            \in \ZZ \setminus \penbrace*{0}}\),
            \item \(M_4 = \penbrace*{x \in \RR : x \ge 0\text{ und
            }x\enbrace*{x - 1}\enbrace*{x - 2} < 0}\).
        \end{enumerate}

        \textbf{Hinweis:} \quad \emph{Für eine Teilmenge \(A \subset \RR\) mit
        \(\sup\enbrace*{A} \in A\) heißt \(s := \sup\enbrace*{A}\) auch das
        Maximum von \(A\). Also \(s = \max\enbrace*{A}\). Falls
        \(\sup\enbrace*{A} \not\in A\), so hat die Menge kein Maximum. Analog
        folgt \(t := \inf\enbrace*{A} \in A \implies t = \min\enbrace*{A}\) und
        falls \(\inf{A} \not\in A\), dann hat die Menge kein Minimum.}
    \end{problem}

    \subsection*{Lösung}
    \begin{enumerate}
        \item \(b\) ist nach der Definition des Intervalls eine obere Schranke
        von \(M_1\). Da jedoch \(b \not\in M_1\) müssen wir noch zeigen, dass
        \(b\) wirklich die kleinste obere Schranke von \(M_1\) ist. Dies zeigen
        wir mithilfe eines Widerspruchs. Angenommen, es existiert ein \(b' =
        \sup{M_1}\) mit \(b' < b\). Da für alle \(0 < x < b - a\) gilt \(b - x
        \in M_1\), also
        \[b' \in M_1 \land b' = \sup{M_1} \implies b' = \max{M_1}.\]
        Daraus folgt aber auch \(b' + \frac{b - b'}{2} \in \left[a, b\right)\).
        Somit kann \(b'\) nicht Maximum von \(M_1\) sein, womit wir einen
        Widerspruch erreicht haben. \(a\) ist offensichtlich die größte untere
        Schranke, da sie in \(M_1\) enthalten ist. Insgesamt also
        \[\sup{M_1} = b \quad \text{und} \quad \inf{M_1} = \min{M_1} = a.\]
        Für \(M_2\) ist das Verhalten gespiegelt, d.h. es exisitiert ein
        Infimum mit \(\inf{M_2} = a\) aber kein Minimum, da \(a \not\in M_2\).
        \(b\) ist offensichtlich die größte obere Schranke, da sie in \(M_2\)
        enthalten ist. Insgesamt also
        \[\sup{M_2} = \max{M_2} = b \quad \text{und} \quad \inf{M_2} = a.\]
        \item Für alle \(x \in M_3, x = \frac{1}{z}\) mit \(z \in \ZZ \setminus
        \penbrace*{0}\) muss gelten \(\absolute*{x} \le 1\). Daraus folgt sofort
        \[\sup{M_3} = \max{M_3} = 1 \quad \text{und} \quad \inf{M_3} =
        \min{M_3} = -1.\]
        \item Wir formen die beiden Bedingungen für die \(x \in M_4\)
        gleichzeitig um:
        \begin{align*}
            x \ge 0 \land x\enbrace*{x - 1}\enbrace*{x - 2} < 0 &\iff x > 0
            \land \enbrace*{x - 1}\enbrace*{x - 2} < 0\\
            &\iff x > 0 \land \enbrace*{\enbrace*{x - 1} > 0 \land \enbrace*{x
            - 2} < 0} \lor \enbrace*{\enbrace*{x - 1} < 0 \land \enbrace*{x -
            2} > 0}\\
            & \iff x > 0 \land \enbrace*{\enbrace*{x > 1 \land x < 2} \lor
            \enbrace*{x < 1 \land x > 2}}\\
            & \iff x > 0 \land \enbrace*{x > 1 \land x < 2}\\
            & \iff x \in \enbrace*{1, 2}.
        \end{align*}
        Es gilt also \(\sup{M_4} = 2\) und \(\inf{M_4} = 1\). Minimum und
        Maximum exisitieren beide nicht, da das Intervall für \(x \in M_4\) in
        beide Richtungen offen ist und somit \(\sup{M_4}, \inf{M_4} \not\in
        M_4\).
    \end{enumerate}

    \section*{Aufgabe 3. (Reelle Zahlen)}

    \begin{problem}
        Seien \(a, b \in \RR\) und \(a < b\):
        \begin{enumerate}
            \item Sei \(w\) eine positive irrationale Zahl. Zeigen Sie, dass es
            eine rationale Zahl \(r \in \QQ\) gibt, so dass \(a < wr < b\) gilt.
            \item Zeigen Sie, dass \(wr\) auch eine irrationale Zahl ist. Daher
            existiert zwischen zwei verschiedenen reellen Zahlen immer auch
            eine irrationale Zahl.
            \item Sei \(\alpha = \sup\enbrace*{A} < \infty\) und \(\varepsilon
            > 0\). Zeigen Sie, dass eine Zahl \(x \in A\) existiert, so dass
            \(\alpha - \varepsilon < x\) gilt.
        \end{enumerate}
    \end{problem}

    \subsection*{Lösung}
    \begin{enumerate}
        \item Aus der Dichtheit der rationalen Zahlen, welche bereits in der
        Vorlesung bewiesen wurde, folgt, dass es ein \(r \in \QQ\) gibt, für
        dass gilt
        \[\frac{a}{w} < r < \frac{b}{w},\]
        denn offensichtlich sind \(\frac{a}{w}, \frac{b}{w} \in \RR\).
        Multiplizieren wir alles mit \(w\) erhalten wir, da \(w\) positiv ist,
        direkt
        \[a < wr < b,\] was zu zeigen war.
        \item Wir zeigen diese Aussage mithilfe eines Widerspruchs. Angenommen
        es gelte \(wr \in \QQ\). Dann gilt
        \[w \cdot \frac{m}{n} = \frac{x}{y} \iff w = \frac{xm}{yn}.\]
        Hiernach muss \(w\) eine rationale Zahl sein, was den Widerspruch
        darstellt. Folglich ist \(wr \in \RR \setminus \QQ\). Somit ist
        bewiesen, dass zwischen zwei reellen Zahlen auch immer eine irrationale
        Zahl existiert.
        \item Aus \(\alpha = \sup{A}\) folgt, dass \(\alpha\) die kleinste
        obere Schranke der Menge \(A\) ist. Somit gilt für eine reelle Zahl
        \(\varepsilon > 0\) immer \(\alpha - \epsilon < \alpha\). Daraus folgt,
        dass \( \alpha - \epsilon\) keine kleinste obere Schranke der Menge
        \(A\) sein kann, woraus wir folgern können, dass ein \(x \in A\)
        existiert, für das \(\alpha - \epsilon < x\) gilt.
    \end{enumerate}

    \section*{Aufgabe 4. (Bernoulli-Ungleichung)}

    \begin{problem}
        Beweisen Sie die \emph{Bernoulli-Ungleichung}: Für alle \(x \in \RR\)
        mit \(x > -1\) und für alle \(n \in \NN\) gilt:
        \[\enbrace*{1 + x}^n \ge 1 + nx.\]
    \end{problem}

    \subsection*{Lösung}
    \begin{itemize}
        \item Induktionsverankerung: (\(n = 1\))
        \[\enbrace*{1 + x}^1 = 1 + x \ge 1 + x.\]
        \item Induktionsvoraussetzung: Die Aussage gelte für alle \(x \in \RR\)
        mit \(x > -1\) und ein \(n \in \NN\).
        \item Induktionsschritt: (\(n \to \enbrace*{n + 1}\))
        \begin{align*}
            \enbrace*{1 + x}^{n + 1} &= \enbrace*{1 + x}^n \cdot \enbrace*{1 +
            x}\\
            &\underbrace{\ge}_\text{I.V.} \enbrace*{1 + nx} \cdot \enbrace*{1 +
            x}\\
            &= 1 + x + nx + nx^2\\
            &\ge 1 + x + nx = 1 + \enbrace*{n + 1}x.
        \end{align*}
    \end{itemize}
    Nach dem Prinzip der vollständigen Induktion gilt die Behauptung für alle
    \(x \in \RR\) mit \(x > -1\) und für alle \(n \in \NN\).

    \section*{Aufgabe 5. (Ungleichung)}

    \begin{problem}
        Seien \(x, y \in \RR\) beliebig. Beweisen Sie, dass gilt:
        \[2xy \le x^2 + y^2.\]
    \end{problem}

    \subsection*{Lösung} Seien \(x, y \in \RR\). Offensichtlich gilt \(0 \le
    \enbrace*{x - y}^2 = x^2 - 2xy + y^2\). Addieren wir auf jeder Seite
    \(2xy\), so erhalten wir \(2xy \le x^2 + y^2\).

    \section*{Aufgabe 6. (Cauchy-Schwarz-Ungleichung)}

    \begin{problem}
        Sei \(V\) ein Vektorraum und \(\dotproduct{\cdot}{\cdot}\) ein
        Skalarprodukt auf \(V\). Zeigen Sie, dass das Skalarprodukt die Cauchy-
        Schwarz-Ungleichung
        \[\absolute*{\dotproduct{x}{y}} \le \norm*{x} \cdot \norm*{y} \quad
        {\forall}x, y \in V\]
        erfüllt. Hierbei ist \(\norm*{\cdot}\) die durch
        \(\dotproduct{\cdot}{\cdot}\) \emph{induzierte Norm}, d.h. \(\norm*{x}
        := \sqrt{\dotproduct{x}{x}}\).

        \textbf{Hinweis:} \quad \emph{Berechnen Sie \(\dotproduct{x + \lambda
        \cdot y}{x + \lambda \cdot y}\) und wählen Sie anschließend \(\lambda
        \in \RR\) geeignet.}
    \end{problem}

    \subsection*{Lösung} Für den Fall \(x = y = 0\) Ist die Ungleichung
    trivial, denn dann gilt
    \[\absolute*{\dotproduct{0}{0}} = 0 \le 0 = \norm*{0} \cdot \norm*{0}.\]
    Seien \(x, y \in V\) und \(\lambda \in \RR\). Wir nehmen zunächst an, es
    gilt \(y \ne 0\). Wir betrachten den Ausdruck
    \begin{align*}
        \dotproduct{x + \lambda \cdot y}{x + \lambda \cdot y} &=
        \dotproduct{x}{x} + \dotproduct{\lambda{y}}{x} +
        \dotproduct{x}{\lambda{y}} + \dotproduct{\lambda{y}}{\lambda{y}}\\
        &= \dotproduct{x}{x} + 2\dotproduct{x}{\lambda{y}} +
        \dotproduct{\lambda{y}}{\lambda{y}}\\
        &= \dotproduct{x}{x} + 2\lambda\dotproduct{x}{y} +
        \lambda^2\dotproduct{y}{y}\\
        &= a + 2\lambda{b} + \lambda^2c
    \end{align*}
    mit \(a = \dotproduct{x}{x}\), \(b = \dotproduct{x}{y}\) und \(c =
    \dotproduct{y}{y}\). Es gilt \(a + 2\lambda{b} + \lambda^2c \ge 0\). Wir
    wählen \(\lambda = -\frac{\absolute*{b}}{c}\) und setzen dies ein
    \begin{align*}
        a - 2\frac{\absolute*{b}}{c}b + \enbrace*{-\frac{\absolute*{b}}{c}}^2c
        &\ge 0\\
        a - 2\frac{\absolute*{b}^2}{c} + \frac{\absolute*{b}^2}{c^2}c &\ge 0\\
        a - 2\frac{\absolute*{b}^2}{c} + \frac{\absolute*{b}^2}{c} &\ge 0\\
        a - \frac{\absolute*{b}^2}{c} &\ge 0
    \end{align*}
    Hieraus können wir nun durch Multiplikation mit \(c\) folgern, dass \(ax -
    \absolute*{b}^2 \ge 0\) beziehungsweise durch umformen \(\absolute*{b}^2
    \le ac\). Wir setzen nun unsere ursprünglichen Werte für \(a, b, c\) ein
    und erhalten
    \[\absolute*{\dotproduct{x}{y}}^2 \le \dotproduct{x}{x} \cdot
    \dotproduct{y}{y} = \norm*{x} \cdot \norm*{y}.\]
    Nehmen wir also nun es sei \(y = 0\) und \(x\) beliebig. Dann folgt, dass
    \(c = 0\) und somit auch \(\norm*{y} = 0\). Außerdem gilt auch
    \(\dotproduct{x}{y} = 0\). Folglich gilt die Ungleichung auch in diesem
    Fall:
    \[\absolute*{0}^2 = 0 \le 0 = \norm*{x} \cdot 0.\]
\end{document}
