\documentclass[german,12pt]{homework}

\usepackage[ngerman]{babel}
\usepackage[utf8]{inputenc}
\usepackage[T1]{fontenc}

\newcommand{\RR}{\mathbb{R}}
\newcommand{\PP}{\mathcal{P}}

\DeclarePairedDelimiter{\enbrace}{(}{)}

\title{Hausaufgabenübung 7}
\author{Joshua Feld, 406718 \quad Jeff Vogel, 407758 \quad Henrik Herrmann, 421853}
\date{21. Dezember, 2020}
\institute{RWTH Aachen University\\Center for Computational Engineering Science}
\class{Mathematische Grundlagen I}
\professor{Prof. Dr. Torrilhon \& Prof. Dr. Stamm}

\begin{document}
    \maketitle

    \section*{Aufgabe 1. (Konvergente Reihe und Potenzreihen)}

    \begin{problem}
        \begin{enumerate}
            \item Untersuchen Sie die folgende Reihe auf Konvergenz und
            bestimmen Sie gegebenenfalls den Grenzwert:
            \[\sum_{n = 0}^\infty\frac{i^n + 7}{3^n}, \quad i\text{: komplexe
            Einheit}.\]
            \item Bestimmen Sie den Konvergenzradius der folgenden Potenzreihe:
            \[\sum_{n = 0}^\infty\enbrace*{-1}^{n - 1}\frac{\enbrace*{2x}^n}
            {n^n}.\]
        \end{enumerate}
    \end{problem}

    \subsection*{Lösung}

    \section*{Aufgabe 2. (Konvergenz von Reihen)}

    \begin{problem}
        Gegeben sei die Reihe
        \[\sum_{k = 0}^\infty\frac{2 + \enbrace*{-1}^{k + 1}}{2^k}.\]
        Zeigen Sie, dass man mittels des Majorantenkriteriums und mittels des
        Wurzelkriteriums die Konvergenz der Reihe zeigen kann, jedoch nicht mit
        Hilfe des Quotientenkriteriums. Bestimmen Sie anschließend den Wert der
        Reihe.

        \textbf{Hinweis:} \quad \emph{Es gilt für \(a > 0\): \(\sqrt[n]{a}
        \xrightarrow{n \to \infty} 1\).}
    \end{problem}

    \subsection*{Lösung}

    \section*{Aufgabe 3. (Stetigkeit)}

    \begin{problem}
        Seien \(f: \RR_+ \to \RR\) und \(g: \RR \to \RR\) die Funktionen
        \begin{enumerate}
            \item \(f\enbrace*{x} = \sqrt{x}\){\quad}für alle \(x \in \RR_+\).
            \item \(g\enbrace*{x} = \begin{cases}
                1 & \text{für }x > 0,\\
                0 & \text{für }x = 0,\\
                -1 & \text{für }x < 0.
            \end{cases}\)
        \end{enumerate}
        Zeigen oder widerlegen Sie die Stetigkeit der Funktionen \(f\) und \(g\)
        im Nullpunkt.
    \end{problem}

    \subsection*{Lösung}

    \section*{Aufgabe 4. (Lineare Abbildung)}

    \begin{problem}
        Welche der folgenden Abbildungen sind linear? Geben Sie jeweils eine
        kurze Begründung an. Geben Sie für die linearen Abbildungen den Kern der
        Abbildung und dessen Dimension an.
        \begin{enumerate}
            \item \(f_1: \RR \to \RR^3 \quad f_1\enbrace*{x} = \begin{pmatrix}
            x + 1\\2x\\x - 3\end{pmatrix}\)
            \item \(f_2: \RR^4 \to \RR^2 \quad f_2\enbrace*{x_1, x_2, x_3, x_4}
            = \begin{pmatrix}x_1 + x_2\\x_1 + x_2 + x_3 + x_4\end{pmatrix}\)
            \item \(f_3: \RR^4 \to \RR^2 \quad f_3\enbrace*{x_1, x_2, x_3, x_4}
            = \begin{pmatrix}x_1x_2\\x_3x_4\end{pmatrix}\)
            \item \(f_4: \PP_3 \to \PP_3 \quad f_4\enbrace*{a_0 + a_1x + a_2x^2
            + a_3x^3} = a_1 + 2a_2x + 3a_3x^2\)
        \end{enumerate}
    \end{problem}

    \subsection*{Lösung}
\end{document}
