\documentclass[german,12pt]{homework}

\usepackage[ngerman]{babel}
\usepackage[utf8]{inputenc}
\usepackage[T1]{fontenc}

\newcommand{\NN}{\mathbb{N}}
\newcommand{\RR}{\mathbb{R}}
\newcommand{\PP}{\mathcal{P}}

\DeclarePairedDelimiter{\absolute}{\lvert}{\rvert}
\DeclarePairedDelimiter{\enbrace}{(}{)}
\DeclarePairedDelimiter{\benbrace}{[}{]}
\DeclarePairedDelimiter{\penbrace}{\{}{\}}

\title{Hausaufgabenübung 7}
\author{Joshua Feld, 406718 \quad Jeff Vogel, 407758 \quad Henrik Herrmann, 421853}
\date{21. Dezember, 2020}
\institute{RWTH Aachen University\\Center for Computational Engineering Science}
\class{Mathematische Grundlagen I}
\professor{Prof. Dr. Torrilhon \& Prof. Dr. Stamm}

\begin{document}
    \maketitle

    \section*{Aufgabe 1. (Konvergente Reihe und Potenzreihen)}

    \begin{problem}
        \begin{enumerate}
            \item Untersuchen Sie die folgende Reihe auf Konvergenz und
            bestimmen Sie gegebenenfalls den Grenzwert:
            \[\sum_{n = 0}^\infty\frac{i^n + 7}{3^n}, \quad i\text{: komplexe
            Einheit}.\]
            \item Bestimmen Sie den Konvergenzradius der folgenden Potenzreihe:
            \[\sum_{n = 0}^\infty\enbrace*{-1}^{n - 1}\frac{\enbrace*{2x}^n}
            {n^n}.\]
        \end{enumerate}
    \end{problem}

    \subsection*{Lösung}
    \begin{enumerate}
        \item
        \begin{align*}
            \sum_{n = 0}^\infty\frac{i^n + 7}{3^n} &= \sum_{n = 0}^\infty
            \frac{i^n}{3^n} + \sum_{n = 0}^\infty\frac{7}{3^n}\\
            &= \sum_{n = 0}^\infty\enbrace*{\frac{1}{3}i}^n + 7\sum_{n =
            0}^\infty\frac{1}{3^n}\\
            &\underbrace{=}_\text{geom. Reihe} \frac{1}{1 - \frac{1}{3}i} +
            7\sum_{n = 0}^\infty\enbrace*{\frac{1}{3}}^n\\
            &\underbrace{=}_\text{geom. Reihe} \frac{1 + \frac{1}{3}i}{1 +
            \frac{1}{9}} + 7 \cdot \frac{1}{1 - \frac{1}{3}}\\
            &= \frac{9}{10}\enbrace*{1 + \frac{1}{3}i} + \frac{21}{2}\\
            &= \frac{57}{5} + \frac{3}{10}i
        \end{align*}
        \item Es ist \(x_0 = 0\) und somit \(a_n = \enbrace*{-1}^{n - 1}\frac{2^n}{n^n} = \enbrace*{-1}^{n - 1}\enbrace*{\frac{2}{n}}^n\). Wir
        erhalten damit für den Konvergenzradius
        \begin{align*}
            \rho &= \frac{1}{\lim_{n \to \infty}\sqrt[n]{\enbrace*{-1}^{n - 1}
            \enbrace*{\frac{2}{n}}^n}}\\
            &= \frac{1}{\lim_{n \to \infty}\sqrt[n]{\enbrace*{-1}^{n + 1} \cdot
            2^n \cdot \enbrace*{\frac{1}{n}}^n}}\\
            &= \frac{1}{2\lim_{n \to \infty}\sqrt[n]{\enbrace*{-1}^{n + 1}
            \enbrace*{\frac{1}{n}}^n}}\\
            &= \frac{1}{2\lim_{n \to \infty}\sqrt[n]{\enbrace*{-1}} \cdot \frac{1}{n}}.
        \end{align*}
        Wir wissen, dass sich der Bruch \(\frac{1}{n}\) für \(n \to \infty\)
        immer weiter der Null annähert. Folglich nähert sich auch unser gesamter
        Nenner Null an und wir erhalten einen unendlichen Konvergenzradius
        \(\rho\).
    \end{enumerate}

    \section*{Aufgabe 2. (Konvergenz von Reihen)}

    \begin{problem}
        Gegeben sei die Reihe
        \[\sum_{k = 0}^\infty\frac{2 + \enbrace*{-1}^{k + 1}}{2^k}.\]
        Zeigen Sie, dass man mittels des Majorantenkriteriums und mittels des
        Wurzelkriteriums die Konvergenz der Reihe zeigen kann, jedoch nicht mit
        Hilfe des Quotientenkriteriums. Bestimmen Sie anschließend den Wert der
        Reihe.

        \textbf{Hinweis:} \quad \emph{Es gilt für \(a > 0\): \(\sqrt[n]{a}
        \xrightarrow{n \to \infty} 1\).}
    \end{problem}

    \subsection*{Lösung}
    \begin{itemize}
        \item Majorantenkriterium: Wir wählen als Majorante die geometrische Reihe \(\sum_{k = 0}^\infty{q}^k\) mit \(b_k = q^k\) und \(q \in \RR\).
        Von dieser wissen wir aus der Vorlesung, dass sie für \(k \to \infty\)
        gegen den Grenzwert \(\frac{1}{1 - q}\) konvergiert.
        \begin{align*}
            \absolute*{a_k} \le b_k \iff \absolute*{\frac{2 + \enbrace*{-1}^{k + 1}}{2^k}} \le b_k
        \end{align*}
        \[\frac{2 + \enbrace*{-1}^{k + 1}}{2^k} = \frac{2}{2^k} +
        \frac{\enbrace*{-1}^{k + 1}}{2^k} = \enbrace*{\frac{1}{2}}^{k - 1}
        - \enbrace*{-\frac{1}{2}}^k\]
        %\begin{align*}
        %    \sum_{k = 0}^\infty\frac{2 + \enbrace*{-1}^{k + 1}}{2^k}
        %    &= \sum_{k = 0}^\infty\frac{2}{2^k} + \sum_{k = 0}^\infty
        %    \frac{\enbrace*{-1}^{k + 1}}{2^k}\\
        %    &= \sum_{k = 0}^\infty\enbrace*{\frac{1}{2}}^{k - 1} -
        %    \sum_{k = 0}^\infty\enbrace*{-\frac{1}{2}}^k\\
        %    &=
        %\end{align*}
        \item Wurzelkriterium: Es gilt
        \begin{align*}
            \sqrt[k]{\absolute*{a_k}} = \sqrt[k]{\absolute*{\frac{2 + \enbrace*{-1}^{k + 1}}{2^k}}} &= \sqrt[k]{\absolute*{2^{-k} \cdot \enbrace*{2 + \enbrace*{-1}^{k + 1}}}}\\
            &= \frac{1}{2} \cdot \sqrt[k]{\absolute*{2 + \enbrace*{-1}^{k + 1}}}\\
        \end{align*}
        Es ist \(\enbrace*{-1}^{k + 1}\) \in \(\benbrace*{-1, 1}\), also ist der
        Teil unter der Wurzel immer größer als \(0\). Folglich konvergiert
        dieser nach dem Hinweis für \(k \to \infty\) gegen \(1\). Insgesamt
        konvergiert \(\sqrt[k]{\absolute*{a_k}}\) somit für \(k \to \infty\)
        gegen den Grenzwert \(\frac{1}{2}\). Nach dem Wurzelkriterium
        konvergiert die gesamte Reihe somit auch.
        \item Quotientenkriterium: Es gilt
        \[a_k = \frac{2 + \enbrace*{-1}^{k + 1}}{2^k} \qquad
        a_{k + 1} &= \frac{2 + \enbrace*{-1}^{k + 2}}{2^{k + 1}}\]
        \begin{align*}
            \absolute*{\frac{a_{k + 1}}{a_k}} = \absolute*{\frac{\frac{2 +
            \enbrace*{-1}^{k + 2}}{2^{k + 1}}}{\frac{2 + \enbrace*{-1}^{k + 1}}{2^k}}} &= \absolute*{\frac{\enbrace*{2 + \enbrace*{-1}^{k + 2}} \cdot 2^k}{2^{k + 1} \cdot \enbrace*{2 + \enbrace*{-1}^{k + 1}}}}\\
            &= \absolute*{\frac{2 + \enbrace*{-1}^{k + 2}}{2 \cdot \enbrace*{2 + \enbrace*{-1}^{k + 1}}}}\\
            &= \absolute*{\frac{2}{2 \cdot \enbrace*{2 + \enbrace*{-1}^{k + 1}}} + \frac{\enbrace*{-1}^{k + 2}}{2 \cdot \enbrace*{2 + \enbrace*{-1}^{k + 1}}}}\\
            &= \absolute*{\frac{1}{2 + \enbrace*{-1}^{k + 1}} + \frac{\enbrace*{-1}^k}{2 \cdot \enbrace*{2 - \enbrace*{-1}^k}}}\\
            &= \absolute*{\frac{1}{2 + \enbrace*{-1}^{k + 1}} + \frac{1}{2 + \enbrace*{-1}^{k + 1}} - \frac{1}{2}}\\
            &= \absolute*{\frac{2}{2 + \enbrace*{-1}^{k + 1}} - \frac{1}{2}} = \absolute*{-\frac{2}{2 + \enbrace*{-1}^k} - \frac{1}{2}}
        \end{align*}
        Wir können sehen, dass der Nenner des Bruchs divergiert, denn der Wert
        wechselt immer zwischen \(1\) und \(3\) (für ungerade und gerade \(k\) respektive). Somit alterniert unsere gesamte Folge zwischen den Werten
        \(\frac{1}{6}\) und \(\frac{3}{2}\). Da kein Grenzwert für \(\absolute*{\frac{a_{k + 1}}{a_k}}\) existiert, können wir mit dem Quotientenkriterium keine Aussage über die Konvergenz der ursprünglichen Reihe tätigen.
    \end{itemize}


    \section*{Aufgabe 3. (Stetigkeit)}

    \begin{problem}
        Seien \(f: \RR_+ \to \RR\) und \(g: \RR \to \RR\) die Funktionen
        \begin{enumerate}
            \item \(f\enbrace*{x} = \sqrt{x}\){\quad}für alle \(x \in \RR_+\).
            \item \(g\enbrace*{x} = \begin{cases}
                1 & \text{für }x > 0,\\
                0 & \text{für }x = 0,\\
                -1 & \text{für }x < 0.
            \end{cases}\)
        \end{enumerate}
        Zeigen oder widerlegen Sie die Stetigkeit der Funktionen \(f\) und \(g\)
        im Nullpunkt.
    \end{problem}

    \subsection*{Lösung}
    \begin{enumerate}
        \item Wir wissen aus der letzten Hausaufgabenübung, dass die
        Wurzelfunktion stetig ist. Dies wollen wir nun formal für den Nullpunkt
        zeigen. Sei \(\enbrace*{a_n}_{n \in \NN}\) in \(\RR_+\) eine Folge mit
        \(\lim_{n \to \infty}a_n = 0\). Wir wollen nun zeigen, dass
        \[\lim_{n \to \infty}\sqrt{a_n} = \sqrt{0}.\]
        Sei \(\varepsilon > 0\) dazu nun beliebig aber fest. Dann existiert ein
        \(N \in \NN\), sodass für alle \(n \ge N\) gilt \(a_n < \varepsilon^2\).
        Dies können wir umformen zu
        \[{\forall}\varepsilon > 0{\exists}N = N\enbrace*{\varepsilon} \in \NN
        {\forall}n \ge N: \sqrt{a_n} < \varepsilon.\]
        Also gilt \(\lim_{n \to \infty}\sqrt{a_n} = 0 = \sqrt{0}\) und somit ist
        die Funktion \(f\) stetig.
        \item Die Funktion \(g\) ist nicht stetig im Nullpunkt. Dies wollen wir
        mithilfe eines Widerspruchsbeweises zeigen. Wir nehmen also an, \(g\)
        wäre stetig im Nullpunkt.
    \end{enumerate}

    \section*{Aufgabe 4. (Lineare Abbildung)}

    \begin{problem}
        Welche der folgenden Abbildungen sind linear? Geben Sie jeweils eine
        kurze Begründung an. Geben Sie für die linearen Abbildungen den Kern der
        Abbildung und dessen Dimension an.
        \begin{enumerate}
            \item \(f_1: \RR \to \RR^3 \quad f_1\enbrace*{x} = \begin{pmatrix}
            x + 1\\2x\\x - 3\end{pmatrix}\)
            \item \(f_2: \RR^4 \to \RR^2 \quad f_2\enbrace*{x_1, x_2, x_3, x_4}
            = \begin{pmatrix}x_1 + x_2\\x_1 + x_2 + x_3 + x_4\end{pmatrix}\)
            \item \(f_3: \RR^4 \to \RR^2 \quad f_3\enbrace*{x_1, x_2, x_3, x_4}
            = \begin{pmatrix}x_1x_2\\x_3x_4\end{pmatrix}\)
            \item \(f_4: \PP_3 \to \PP_3 \quad f_4\enbrace*{a_0 + a_1x + a_2x^2
            + a_3x^3} = a_1 + 2a_2x + 3a_3x^2\)
        \end{enumerate}
    \end{problem}

    \subsection*{Lösung}
    \begin{enumerate}
        \item Seien \(x_1 = 0, x_2 = 1 \in \RR\). Dann gilt
        \[f_1\enbrace*{x_1 + x_2} = \begin{pmatrix}2\\2\\-2
        \end{pmatrix} \ne \begin{pmatrix}3\\2\\-5\end{pmatrix} = \begin{pmatrix}
        1\\0\\-3\end{pmatrix} + \begin{pmatrix}2\\2\\-2\end{pmatrix} =
        f_1\enbrace*{x_1} + f_1\enbrace*{x_2}.\]
        Da die Additivität nicht erfüllt ist, ist \(f_1\) nicht linear.
        \item Seien \(\lambda, \mu \in \RR\) und \(v, w \in \RR^4\).
        Dann gilt
        \begin{align*}
            f_2\enbrace*{\lambda{v} + \mu{w}} &= \begin{pmatrix}
                \lambda{v_1} + \lambda{v_2} + \mu{w_1} + \mu{w_2}\\
                \lambda{v_1} + \lambda{v_2} + \lambda{v_3} + \lambda{v_4}
                + \mu{w_1} + \mu{w_2} + \mu{w_3} + \mu{w_4}
            \end{pmatrix}\\
            &= \begin{pmatrix}
                \lambda\enbrace*{v_1 + v_2} + \mu\enbrace*{w_1 + w_2}\\
                \lambda\enbrace*{v_1 + v_2 + v_3 + v_4}
                + \mu\enbrace*{w_1 + w_2 + w_3 + w_4}
            \end{pmatrix}\\
            &= \lambda \cdot \begin{pmatrix}
                v_1 + v_2\\
                v_1 + v_2 + v_3 + v_4
            \end{pmatrix} + \mu \cdot \begin{pmatrix}
                w_1 + w_2\\
                w_1 + w_2 + w_3 + w_4
            \end{pmatrix}\\
            &= \lambda \cdot f_2\enbrace*{v} + \mu \cdot f_2\enbrace*{w}.
        \end{align*}
        Da sowohl Additivität als auch Homogenität erfüllt sind, ist die
        Abbildung linear. Für den Kern von \(f_2\) gilt
        \begin{align*}
            \ker\enbrace*{f_2} &= \penbrace*{\left.v \in \RR^4\,\right|\,
            f_2\enbrace*{v} = 0 \in \RR^2}\\
            &= \penbrace*{\left.v \in \RR^4\,\right|\,v_1 + v_2 = 0 \land
            v_1 + v_2 + v_3 + v_4 = 0}\\
            &= \penbrace*{v \in \RR^4\,\left|\,v = \begin{pmatrix}v_1\\-v_1\\
            v_3\\-v_3\end{pmatrix}\right.}.
        \end{align*}
        Es folgt, dass \(\dim\enbrace*{\ker\enbrace*{f_2}} = 2\), denn jedes
        \(v \in \ker\enbrace*{f_2}\) ist durch die folgende Linearkombination
        aus zwei Vektoren des \(\RR^4\) darstellbar:
        \[\lambda_1 \cdot \begin{pmatrix}1\\-1\\0\\0\end{pmatrix} + \lambda_2
        \cdot \begin{pmatrix}0\\0\\1\\-1\end{pmatrix} \quad \text{für }
        \lambda_1, \lambda_2 \in \RR.\]
        \item Sei \(\lambda = 3\) und \(v = \begin{pmatrix}1\\1\\1\\1
        \end{pmatrix} \in \RR^4\). Dann gilt
        \[f_3\enbrace*{\lambda{v}} = \begin{pmatrix}3 \cdot 1 \cdot 3 \cdot 1\\
        3 \cdot 1 \cdot 3 \cdot 1 \end{pmatrix} = \begin{pmatrix}9\\9
        \end{pmatrix} \ne \begin{pmatrix}3\\3\end{pmatrix} = 3 \cdot
        \begin{pmatrix}1 \cdot 1\\1 \cdot 1\end{pmatrix} = \lambda \cdot
        f_3\enbrace*{v}.\]
        Da Homogenität für diese Abbildung nicht erfüllt ist, ist sie nicht
        linear.
        \item Seien \(p, q \in \PP_3\) mit \(p\enbrace*{x} = \sum_{i = 0}^3
        \alpha_ix^i\) und \(q\enbrace*{x} = \sum_{i = 0}^3\beta_ix^i\) und
        \(\lambda, \mu \in \RR\). Dann gilt
        \begin{align*}
            f\enbrace*{\lambda{p\enbrace*{x}} + \mu{q\enbrace*{x}}} &=
            f\enbrace*{\lambda{\alpha_0} + \mu{\beta_0} +
            \enbrace*{\lambda{\alpha_1} + \mu{\beta_1}}x +
            \enbrace*{\lambda{\alpha_2} + \mu{\beta_2}}x^2 +
            \enbrace*{\lambda{\alpha_3} + \mu{\beta_3}}x^3}\\
            &= \lambda{\alpha_1} + \mu{\beta_1} + 2\enbrace*{\lambda{\alpha_2}
            + \mu{\beta_2}}x + 3\enbrace*{\lambda{\alpha_3} + \mu{\beta_3}}x^2\\
            &= \lambda\enbrace*{\alpha_1 + 2\alpha_2x + 3\alpha_3x^2} +
            \mu\enbrace*{\beta_1 + 2\beta_2x + 3\beta_3x^2}\\
            &= \lambda{f\enbrace*{p\enbrace*{x}}} +
            \mu{f\enbrace*{q\enbrace*{x}}}
        \end{align*}
        Da Additivität und Homogenität erfüllt sind, ist die Abbildung linear.
        Für den Kern der Abbildung gilt
        \begin{align*}
            \ker\enbrace*{f_3} &= \penbrace*{\left.p \in \PP_3\,\right|\,
            f_3\enbrace*{p\enbrace*{x}} = 0}\\
            &= \penbrace*{\left.p \in \PP_3\,\right|\,f_3\enbrace*{\alpha_0 +
            \alpha_1x + \alpha_2x^2 + \alpha_3x^3} = 0}\\
            &= \penbrace*{\left.p \in \PP_3\,\right|\,\alpha_1 + 2\alpha_2x +
            3\alpha_3x^2 = 0\text{ für alle }x \in \RR}\\
            &= \penbrace*{\left.p \in \PP_3\,\right|\,\alpha_1 = \alpha_2 =
            \alpha_3 = 0}\\
            &= \penbrace*{\left.p \in \PP_3\,\right|\,p\enbrace*{x} =
            \alpha_0} = \PP_0.
        \end{align*}
        Der Kern enthält also alle Polynome mit Grad \(0\). Daraus können wir
        direkt folgern, dass \(\dim\enbrace*{\ker\enbrace*{f_4} = 1}\) für die
        Dimension des Kerns gilt.
    \end{enumerate}
\end{document}
