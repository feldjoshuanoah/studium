\documentclass[german,12pt]{homework}

\usepackage[ngerman]{babel}
\usepackage[utf8]{inputenc}
\usepackage[T1]{fontenc}

\newcommand{\NN}{\mathbb{N}}
\newcommand{\ZZ}{\mathbb{Z}}

\DeclareMathOperator{\pot}{\mathcal{P}}

\DeclarePairedDelimiter{\enbrace}{(}{)}
\DeclarePairedDelimiter{\benbrace}{\{}{\}}
\DeclarePairedDelimiter{\penbrace}{\{}{\}}

\title{Hausaufgabenübung 1}
\author{Joshua Feld, 406718 \quad Jeff Vogel, 407758 \quad Henrik Herrmann, 421853}
\date{9. November, 2020}
\institute{RWTH Aachen University\\Center for Computational Engineering Science}
\class{Mathematische Grundlagen I}
\professor{Prof. Dr. Torrilhon \& Prof. Dr. Stamm}

\begin{document}
    \maketitle

    \section*{Aufgabe 1. (Morgansche Regeln)}

    \begin{problem}
        Beweisen Sie die Regeln von de Morgan für beliebige Mengen \(M, N, P\):
        \begin{enumerate}
            \item \(M \setminus \enbrace*{N \cap P} = \enbrace*{M \setminus N}
            \cup \enbrace*{M \setminus P}\),
            \item \(M \setminus \enbrace*{N \cup P} = \enbrace*{M \setminus N}
            \cap \enbrace*{M \setminus P}\).
        \end{enumerate}

        \textbf{Hinweis:} \quad \emph{Zeigen Sie die Gleichheit durch
        gegenseitige Inklusion \(\enbrace*{A = B} :\iff \enbrace*{A \subset B
        \land B \subset A}\). Achten Sie in Ihrer lösung auf korrekte und
        saubere Notation.}
    \end{problem}

    \subsection*{Lösung}
    \begin{enumerate}
        \item Sei \(x \in M \setminus \enbrace*{N \cap P}\) beliebig. Dann gilt
        \begin{align*}
            x \in M \setminus \enbrace*{N \cap P} &\iff x \in M \land x \not\in
            N \cap P\\
            &\iff x \in M \land \enbrace*{x \not\in N \lor x \not\in P}\\
            &\iff \enbrace*{x \in M \land x \not\in N} \lor \enbrace*{x \in M
            \land x \not\in P}\\
            &\iff x \in M \setminus N \lor x \in M \setminus P\\
            &\iff x \in \enbrace*{M \setminus N} \cup \enbrace*{M \setminus P}.
        \end{align*}
        \item Sei \(x \in M \setminus \enbrace*{N \cup P}\) beliebig. Dann gilt
        \begin{align*}
            x \in M \setminus \enbrace*{N \cup P} &\iff x \in M \land x \not\in
            N \cup P\\
            &\iff x \in M \land \enbrace*{x \not\in N \land x \not\in P}\\
            &\iff \enbrace*{x \in M \land x \not\in N} \land \enbrace*{x \in M
            \land x \not\in P}\\
            &\iff x \in M \setminus N \land x \in M \setminus P\\
            &\iff x \in \enbrace*{M \setminus N} \cap \enbrace*{M \setminus P}.
        \end{align*}
    \end{enumerate}

    \section*{Aufgabe 2. (Abbildungen)}

    \begin{problem}
        Die Aussagenlogik bildet die Grundlage der Digitalelektronik: Es lassen
        sich beliebig komplizierte Schaltungen mit einer einzigen Art von
        Bauelement, dem sogenannten NAND-Gatter (NAND \(=\) not and),
        realisieren. Das NAND \(\uparrow\) ist definiert durch
        \[\enbrace*{A \uparrow B} :\iff \lnot\enbrace*{A \land B}.\]
        Drücken Sie folgende aussagenlogische Formeln durch äquivalente NANDs
        aus und begründen Sie Ihre Antwort:
        \begin{enumerate}
            \item \({\lnot}A\),
            \item \(A \lor B\),
            \item \(A \land B\).
        \end{enumerate}

        \textbf{Hinweis:} \quad Verwenden Sie nur \(\uparrow\) und keinen der
        anderen logischen Operatoren (\(\lor, \land, \lnot\)).
    \end{problem}

    \subsection*{Lösung}
    \begin{enumerate}
        \item Es gilt \(\lnot{A} \iff \lnot\enbrace*{A \land A} \iff A \uparrow
        A\).
        \item Wir starten mit \(A \uparrow B \iff \lnot\enbrace*{A \land B}\).
        Hier können wir die De-morganschen Regeln für Aussagen anwenden und
        erhalten \(\lnot{A} \lor \lnot{B}\). Die Negation haben wir bereits in
        Teilaufgabe a) gezeigt und wir erhalten somit \(\enbrace*{A \uparrow A}
        \uparrow \enbrace*{B \uparrow B}\).
        \item Es gilt \(A \uparrow B \iff \lnot\enbrace*{A \land B}\). Wir
        wollen diese Aussage nun negieren. Dies haben wir schon in Teilaufgabe
        a) gezeigt und erhalten somit \(\enbrace*{A \uparrow B} \uparrow \enbrace*{A \uparrow B}\).
    \end{enumerate}

    \section*{Aufgabe 3. (Relationen)}

    \begin{problem}
        Untersuchen Sie die folgenden Relationen auf Reflexivität, Symmetrie,
        Transitivität, Antisymmetrie und Totalität. Hier bezeichne \(\NN =
        \penbrace*{1, 2, 3, \ldots}\) die Menge der natürlichen Zahlen.
        \begin{enumerate}
            \item \(=\) auf \(\NN\),
            \item \(\ne\) auf \(\NN\),
            \item \(\le\) auf \(\NN\),
            \item \(<\) auf \(\NN\),
            \item \(|\) auf \(\NN\) (Teilbarkeit: \(a | b \iff \exists{n \in
            \NN}: an = b\)),
            \item \(\subseteq\) auf \(\pot\enbrace*{\penbrace*{1, 2}}\).
        \end{enumerate}
        Entscheiden Sie jeweils, ob es sich um eine Äquivalenzrelation und/oder
        eine (Total-) Ordnung handelt.
    \end{problem}

    \subsection*{Lösung}
    \begin{enumerate}
        \item\ \\
        \begin{itemize}
            \item Reflexivität: \({\forall}x \in \NN: x = x\). (ja)
            \item Symmetrie: Seien \(x, y \in \NN\) mit \(x = y \implies y =
            x\). (ja)
            \item Transitivität: Seien \(x, y, z \in \NN\) mit \(x = y\) und
            \(y = z\) \(\implies\) \(x = z\). (ja)
            \item Antisymmetrie: Seien \(x, y \in \NN\) mit \(x = y\) und \(y =
            x\) \(\implies\) \(x = y\). (ja)
            \item Totalität: Sei \(x = 1\) und \(y = 2\), dann gilt weder \(x =
            y\) noch \(y = x\). (nein)
    	\end{itemize}
    	Die Relation ist eine Äquivalenzrelation und eine Ordnung aber keine
        Totalordnung.
        \item\ \\
        \begin{itemize}
            \item Reflexivität: Sei \(x = 2 \in \NN\), dann gilt \(x \ne x\)
            nicht. (nein)
            \item Symmetrie: Seien \(x, y \in \NN\) mit \(x \ne y\)
            \(\implies\) \(y \ne x\). (ja)
            \item Transitivität: Sei \(x = 1, y = 2\) und \(z = 1\), dann gilt
            \(x \ne y\) und \(y \ne z\) aber nicht \(x \ne z\). (nein)
            \item Antisymmetrie: Sei \(x = 1\) und \(y = 2\), dann gilt \(x \ne
            y\) und \(y \ne x\). (nein)
            \item Totalität: Sei \(x = 1\) und \(y = 1\), dann gilt weder \(x
            \ne y\) noch \(y \ne x\). (nein)
    	\end{itemize}
    	Die Relation ist weder eine Äquivalenzrelation noch eine Ordnung, also
        folglich auch keine Totalordnung.
        \item\ \\
        \begin{itemize}
            \item Reflexivität: \({\forall}x \in \NN: x \le x\). (ja)
            \item Symmetrie: Sei \(x = 1\) und \(y = 2\), dann gilt \(x \le y\)
            aber \(y > a\). (nein)
            \item Transitivität: Seien \(x, y, z \in \NN\) mit \(x \le y\) und
            \(y \le z\) \(\implies\) \(x \le z\). (ja)
            \item Antisymmetrie: Seien \(x, y \in \NN\) mit \(x \le y\) und \(y
            \le x\) \(\implies\) \(x = y\). (ja)
            \item Totalität: Seien \(x, y \in \NN\), dann gilt \(x \le y\) oder
            \(y \le x\). (ja)
    	\end{itemize}
    	Die Relation ist keine Äquivalenzrelation aber eine Totalordnung also
        folglich auch eine Ordnung.
        \item\ \\
        \begin{itemize}
            \item Reflexivität: Sei \(x = 1\), dann gilt nicht \(x < x\). (nein)
            \item Symmetrie: Sei \(x = 1\) und \(y = 2\), dann gilt \(x < y\)
            aber nicht \(y < x\). (nein)
            \item Transitivität: Seien \(x, y, z \in \NN\) mit \(x < y\) und
            \(y < z\) \(\implies\) \(x < z\). (ja)
            \item Antisymmetrie: Für \(x, y \in \NN\) kann nicht gleichzeitig
            \(x < y\) und \(y < x\) gelten. Da die Voraussetzung nicht gilt ist
            die Aussage immer wahr. (ja)
            \item Totalität: Sei \(x = 1\) und \(y = 1\), dann gilt weder \(x <
            y\) noch \(y < x\). (nein)
    	\end{itemize}
    	Die Relation ist weder eine Äquivalenzrelation noch eine Ordnung, also
        folglich auch keine Totalordnung.
        \item\ \\
        \begin{itemize}
            \item Reflexivität: \({\forall}x \in \NN: x | x\). (ja)
            \item Symmetrie: Sei \(x = 1\) und \(y = 2\), dann gilt \(x | y\)
            aber nicht \(y | x\). (nein)
            \item Transitivität: Seien \(x, y, z \in \NN\) mit \(x | y\) und
            \(y | z\), dann existieren \(m, n \in \NN\) für die gilt \(x \cdot
            m = y\) und \(y \cdot n = z\). Also ist \(z = y \cdot n =
            \enbrace*{x \cdot m} \cdot n = x \cdot mn\). Da \(mn \in \NN\),
            gilt \(x | z\). (ja)
            \item Antisymmetrie: Seien \(x, y \in \NN\) mit \(x | y\) und \(y |
            x\), dann existieren \(m, n \in \NN\) für die gilt \(x \cdot m =
            y\) und \(y \cdot n = x\). Damit gilt \(x = y \cdot n = \enbrace*{x
            \cdot m} \cdot n\). Hieraus folgt, dass \(m = n = 1\) gelten muss,
            da \(m, n \in \NN\). Setzen wir ein erhalten wir direkt \(x = y\).
            (ja)
            \item Totalität: Sei \(x = 2\) und \(y = 3\), dann gilt weder \(x |
            y\) noch \(y | x\). (nein)
    	\end{itemize}
    	Die Relation ist keine Äquivalenzrelation. Die Relation ist eine
        Ordnung aber keine Totalordnung.
        \item\ \\
        \begin{itemize}
            \item Reflexivität: \({\forall}M \in \pot\enbrace*{\penbrace*{1,
            2}}: M \subseteq M\). (ja)
            \item Symmetrie: Es gilt \(\emptyset \subseteq \penbrace*{1}\) aber
            \(\penbrace*{1} \not\subseteq \emptyset\). (nein)
            \item Transitivität: Seien \(M, N, O \in
            \pot\enbrace*{\penbrace*{1, 2}}\) für die gilt \(M \subseteq N\)
            und \(N \subseteq O\), dann folgt direkt \(M \subseteq O\). (ja)
            \item Antisymmetrie: Seien \(M, N \in \pot\enbrace*{\penbrace*{1,
            2}}\) für die gilt \(M \subseteq N\) und \(N \subseteq M\), dann
            folgt \(M = N\). (ja)
            \item Totalität: Für \(\{1\}, \{2\} \in \pot\enbrace*{\penbrace*{1,
            2}}\) gilt weder \(\penbrace*{1} \subseteq \penbrace*{2}\) noch
            \(\penbrace*{2} \subseteq \penbrace*{1}\). (nein)
    	\end{itemize}
    	Die Relation ist keine Äquivalenzrelation. Die Relation ist eine
        Ordnung aber keine Totalordnung.
    \end{enumerate}

    \section*{Aufgabe 4. (Äquivalenzrelation)}

    \begin{problem}
        \begin{enumerate}
            \item Sei \(M = \penbrace*{\frac{m}{n}: m, n \in \ZZ, n \ne 0}\)
            die Menge aller Brüche. Zeigen Sie, dass durch
            \[\frac{a}{b} \sim \frac{c}{d} \iff bc = ad\]
            eine Äquivalenzrelation auf \(M\) definiert ist.
            \item Beschreiben Sie möglichst genau die Menge aller
            Äquivalenzklassen, in die \(M\) bezüglich \(\sim\) zerfällt.
        \end{enumerate}
    \end{problem}

    \subsection*{Lösung}
    \begin{enumerate}
        \item Wir müssen zeigen, dass die Relation reflexiv, symmetrisch und
        transitiv ist.
        \begin{itemize}
    		\item Reflexivität: \({\forall}x = \frac{m}{n} \in M: x \sim x \iff
            nm = mn\).
    		\item Symmetrie: Seien \(x = \frac{a}{b}, y = \frac{c}{d} \in M\).
            Dann gilt
    		\[x \sim y \iff bc = ad \iff da = cb \iff y \sim x.\]
    		\item Transitivität: Seien \(x = \frac{a}{b}, y = \frac{c}{d}, z =
            \frac{e}{f} \in M\) für die gilt \(x \sim y\) und \(y \sim z\).
            Dann gilt
    		\begin{align*}
    			x \sim y \land y \sim z &\iff bc = ad \land de = cf\\
    			&\iff c = \frac{ad}{b} \land de = cf\\
    			&\iff de = \frac{ad}{b}f\\
    			&\iff bde = adf\\
    			&\iff be = af \iff x \sim z
    		\end{align*}
    	\end{itemize}
    	Da \(\sim\) reflexiv, symmetrisch und transitiv ist, handelt es sich um
        eine Äquivalenzrelation auf \(M\).
        \item \(M\) zerfällt bezüglich \(\sim\) in eine Menge von
        Äquivalenzklassen. Jede Äquivalenzklasse wird durch einen Bruch, der
        sich nicht weiter kürzen lässt repräsentiert. Sei \(A_M\) die Menge
        aller Äquivalenzklassen, dann ist
    	\[A_M = \penbrace*{\left.\benbrace*{\frac{m}{n}}_\sim\ \right|\ m, n
        \in \mathbb{Z}, n \ne 0 \land \text{ggT}\enbrace*{m, n} = 1}.\]
    \end{enumerate}
\end{document}
