\documentclass[german,12pt]{homework}

\usepackage[ngerman]{babel}
\usepackage[utf8]{inputenc}
\usepackage[T1]{fontenc}

\usepackage{graphicx}

\newcommand{\NN}{\mathbb{N}}
\newcommand{\RR}{\mathbb{R}}
\newcommand{\PP}{\mathcal{P}}

\DeclarePairedDelimiter{\absolute}{\lvert}{\rvert}
\DeclarePairedDelimiter{\enbrace}{(}{)}
\DeclarePairedDelimiter{\benbrace}{[}{]}
\DeclarePairedDelimiter{\penbrace}{\{}{\}}

\title{Hausaufgabenübung 8}
\author{Joshua Feld, 406718 \quad Jeff Vogel, 407758 \quad Henrik Herrmann, 421853}
\date{11. Januar, 2021}
\institute{RWTH Aachen University\\Center for Computational Engineering Science}
\subject{Mathematische Grundlagen I}
\professor{Prof. Dr. Torrilhon \& Prof. Dr. Stamm}

\begin{document}
    \maketitle

    \section*{Aufgabe 1. (Stetigkeit)}

    \begin{problem}
        Zeigen Sie die Stetigkeit der Funktion
        \[h: \RR \to \RR, h\enbrace*{x} = \frac{1}{1 + \lvert{x}\rvert}\]
        mittels der \(\varepsilon\)-\(\delta\)-Definition. Ist \(h\) auch gleichmäßig stetig?
    \end{problem}

    \subsection*{Lösung} Sei \(\varepsilon > 0\) beliebig und sei \(\delta = \delta\enbrace*{\varepsilon} = \varepsilon\). Wenn \(\absolute*{x - x_0} < \delta\) erfüllt ist, so folgt:
    \begin{align*}
        \absolute*{h\enbrace*{x} - h\enbrace*{x_0}} &= \absolute*{\frac{1}{1 + \absolute*{x}} - \frac{1}{1 + \absolute*{x_0}}}\\
        &= \absolute*{\frac{1 + \absolute*{x_0}}{\enbrace*{1 + \absolute*{x}}\enbrace*{1 + \absolute*{x_0}}} - \frac{1 + \absolute*{x}}{\enbrace*{1 + \absolute*{x_0}}\enbrace*{1 + \absolute*{x}}}}\\
        &= \absolute*{\frac{\absolute*{x_0} - \absolute*{x}}{\enbrace*{1 + \absolute*{x}}\enbrace*{1 + \absolute*{x_0}}}} \le \absolute*{\frac{\absolute*{x - x_0}}{\enbrace*{1 + \absolute*{x}}\enbrace*{1 + \absolute*{x_0}}}}\\
        &< \frac{\delta}{\enbrace*{1 + \absolute*{x}}\enbrace*{1 + \absolute*{x_0}}} \le \delta = \varepsilon.
    \end{align*}
    Damit ist gezeigt, dass \(f\) stetig ist. \(f\) ist auch gleichmäßig stetig, da \(\delta\) nicht von \(x_0\) abhängig ist.

    \section*{Aufgabe 2. (Lipschitz-Stetigkeit)}

    \begin{problem}
        Zeigen Sie, dass folgende Funktionen Lipschitz-stetig sind und bestimmen
        Sie jeweils eine Lipschitz-Konstante.
        \begin{enumerate}
            \item \(f: \left[0, 3\right) \to \RR, f\enbrace*{x} = \sqrt{2 + 3x}\)
            \item \(g: \enbrace*{-3, 2} \to \RR, g\enbrace*{x} = x^2 + 4x - 1\)
        \end{enumerate}
    \end{problem}

    \subsection*{Lösung}
    \begin{enumerate}
        \item Für alle \(x, y \in \left[0, 3\right)\) gilt:
        \[\absolute*{f\enbrace*{x} - f\enbrace*{y}} = \absolute*{\sqrt{2 + 3x} - \sqrt{2 + 3y}} = \absolute*{\frac{3\enbrace*{x - y}}{\sqrt{2 + 3x} + \sqrt{2 + 3y}}} \le \frac{3}{2\sqrt{2}} \cdot \absolute*{x - y}.\]
        Damit ist \(f\) Lipschitz-stetig mit der Lipschitz-Konstante \(L = \frac{3}{2\sqrt{2}}\).
        \item Für alle \(x, y \in \enbrace*{-3, 2}\) gilt:
        \begin{align*}
            \absolute*{g\enbrace*{x} - g\enbrace*{y}} &= \absolute*{x^2 + 4x - 1 - \enbrace*{y^2 + 4y - 1}} = \absolute*{x^2 - y^2 + 4x - 4y}\\
            &= \absolute*{\enbrace*{x + y}\enbrace*{x - y} + 4\enbrace*{x - y}} = \absolute*{\enbrace*{x + y + 4}\enbrace*{x - y}}\\
            &= \absolute*{x + y + 4} \cdot \absolute*{x - y} \le \enbrace*{2 \cdot \max\penbrace*{\absolute*{x}, \absolute*{y}} + 4} \cdot \absolute*{x - y}\\
            &\le \enbrace*{2 \cdot \max\penbrace*{-3, 2} + 4} \cdot \absolute*{x - y} = 8 \cdot \absolute*{x - y}
        \end{align*}
        Damit ist \(g\) Lipschitz-stetig mit der Lipschitz-Konstante \(L = 8\).
    \end{enumerate}

    \section*{Aufgabe 3. (Zwischenwertsatz)}

    \begin{problem}
        Eine wichtige Anwendung des Zwischenwertsatzes ist das Lösen von Gleichungen.
        \begin{enumerate}
            \item Betrachte die Gleichung
            \[\cos\enbrace*{x} - x = \frac{1}{2}.\]
            Zeigen Sie, dass diese Gleichung mindestens eine Lösung aus der Menge der reellen Zahlen besitzt.
            \item Sei \(f: \benbrace*{-1, 1} \to \RR\) stetig mit \(f\enbrace*{1} = f\enbrace*{-1}\). Zeigen Sie, dass es mindestens ein \(x \in \benbrace*{0, 1}\) gibt mit \(f\enbrace*{x} = f\enbrace*{x - 1}\).
        \end{enumerate}
    \end{problem}

    \subsection*{Lösung}
    \begin{enumerate}
        \item Wir wollen zunächst die Fragestellung umformen. Die Frage \emph{``Hat \(\cos\enbrace*{x} - x = \frac{1}{2}\) mindestens eine Lösung in \(\RR\)?''} lässt sich umformen zu \emph{``Hat die Funktion \(f: \RR \to \RR, x \mapsto f\enbrace*{x} = x - \frac{1}{2} - \cos\enbrace*{x}\) mindestens eine Nullstelle?''}, was wir mithilfe des Zwischenwertsatzes prüfen können. \(f\) ist stetig auf ganz \(\RR\), da \(f\) eine aus stetigen Funktionen zusammengesetzte Funktion ist. Wir betrachten nun die Grenzwerte für \(x \to -\infty\) und \(x \to \infty\):
        \[\lim_{x \to -\infty}f\enbrace*{x} = \lim_{x \to -\infty}\underbrace{x}_{\to -\infty} - \underbrace{\frac{1}{2}}_{\to \frac{1}{2}} - \underbrace{\cos\enbrace*{x}}_{\in \benbrace*{1, -1}} = -\infty\]
        \[\lim_{x \to \infty}f\enbrace*{x} = \lim_{x \to \infty}\underbrace{x}_{\to \infty} - \underbrace{\frac{1}{2}}_{\to \frac{1}{2}} - \underbrace{\cos\enbrace*{x}}_{\in \benbrace*{1, -1}} = \infty\]
        Da \(f\) stetig und \(\lim_{x \to -\infty}f\enbrace*{x} = -\infty < \infty = \lim_{x \to \infty}f\enbrace*{x}\) gilt, muss es nach dem Zwischenwertsatz ein \(x_0 \in \RR\) geben, für welches \(f\enbrace*{x_0} = 0\) gilt. Dieses \(x_0\) löst also auch die Gleichung \(\cos\enbrace*{x} - x = \frac{1}{2}\).
        \item Wir betrachten nun die Funktion \(g: \benbrace*{0, 1} \to \RR, x \mapsto f\enbrace*{x} - f\enbrace*{x  - 1}\). Da die Funktion \(f\) stetig ist, muss auch \(g\) stetig sein. Es gilt:
        \[g\enbrace*{0} = f\enbrace*{0} - f\enbrace*{-1} \underbrace{=}_{f\enbrace*{-1} = f\enbrace*{1}} f\enbrace*{0} - f\enbrace*{1} \quad \text{und} \quad g\enbrace*{1} = f\enbrace*{1} - f\enbrace*{0}.\]
        Wir führen nun eine Fallunterscheidung durch. Sei zunächst \(f\enbrace*{0} < f\enbrace*{1}\). Dann gilt \(g\enbrace*{0} < 0 < g\enbrace*{1}\). Daraus folgt mit dem Zwischenwertsatz, dass ein \(x \in \benbrace*{0, 1}\) existiert, für welches \(g\enbrace*{x} = 0\) gilt. Sei nun im zweiten Fall \(f\enbrace*{0} = f\enbrace*{1}\). Dann ist \(g\enbrace*{0} = 0\), also hat \(g\) in diesem Fall eine Nullstelle bei \(x = 0\). Als letztes sei \(f\enbrace*{0} > f\enbrace*{1}\). Dann gilt \(g\enbrace*{1} < 0 < g\enbrace*{0}\) und somit existiert nach dem Zwischenwertsatz auch hier ein \(x \in \benbrace*{0, 1}\) für welches \(g\enbrace*{x} = 0\) gilt.

        In jedem Fall gibt es also ein \(x \in \benbrace*{0, 1}\) mit \(g\enbrace*{x} = 0\). Für dieses \(x\) gilt also nach der Definition von \(g\), dass \(f\enbrace*{x} = f\enbrace*{x - 1}\).
    \end{enumerate}

    \section*{Aufgabe 4. (Stetige bzw. unstetige Ableitungen)}

    \begin{problem}
        Sei \(f: \RR \to \RR\) definiert durch
        \[f\enbrace*{x} = \begin{cases}
            \enbrace*{x - 1}^2 & \text{falls }x > 1,\\
            0 & \text{falls }x \le 1.
        \end{cases}\]
        Begründen oder widerlegen Sie, dass \(f\) stetig differenzierbar in \(\RR\) bzw. zweimal stetig differenzierbar in \(\RR\) ist.
    \end{problem}

    \subsection*{Lösung} Wir formen zunächst die Funktionsgleichung um. Es gilt
    \[f\enbrace*{x} = \begin{cases}
        \enbrace*{x - 1}^2 & \text{falls }x > 1,\\
        0 & \text{falls }x \le 1
    \end{cases} = \begin{cases}
        x^2 - 2x + 1 & \text{falls }x > 1,\\
        0 & \text{falls }x \le 1.
    \end{cases}\]
    Wir bestimmen nun die erste und zweite Ableitung dieser Funktion:
    \[f'\enbrace*{x} = \begin{cases}
        2x - 2 & \text{falls }x > 1,\\
        0 & \text{falls }x \le 1
    \end{cases} \quad \text{und} \quad f''\enbrace*{x} = \begin{cases}
        2 & \text{falls }x > 1,\\
        0 & \text{falls }x \le 1.
    \end{cases}\]
    Zunächst wollen wir prüfen, ob die erste Ableitung stetig ist. Offensichtlich sind \(2x - 2\) und \(0\) stetig. Wir müssen nun also noch die Stelle \(x_0 = 1\) betrachten. Es gilt
    \[\lim_{x \searrow 1}f'\enbrace*{x} = \lim_{x \searrow 1}2x - 2 = 0 \quad \text{und} \quad \lim_{x \nearrow 1}f'\enbrace*{x} = \lim_{x \nearrow 1}0 = 0.\]
    Da die beiden Grenzwerte übereinstimmen ist \(f'\) auch an der Stelle \(x_0 = 1\) stetig und somit auf dem gesamten Definitionsbereichs. Wir prüfen nun genau so die zweite Abbildung. \(2\) und \(0\) sind konstant, also auch stetig. Wir müssen nun erneut die Stelle \(x_0 = 1\) betrachten. Für die Grenzwerte gilt
    \[\lim_{x \searrow 1}f'\enbrace*{x} = \lim_{x \searrow 1}2 = 2 \quad \text{und} \quad \lim_{x \nearrow 1}f'\enbrace*{x} = \lim_{x \nearrow 1}0 = 0.\]
    \(f''\) ist also an der Stelle \(x_0\) nicht stetig. Insgesamt ist die Funktion \(f\) also nur einmal stetig differenzierbar auf \(\RR\).

    \section*{Aufgabe 5. (Matrixdarstellung linearer Abbildungen)}

    \begin{problem}
        Es sei
        \[F: \RR^3 \to \RR^3, F\begin{pmatrix}x\\y\\z\end{pmatrix} = \begin{pmatrix}
            2y + z\\
            x - 4y\\
            3x
        \end{pmatrix},\]
        eine Abbildung. Ermitteln Sie die Matrixdarstellungen \(_{E_3}F_S\), \(_SF_{E_3}\) und \(_SF_S\) von \(F\), wobei
        \[E_3 = \penbrace*{e_1 = \begin{pmatrix}1\\0\\0\end{pmatrix}, e_2 = \begin{pmatrix}0\\1\\0\end{pmatrix}, e_3 = \begin{pmatrix}0\\0\\1\end{pmatrix}}\]
        die Standardbasis von \(\RR^3\) und
        \[S = \penbrace*{u_1 = \begin{pmatrix}1\\1\\1\end{pmatrix}, u_2 = \begin{pmatrix}1\\1\\0\end{pmatrix}, u_3 = \begin{pmatrix}1\\0\\0\end{pmatrix}}\]
        eine weitere Basis von \(\RR^3\) bezeichnen.

        \textbf{Hinweis:} \emph{Sei \(\phi: V \to W\) eine lineare Abbildung, \(B_1 = \penbrace*{v_1, \ldots, v_n} \subset V\) eine Basis von \(V\) und \(B_2 = \penbrace*{w_1, \ldots, w_m} \subset W\) eine Basis von \(W\). Dann heisst die Matrix
        \[_{B_2}F_{B_1} := \begin{pmatrix}
            \alpha_{11} & \ldots & \alpha_{1n}\\
            \vdots & \ddots & \vdots\\
            \alpha_{m1} & \ldots & \alpha_{mn}
        \end{pmatrix}, \quad \text{sodass }\phi\enbrace*{v_j} = \sum_{i = 1}^m\alpha_{ij}w_i \quad \text{für }j = 1, \ldots, n\]
        Matrixdarstellung von \(\phi\) bezüglich \(B_1\) und \(B_2\).}
    \end{problem}

    \subsection*{Lösung}
    \[F\enbrace*{e_1} = \begin{pmatrix}
        2 \cdot 0 + 0\\
        1 - 4 \cdot 0\\
        3 \cdot 1
    \end{pmatrix} = \begin{pmatrix}0\\1\\3\end{pmatrix} = 3u_1 - 2u_2 - u_3\]
    \[F\enbrace*{e_2} = \begin{pmatrix}
        2 \cdot 1 + 0\\
        0 - 4 \cdot 1\\
        3 \cdot 0
    \end{pmatrix} = \begin{pmatrix}2\\4\\0\end{pmatrix} = 4u_2 - 2u_3\]
    \[F\enbrace*{e_3} = \begin{pmatrix}
        2 \cdot 0 + 1\\
        0 - 4 \cdot 0\\
        3 \cdot 0
    \end{pmatrix} = \begin{pmatrix}1\\0\\0\end{pmatrix} = u_3\]
    Für die erste gesuchte Abbildungsmatrix ergibt sich nun
    \(_{E_3}F_S = \begin{pmatrix}
        3 & 0 & 0\\
        -2 & 4 & 0\\
        -1 & -2 & 1
    \end{pmatrix}\).
    \[F\enbrace*{u_1} = \begin{pmatrix}
        2 \cdot 1 + 1\\
        1 - 4 \cdot 1\\
        3 \cdot 1
    \end{pmatrix} = \begin{pmatrix}3\\-3\\3\end{pmatrix} = 3e_1 - 3e_2 + 3e_3 = 3u_1 - 6u_2 + 6u_3\]
    \[F\enbrace*{u_2} = \begin{pmatrix}
        2 \cdot 1 + 0\\
        1 - 4 \cdot 1\\
        3 \cdot 1
    \end{pmatrix} = \begin{pmatrix}2\\-3\\3\end{pmatrix} = 2e_1 - 3e_2 + 3e_3 = 3u_1 - 6u_2 + 5u_3\]
    \[F\enbrace*{u_3} = \begin{pmatrix}
        2 \cdot 0 + 0\\
        1 - 4 \cdot 0\\
        3 \cdot 1
    \end{pmatrix} = \begin{pmatrix}0\\1\\3\end{pmatrix} = e_2 + 3e_3 = 3u_1 - 2u_2 - u_3\]
    Für die anderen beiden Abbildungsmatrizen ergibt sich
    \[_SF_{E_3} = \begin{pmatrix}
        3 & 2 & 0\\
        -3 & -3 & 1\\
        3 & 3 & 3
    \end{pmatrix} \quad \text{und} \quad _SF_S = \begin{pmatrix}
        3 & 3 & 3\\
        -6 & -6 & -2\\
        6 & 5 & -1
    \end{pmatrix}.\]

    \section*{Aufgabe 6. (Matrixmultiplikation)}

    \begin{problem}
        \begin{enumerate}
            \item Gegeben seien die Matrizen \(A \in \RR^{2 \times 3}\), \(B \in \RR^{3 \times 3}\) und \(C \in \RR^{3 \times 2}\) mit
            \[A = \begin{pmatrix}
                2 & 1 & -1\\
                0 & 2 & 1
            \end{pmatrix}, \quad B = \begin{pmatrix}
                4 & -2 & 0\\
                -3 & -1 & -1\\
                5 & 0 & 2
            \end{pmatrix}, \quad C = \begin{pmatrix}
                -1 & 2\\
                4 & 0\\
                3 & 1
            \end{pmatrix}\]
            Prüfen Sie, ob die Multiplikation dieser drei Matrizen assoziativ ist, d.h. ob
            \[\enbrace*{AB}C = A\enbrace*{BC}\]
            gilt.
            \item Wie aus der Vorlesung bekannt, ist die Matrixmultiplikation \emph{im Allgemeinen} nicht kommutativ. Gegeben sei die Matrix
            \[X = \begin{pmatrix}
                a & b\\
                c & d
            \end{pmatrix} \in \RR^{2 \times 2}.\]
            Welche Bedingungen müssen die Parameter \(a, b, c, d \in \RR\) erfüllen, damit sich das Matrizenprodukt mit der Matrix
            \[D = \begin{pmatrix}
                1 & 0\\
                -1 & 1
            \end{pmatrix} \in \RR^{2 \times 2}\]
            kommutativ verhält (d.h. \(DX = XD\))?
        \end{enumerate}
    \end{problem}

    \subsection*{Lösung}
    \begin{enumerate}
        \item Damit die Multiplikation assoziativ ist, muss \(\enbrace*{AB}C = A\enbrace*{BC}\) gelten. Wir berechnen beide Seiten einzeln:
        \begin{align*}
            \enbrace*{AB}C &= \enbrace*{\begin{pmatrix}
                2 & 1 & -1\\
                0 & 2 & 1
            \end{pmatrix} \cdot \begin{pmatrix}
                4 & -2 & 0\\
                -3 & -1 & -1\\
                5 & 0 & 2
            \end{pmatrix}} \cdot C\\
            &= \scalebox{.65}{\(\begin{pmatrix}
                2 \cdot 4 + 1 \cdot \enbrace*{-3} + \enbrace*{-1} \cdot 5 & 2 \cdot \enbrace*{-2} + 1 \cdot \enbrace*{-1} + \enbrace*{-1} \cdot 0 & 2 \cdot 0 + 1 \cdot \enbrace*{-1} + \enbrace*{-1} \cdot 2\\
                0 \cdot 4 + 2 \cdot \enbrace*{-3} + 1 \cdot 5 & 0 \cdot \enbrace*{-2} + 2 \cdot \enbrace*{-1} + 1 \cdot 0 & 0 \cdot 0 + 2 \cdot \enbrace*{-1} + 1 \cdot 2
            \end{pmatrix}\)} \cdot C\\
            &= \begin{pmatrix}
                0 & -5 & -3\\
                -1 & -2 & 0
            \end{pmatrix} \cdot \begin{pmatrix}
                -1 & 2\\
                4 & 0\\
                3 & 1
            \end{pmatrix} = \scalebox{.65}{\(\begin{pmatrix}
                0 \cdot \enbrace*{-1} + \enbrace*{-5} \cdot 4 + \enbrace*{-3} \cdot 3 & 0 \cdot 2 + \enbrace*{-5} \cdot 0 + \enbrace*{-3} \cdot 1\\
                \enbrace*{-1} \cdot \enbrace*{-1} + \enbrace*{-2} \cdot 4 + 0 \cdot 3 & \enbrace*{-1} \cdot 2 + \enbrace*{-2} \cdot 0 + 0 \cdot 1
            \end{pmatrix}\)}\\
            &= \begin{pmatrix}
                -29 & -3\\
                -7 & -2
            \end{pmatrix},
        \end{align*}
        \begin{align*}
            A\enbrace*{BC} &= A \cdot \enbrace*{\begin{pmatrix}
                4 & -2 & 0\\
                -3 & -1 & -1\\
                5 & 0 & 2\\
            \end{pmatrix} \cdot \begin{pmatrix}
                -1 & 2\\
                4 & 0\\
                3 & 1
            \end{pmatrix}}\\
            &= A \cdot \scalebox{.65}{\(\begin{pmatrix}
                4 \cdot \enbrace*{-1} + \enbrace*{-2} \cdot 4 + 0 \cdot 3 & 4 \cdot 2 + \enbrace*{-2} \cdot 0 + 0 \cdot 1\\
                \enbrace*{-3} \cdot \enbrace*{-1} + \enbrace*{-1} \cdot 4 + \enbrace*{-1} \cdot 3 & \enbrace*{-3} \cdot 2 + \enbrace*{-1} \cdot 0 + \enbrace*{-1} \cdot 1\\
                5 \cdot \enbrace*{-1} + 0 \cdot 4 + 2 \cdot 3 & 5 \cdot 2 + 0 \cdot 0 + 2 \cdot 1
            \end{pmatrix}\)}\\
            &= \begin{pmatrix}
                2 & 1 & -1\\
                0 & 2 & 1
            \end{pmatrix} \cdot \begin{pmatrix}
                -12 & 8\\
                -4 & -7\\
                1 & 12
            \end{pmatrix} = \scalebox{.65}{\(\begin{pmatrix}
                2 \cdot \enbrace*{-12} + 1 \cdot \enbrace*{-4} + \enbrace*{-1} \cdot 1 & 2 \cdot 8 + 1 \cdot \enbrace*{-7} + \enbrace*{-1} \cdot 12\\
                0 \cdot \enbrace*{-12} + 2 \cdot \enbrace*{-4} + 1 \cdot 1 & 0 \cdot 8 + 2 \cdot \enbrace*{-7} + 1 \cdot 12\\
            \end{pmatrix}\)}\\
            &= \begin{pmatrix}
                -29 & -3\\
                -7 & -2
            \end{pmatrix}.
        \end{align*}
        Somit ist gezeigt, dass die Multiplikation assoziativ ist.
        \item Wir berechnen
        \[D \cdot X = \begin{pmatrix}
            1 & 0\\
            -1 & 1
        \end{pmatrix} \cdot \begin{pmatrix}
            a & b\\
            c & d
        \end{pmatrix} = \begin{pmatrix}
            a & b\\
            a - c & b - d
        \end{pmatrix},\]
        \[X \cdot D = \begin{pmatrix}
            a & b\\
            c & d
        \end{pmatrix} \cdot \begin{pmatrix}
            1 & 0\\
            -1 & 1
        \end{pmatrix} = \begin{pmatrix}
            a - b & b\\
            c - d & d
        \end{pmatrix}.\]
        Wir wissen, dass zwei Matrizen gleich sind, wenn alle ihre Einträge gleich sind. Damit erhalten wir durch gleichsetzen unserer beiden Matrizen die folgenden vier Gleichungen:
        \[a = a - b, \quad b = b, \quad a - c = c - d, \quad b - d = d.\]
        Aus der ersten Gleichung folgt sofort \(b = 0\), was auch die zweite Gleichung erfüllt. Setzen wir dies in die vierte Gleichung ein, so erhalten wir \(d = 0\). Aus der dritten Gleichung erhalten wir dann \(2c = a\). Wir können also \(c \in \RR\) frei wählen und erhalten für unsere Matrix
        \[X = \begin{pmatrix}
            2c & 0\\
            c & 0
        \end{pmatrix}, \quad c \in \RR.\]
    \end{enumerate}
\end{document}
