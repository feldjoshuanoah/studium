\documentclass[german,12pt]{homework}

\usepackage[ngerman]{babel}
\usepackage[utf8]{inputenc}
\usepackage[T1]{fontenc}

\newcommand{\NN}{\mathbb{N}}
\newcommand{\RR}{\mathbb{R}}
\newcommand{\KK}{\mathbb{K}}

\newcommand{\dotproduct}[2]{\left\langle#1, #2\right\rangle}
\newcommand{\dd}{\,\differ}

\DeclareMathOperator{\differ}{d}
\DeclareMathOperator{\vecspan}{span}

\DeclarePairedDelimiter{\absolute}{\lvert}{\rvert}
\DeclarePairedDelimiter{\norm}{\lVert}{\rVert}
\DeclarePairedDelimiter{\enbrace}{(}{)}
\DeclarePairedDelimiter{\benbrace}{[}{]}
\DeclarePairedDelimiter{\penbrace}{\{}{\}}

\title{Hausaufgabenübung 6}
\author{Joshua Feld, 406718 \quad Jeff Vogel, 407758 \quad Henrik Herrmann, 421853}
\date{14. Dezember, 2020}
\institute{RWTH Aachen University\\Center for Computational Engineering Science}
\class{Mathematische Grundlagen I}
\professor{Prof. Dr. Torrilhon \& Prof. Dr. Stamm}

\begin{document}
    \maketitle

    \section*{Aufgabe 1. (Konvergenz von Folgen)}

    \begin{problem}
        Sei \(a_n = \frac{8n^2 - 5}{4n^2 + 7}\). Zeigen Sie, dass die Folge
        monoton und beschränkt ist. Hieraus folgt die Konvergenz. Bestimmen Sie
        den Grenzwert und beweisen Sie, dass die Folge \(\enbrace*{a_n}_{n \in
        \NN}\) diesen Grenzwert hat mittels direkter Anwendung der Definition
        des Grenzwertes.
    \end{problem}

    \subsection*{Lösung} Wir wollen zunächst zeigen, dass \(a_{n + 1} \ge a_n\)
    für alle \(n \in \NN\) gilt, denn dann ist die Folge monoton wachsend. Es
    gilt
    \begin{align*}
        &\frac{8\enbrace*{n + 1}^2 + 5}{4\enbrace*{n + 1}^2 + 7} \ge \frac{8n^2
        + 5}{4n^2 + 7}\\ \iff &\enbrace*{8\enbrace*{n + 1}^2 - 5}\enbrace*{4n^2
        + 7} \ge \enbrace*{8n^2 + 5}\enbrace*{4\enbrace*{n  1}^2 + 7}\\
        \iff &32\enbrace*{n + 1}^2n^2 - 20n^2 + 56\enbrace*{n + 1}^2 - 35 \ge
        32\enbrace*{n + 1}^2n^2 - 20\enbrace*{n + 1}^2 + 56n^2 - 35\\
        \iff &56\enbrace*{n + 1}^2 - 20n^2 \ge 56n^2 - 20\enbrace*{n + 1}^2\\
        \iff &56\enbrace*{2n + 1} \ge -20\enbrace*{2n + 1}
    \end{align*}
    Dies ist offensichtlich wahr für alle \(n \in \NN\), denn \(56 > -20\). Nun
    wollen wir noch die Beschränktheit zeigen. Da die Folge monoton wächst,
    müssen wir die Beschränktheit nach oben zeigen. Sei \(a_n \le x\) für ein
    \(x \in \NN\) und für alle \(n \in \NN\). Dann gilt
    \[\frac{8n^2 - 5}{4n^2 + 7} \le x \iff 8n^2 - 5 \le x \cdot \enbrace*{4n^2
    + 7} \iff \enbrace*{8 - 4x} \cdot n^2 \le 7x + 5.\]
    Diese Ungleichung ist beispielsweise für \(x = 2\) erfüllt, denn dann
    lautet die Ungleichung \(0 \le 5\) was für alle \(n \in \NN\) gilt.
    Folglich ist \(x\) eine obere Schranke für \(\enbrace*{a_n}_{n \in \NN}\)
    und die Folge ist beschränkt. Für den Grenzwert gilt
    \[\lim_{n \to \infty}\frac{8n^2 - 5}{4n^2 + 7} = \lim_{n \to \infty}\frac{8
    - \frac{5}{n^2}}{4 + \frac{7}{n^2}} = \frac{8 - \lim_{n \to
    \infty}\frac{5}{n^2}}{4 + \lim_{n \to \infty}\frac{7}{n^2}} = \frac{8}{4} =
    2.\]
    Wir wollen nun mithilfe der Definition des Grenzwertes zeigen, dass \(2\)
    tatsächlich der Grenzwert der Folge ist. Sei \(\varepsilon > 0\) beliebig.
    Dann gilt
    \begin{align*}
        \absolute*{\frac{8n^2 - 5}{4n^2 + 7} - 2} = \absolute*{\frac{8n^2 - 5 -
        \enbrace*{8n^2 + 14}}{4n^2 + 7}} = \absolute*{-\frac{19}{4n^2 + 7}} \le
        \frac{19}{4n^2} \le \frac{20}{4n^2} \le \frac{5}{n} < \varepsilon
    \end{align*}

    \section*{Aufgabe 2. (Konvergenz von Reihen)}

    \begin{problem}
        Untersuchen Sie die folgenden Reihen auf Konvergenz und bestimmen Sie
        gegebenenfalls den Grenzwert:
        \begin{enumerate}
            \item \(\sum_{k = 1}^\infty\frac{\enbrace*{2i}^k + 3^{k - 1}}{5^k}\)
            \item \(\sum_{n = 1}^\infty\frac{1}{4n^2 - 1}\)
        \end{enumerate}

        \textbf{Hinweis:} \emph{Bestimmen Sie zunächst \(a, b \in \RR\), so
        dass \(\frac{1}{4n^2 - 1} = \frac{a}{2n - 1} + \frac{b}{2n + 1}\) für
        alle \(n \in \NN\) gilt.}
    \end{problem}

    \subsection*{Lösung}

    \section*{Aufgabe 3. (Orthonormalbasen)}

    \begin{problem}
        Gegeben seien die Vektoren
        \[v_1 = \frac{1}{3} \cdot \begin{pmatrix}2\\2\\1\end{pmatrix}, \quad
        v_2 = \frac{1}{9} \cdot \begin{pmatrix}6\\-3\\-6\end{pmatrix}, \quad
        v_3 = \frac{2}{9} \cdot \begin{pmatrix}-\frac{3}{2}\\3\\-3
        \end{pmatrix}.\]
        Zeigen Sie, dass es sich um eine Orthonormalbasis bzgl. des
        euklidischen Skalarprodukts und der dazugehörigen Norm handelt.
    \end{problem}

    \subsection*{Lösung}

    \section*{Aufgabe 4. (Orthogonales Komplement)}

    \begin{problem}
        Sei \(V = \RR^4\) mit dem (Standard-)Skalarprodukt \(\dotproduct{x}{y}
        := \sum_{i = 1}^4x_iy_i\) für \(x, y \in \RR^4\) gegeben. Berechnen Sie
        das orthogonale Komplement von
        \[U := \vecspan\penbrace*{\begin{pmatrix}1\\2\\0\\2\end{pmatrix},
        \begin{pmatrix}2\\-2\\0\\1\end{pmatrix}}.\]

        \textbf{Hinweis:} \emph{Ergänzen Sie
        \(\begin{pmatrix}1\\2\\0\\2\end{pmatrix}, \begin{pmatrix}2\\-
        2\\0\\1\end{pmatrix}\) zu einer Basis des \(\RR^4\) und verwenden Sie
        dann das Gram-Schmidtsche Orthonormalisierungsverfahren.}
    \end{problem}

    \subsection*{Lösung}

    \section*{Aufgabe 5. (Bestapproximation)}

    \begin{problem}
        Seien
        \[v_1 = \begin{pmatrix}1\\0\\1\end{pmatrix}, \quad v_2 =
        \begin{pmatrix}0\\1\\1\end{pmatrix}, \quad v_3 =
        \begin{pmatrix}1\\0\\3\end{pmatrix}\]
        Vektoren im \(\RR^3\) und \(U = \vecspan\penbrace*{v_1, v_2}\) ein
        Unterraum von \(\RR^3\). Bestimmen Sie die Bestapproximation von
        \(v_3\) durch ein Element \(u^* \in U\).
    \end{problem}

    \subsection*{Lösung}

    \section*{Aufgabe 6. (Gram-Schmidtsches Orthonormalisierungsverfahren)}

    \begin{problem}
        Sei \(V = C^0\enbrace*{\benbrace*{-1, 1}, \RR}\) sowie
        \(v_i\enbrace*{x} = x^i, i = 0, 1, 2, 3\). Verwenden Sie das Gram-
        Schmidtsche Orthonormalisierungsverfahren, um eine Orthonormalbasis des
        \(\mathcal{P}_3\) unter dem Skalarprodukt
        \[\dotproduct{f}{g} := \int_{-1}^1f\enbrace*{x}g\enbrace*{x}\dd{x}\]
        zu bestimmen.

        \textbf{Hinweis:} \emph{Diese orthogonalen Polynome werden Legendre-
        Polynome genannt. Eine wichtige Roll spielen die Legendre-Polynome in
        der theoretischen Physik, insbesondere in der Elektrodynamik und in der
        Quantenmechanik.}
    \end{problem}

    \subsection*{Lösung}
\end{document}
