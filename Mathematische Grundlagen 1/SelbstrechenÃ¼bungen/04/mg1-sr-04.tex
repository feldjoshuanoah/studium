\documentclass[german,12pt]{homework}

\usepackage[ngerman]{babel}
\usepackage[utf8]{inputenc}
\usepackage[T1]{fontenc}

\usepackage{tikz}

\newcommand{\NN}{\mathbb{N}}
\newcommand{\ZZ}{\mathbb{Z}}
\newcommand{\QQ}{\mathbb{Q}}
\newcommand{\RR}{\mathbb{R}}
\newcommand{\CC}{\mathbb{C}}

\newcommand{\dotproduct}[2]{\left\langle#1, #2\right\rangle}
\newcommand{\dd}{\,\differ}

\DeclareMathOperator{\differ}{d}
\DeclareMathOperator{\vecspan}{span}

\DeclarePairedDelimiter{\absolute}{\lvert}{\rvert}
\DeclarePairedDelimiter{\norm}{\lVert}{\rVert}
\DeclarePairedDelimiter{\enbrace}{(}{)}
\DeclarePairedDelimiter{\benbrace}{[}{]}
\DeclarePairedDelimiter{\penbrace}{\{}{\}}

\title{Selbstrechenübung 4}
\author{Joshua Feld, 406718}
\institute{RWTH Aachen University\\Center for Computational Engineering Science}
\class{Mathematische Grundlagen I}
\professor{Prof. Dr. Torrilhon \& Prof. Dr. Stamm}

\begin{document}
    \maketitle

    \section*{Aufgabe 1. (Komplexe Zahlen)}

    \begin{problem}
        Schreiben Sie die folgende komplexe Zahl in der Form \(z = x + iy\) mit \(x, y \in \mathbb{R}\) und in Polarkoordinaten, d.h. in der Form \(z = r\enbrace*{\cos\enbrace*{\varphi} + i\sin\enbrace*{\varphi}}\) Bestimmen Sie dazu den Betrag \(r\) und das Argument \(\varphi\) von \(z\) exakt in \(\left(-\pi, \pi\right]\).
        \[z = \enbrace*{\frac{1 - i}{1 + i}}^5\]
    \end{problem}

    \subsection*{Lösung} Wir wollen zunächst den Bruch ohne den Exponenten vereinfachen. Dazu multiplizieren wir diesen mit dem komplexen Konjugat des Nenners:
    \[\frac{1 - i}{1 + i} \cdot \frac{1 - i}{1 - i} = \frac{\enbrace*{1 - i}\enbrace*{1 - i}}{\enbrace*{1 + i}\enbrace*{1 - i}} = \frac{-i}{1} = -i.\]
    Folglich können wir \(z = \enbrace*{-i}^5\) schreiben. Wir wenden nun die Exponentenregel an, die besagt, dass \(\enbrace*{-a}^n = -a^n\) gilt, falls \(n\) ungerade ist. Somit gilt insgesamt
    \[z = -i^5 = -i.\]
    Um \(\varphi\) zu bestimmen benötigen wir zunächst noch den Betrag von \(z\). Dieser ist
    \[r = \absolute*{z} = \sqrt{0^2 + \enbrace*{-1}^2} = 1.\]
    Nun können wir \(\varphi\) wie folgt bestimmen:
    \[x = r \cdot \cos\enbrace*{\varphi} \iff 0 = \cos\enbrace*{\varphi} \iff \varphi = \arccos\enbrace*{0} = \frac{\pi}{2}.\]
    Also ist
    \[z = r\enbrace*{\cos\enbrace*{\varphi} + i\sin\enbrace*{\varphi}} = \cos\enbrace*{\frac{\pi}{2}} + i\sin\enbrace*{\frac{\pi}{2}}.\]

    \section*{Aufgabe 2. (Komplexe Zahlen)}

    \begin{problem}
        \begin{enumerate}
            \item Bestimmen Sie Realteil und Imaginärteil der komplexen Zahlen
            \[w_1 = \frac{2}{1 - 3i}, \quad w_2 = \frac{1}{i}, \quad w_3 = \frac{1 + it}{1 - it} \text{ mit }t \in \mathbb{R}.\]
            \item Berechnen Sie den Betrag von \(z = \frac{\enbrace*{3 + 4i}\enbrace*{-1 + 2i}}{\enbrace*{-1 - i}\enbrace*{3 - i}}\).
            \item Für welche \(z \in \mathbb{C}\) ist \(\absolute*{\frac{z + i}{z - i}} \le 1\)? Skizzieren Sie die Lösungsmenge in der komplexen Ebene.
            \item Für welche \(z \in \mathbb{C}\) ist \(\Re\enbrace*{\frac{1 + z}{1 - z}} \ge 0\)? Skizzieren Sie die Lösungsmenge in der komplexen Ebene.
        \end{enumerate}
    \end{problem}

    \subsection*{Lösung}
    \begin{enumerate}
        \item Wir multiplizieren zunächst den Zähler und den Nenner von \(w_1\) mit dem komplexen Konjugat des Nenners, um diesen zu eliminieren:
        \[\frac{2}{1 - 3i} \cdot \frac{1 + 3i}{1 + 3i} = \frac{2\enbrace*{1 + 3i}}{\enbrace*{1 - 3i}\enbrace*{1 + 3i}} = \frac{2\enbrace*{1 + 3i}}{1 - 3i + 3i + \enbrace*{3i}^2} = \frac{2\enbrace*{1 + 3i}}{10}.\]
        Wir können nun den gemeinsamen Faktor \(2\) rauskürzen und erhalten
        \[\frac{1 + 3i}{5} = \frac{1}{5} + \frac{3}{5}i,\]
        d.h. es gilt \(\Re\enbrace*{w_1} = \frac{1}{5}\) und \(\Im\enbrace*{w_1} = \frac{3}{5}\).

        Wir multiplizieren wieder mit dem komplexen Konjugat und erhalten
        \[\frac{1}{i} \cdot \frac{-i}{-i} = \frac{1 \cdot \enbrace*{-i}}{i\enbrace*{-i}} = \frac{-i}{1} = -i.\]
        Somit ist \(\Re\enbrace*{w_2} = 0\) und \(\Im\enbrace*{w_2} = -1\).

        Wir multiplizieren erneut mit dem komplexen Konjugat und erhalten
        \[\frac{1 + it}{1 - it} \cdot \frac{1 + it}{1 + it} = \frac{\enbrace*{1 + it}\enbrace*{1 + it}}{\enbrace*{1 - it}\enbrace*{1 + it}} = \frac{1 - t^2 + 2it}{1 + t^2} = \frac{1 - t^2}{t^2 + 1} + \frac{2t}{t^2 + 1}i,\]
        also \(\Re\enbrace*{w_3} = \frac{1 - t^2}{t^2 + 1}\) und \(\Im\enbrace*{w_3} = \frac{2t}{t^2 + 1}\).
        \item Wir definieren zunächst vier neue komplexe Zahlen \(z_1, \ldots, z_4 \in \mathbb{C}\) mit
        \[z_1 = 3 + 4i, \quad z_2 = -1 + 2i, \quad z_3 = -1 - i, \quad z_4 = 3 - i.\]
        Nun können wir den Betrag von \(z\) berechnen mit
        \[\absolute*{z} = \frac{\absolute*{z_1} \cdot \absolute*{z_2}}{\absolute*{z_3} \cdot \absolute*{z_4}} = \sqrt{\frac{\enbrace*{3^2 + 4^2}\enbrace*{\enbrace*{-1}^2 + 2^2}}{\enbrace*{\enbrace*{-1}^2 + \enbrace*{-1}^2}\enbrace*{3^2 + \enbrace*{-1}^2}}} = \sqrt{\frac{25 \cdot 5}{2 \cdot 10}} = \sqrt{\frac{125}{20}} = \frac{5}{2}.\]
        \item Seien \(x := \Re\enbrace*{z}, y := \Im\enbrace*{z}\) für \(z \in \CC\). Wir berechnen zunächst \(\absolute*{\frac{z + i}{z - i}}\) für \(z \in \CC \setminus \penbrace*{i}\):
        \[\absolute*{\frac{z + i}{z - i}} = \absolute*{\frac{x + iy + i}{x + iy - i}} = \absolute*{\frac{x + \enbrace*{y + 1}i}{x + \enbrace*{y - 1}i}} = \sqrt{\frac{x^2 + \enbrace*{y + 1}^2}{x^2 + \enbrace*{y - 1}^2}}.\]
        Für \(c \in \CC \setminus \penbrace*{i}\) gilt somit
        \begin{align*}
            \absolute*{\frac{z + i}{z - i}} \le 1 &\iff \sqrt{\frac{x^2 + \enbrace*{y + 1}^2}{x^2 + \enbrace*{y - 1}^2}} \le 1 \iff \frac{x^2 + \enbrace*{y + 1}^2}{x^2 + \enbrace*{y - 1}^2} \le 1\\
            &\iff x^2 + \enbrace*{y + 1}^2 \le x^2 + \enbrace*{y - 1}^2 \iff \enbrace*{y + 1}^2 \le \enbrace*{y - 1}^2\\
            &\iff y^2 + 2y + 1 \le y^2 - 2y + 1 \iff 4y \le 0 \iff y \le 0.
        \end{align*}
        Daraus folgt für die Lösungsmenge
        \[M := \penbrace*{z \in \CC : \absolute*{\frac{z + i}{z - i}} \le 1} = \penbrace*{z \in \CC : \Im\enbrace*{z} \le 0}.\]
        In der komplexen Zahlenebene sieht die Menge wie folgt aus:
        \begin{center}
            \begin{tikzpicture}[scale=.75]
                \begin{scope}[thick,font=\scriptsize]
                    \draw [->] (-5,0) -- (5,0) node [above left] {\(\Re\enbrace*{z}\)};
                    \draw [->] (0,-5) -- (0,5) node [below right] {\(\Im\enbrace*{z}\)};

                    \foreach \n in {-4,-3,...,-1,1,2,...,4}{
                        \draw (\n,3pt) -- (\n,-3pt) node [below] {\(\n\)};
                        \draw (3pt,\n) -- (-3pt,\n) node [left] {\(\n\)};
                    }

                    \filldraw[draw=none,fill=gray,fill opacity=0.45] (-5,0) -- (-5,-5) -- (5,-5) -- (5,0) -- cycle;
                \end{scope}
            \end{tikzpicture}
        \end{center}
        \item Seien \(x := \Re\enbrace*{z}, y := \Im\enbrace*{z}\) für \(z \in \CC\). Wir wollen nun \(\Re\enbrace*{\frac{1 + z}{1 - z}}\) für \(z \ne 1\) (\(z \in \CC\)) berechnen. Es gilt
        \begin{align*}
            \frac{1 + z}{1 - z} &= \frac{1 + x + iy}{1 - x - iy} = \frac{\enbrace*{1 + x + iy}\enbrace*{1 - x + iy}}{\enbrace*{1 - x}^2 + y^2}\\
            &= \frac{1 - x^2 + \enbrace*{1 + x + 1 - x}yi - y^2}{\enbrace*{1 - x}^2 + y^2}\\
            &= \frac{1 - x^2 - y^2}{\enbrace*{1 - x}^2 + y^2} + \frac{2y}{\enbrace*{1 - x}^2 + y^2} \cdot i,
        \end{align*}
        also \(\Re\enbrace*{\frac{1 + z}{1 - z}} = \frac{1 - x^2 - y^2}{\enbrace*{1 - x}^2 + y^2}\). Für \(z \in \CC \setminus \penbrace*{1}\) gilt
        \begin{align*}
            \Re\enbrace*{\frac{1 + z}{1 - z}} \ge 0 &\iff 1 - x^2 - y^2 \ge 0 \iff 1 \ge x^2 + y^2\\
            &\iff 1 \ge \absolute*{z}^ 2 \iff 1 \ge \absolute*{z}.
        \end{align*}
        Daraus folgt für die Lösungsmenge
        \[M := \penbrace*{z \in \CC : \Re\enbrace*{\frac{1 + z}{1 - z}} \ge 0} = \penbrace*{z \in \CC : z \ne 1, \absolute*{z} \le 1}.\]
        \begin{center}
            \begin{tikzpicture}[scale=.75]
                \begin{scope}[thick,font=\scriptsize]
                    \draw [->] (-5,0) -- (5,0) node [above left] {\(\Re\enbrace*{z}\)};
                    \draw [->] (0,-5) -- (0,5) node [below right] {\(\Im\enbrace*{z}\)};

                    \foreach \n in {-2,-1,1,2} {
                        \draw (\n * 2,3pt) -- (\n * 2,-3pt) node [below] {\(\n\)};
                        \draw (3pt,\n * 2) -- (-3pt,\n * 2) node [left] {\(\n\)};
                    }

                    \path [draw=black,fill=gray,fill opacity=0.45] (0,0) circle (2);
                \end{scope}
            \end{tikzpicture}
        \end{center}
    \end{enumerate}

    \section*{Aufgabe 3. (Unterräume)}

    \begin{problem}
        Überprüfen Sie, ob die folgenden Mengen Unterräume der jeweils angegebenen \(\mathbb{R}\)-Vektorräume \(V\) sind:
        \begin{enumerate}
            \item \(U = \penbrace*{a, b \in \mathbb{Z}: \enbrace*{a + b, 3a - b, 2a + b}}\), \quad \(V = \mathbb{R}^3\)
            \item \(U = \penbrace*{a, b \in \mathbb{R}: \enbrace*{a + b, 3a - b, 2a + b}}\), \quad \(V = \mathbb{R}^3\)
            \item \(U = \penbrace*{a, b \in \mathbb{R}: \enbrace*{a + b, 3a - b, 2a + b + 1}}\), \quad \(V = \mathbb{R}^3\)
            \item \(U = \penbrace*{a, b \in \mathbb{R}: \enbrace*{a + b, 3a - b, 2a + b, a}}\), \quad \(V = \mathbb{R}^3\)
        \end{enumerate}
    \end{problem}

    \subsection*{Lösung}
    \begin{enumerate}
        \item Sei \(a = 1\) und \(b = 0\). Dann ist \(\enbrace*{1, 2, 3} \in U\). Sei nun \(\lambda = \frac{1}{2}\). Dann gilt
        \[\lambda \cdot \enbrace*{1, 2, 3} = \enbrace*{\frac{1}{2}, 1, \frac{3}{2}} \not\in U,\]
        denn es es existieren keine zwei ganzen Zahlen \(a, b \in \ZZ\), so dass \(a + b = \frac{1}{2}\). Folglich ist \(U\) kein Unterraum von \(V\).
        \item Es gilt \(U \subset V\) und \(0 \in U\) offensichtlich mit \(a = b = 0\). Seien nun
        \[v = \enbrace*{a_1 + b_1, 3a_1 - b_1, 2a_1 + b_1}, \quad w = \enbrace*{a_2 + b_2, 3a_2 - b_2, 2a_2 + b_2} \in U\]
        und \(\lambda \in \RR\). Dann gilt
        \begin{align*}
            \lambda \cdot \enbrace*{v + w} &= \lambda \cdot \enbrace*{a_1 + b_1 + a_2 + b_2, 3a_1 - b_1 + 3a_2 - b_2, 2a_1 + b_1 + 2a_2 + b_2}\\
            &= \lambda \cdot \enbrace*{a' + b', 3a' - b', 2a' + b'} \quad \text{mit }a' = a_1 + a_2\text{ und }b' = b_1 + b_2\\
            &= \enbrace*{a + b, 3a - b, 2a + b} \in U \quad \text{mit }a = \lambda{a'}\text{ und }b = \lambda{b'}
        \end{align*}
        Da \(U\) abgeschlossen ist im Bezug auf Addition und Multiplikation mit Skalaren, ist \(U\) ein Unterraum von \(V\).
        \item Falls \(U\) ein Unterraum von \(V\) ist, muss \(U\) den Nullvektor enthalten. In anderen Worten, wir müssten \(a, b \in \RR\) finden, so dass
        \[a + b = 0, \quad 3a - b = 0, \quad 2a + b + 1 = 0.\]
        Allerdings folgt aus den ersten zwei Gleichungen \(a = b = 0\), was die dritte Gleichung nicht erfüllt. Daraus schließen wir, dass \(U\) kein Unterraum von \(V\) ist.
        \item \(U\) ist offensichtlich kein Unterraum von \(V\), denn \(U \subset \mathbb{R}^4\) aber \(V = \mathbb{R}^3\).
    \end{enumerate}

    \section*{Aufgabe 4. (Parallelogrammidentität)}

    \begin{problem}
        \begin{enumerate}
            \item Gegeben sei ein Vektorraum \(V\) in dem ein Skalarprodukt definiert ist mit \(\norm*{x} = \sqrt{\dotproduct{x}{x}}\), \(x \in V\). Beweisen Sie die Parallelogrammidentität
            \[\norm*{x + y}^2 + \norm*{x - y}^2 = 2\enbrace*{\norm*{x}^2 + \norm*{y}^2}.\]
            \item Gilt in einem normierten Vektorraum die Parallelogrammidentität, so existiert ein Skalarprodukt welches die Norm erzeugt:
            \[\norm*{x} = \sqrt{\dotproduct{x}{x}}.\]
            Beweisen Sie diese Aussage.
        \end{enumerate}
    \end{problem}

    \subsection*{Lösung}

    \begin{enumerate}
        \item Wir können die Parallelogrammidentität mit simplen Umformungen direkt zeigen:
        \begin{align*}
            \norm*{x + y}^2 + \norm*{x - y}^2 &= \dotproduct{x + y}{x + y} + \dotproduct{x - y}{x - y}\\
            &= \dotproduct{x}{x + y} + \dotproduct{y}{x + y} + \dotproduct{x}{x - y} - \dotproduct{y}{x - y}\\
            &= \dotproduct{x}{x} + \dotproduct{x}{y} + \dotproduct{y}{x} + \dotproduct{y}{y} + \dotproduct{x}{x} - \dotproduct{x}{y}\\
            &\quad\, - \dotproduct{y}{x} + \dotproduct{y}{y}\\
            &= 2\dotproduct{x}{x} + 2\dotproduct{y}{y} = 2\enbrace*{\norm*{x}^2 + \norm*{y}^2}.
        \end{align*}
        \item Mit Hilfe der Polarisationsformel wählen wir
        \[\dotproduct{x}{y} = \frac{1}{4}\enbrace*{\norm*{x + y}^2 - \norm*{x - y}^2}.\]
        Es bleibt zu zeigen, dass dies ein Skalarprodukt definiert. Die Symmetrie, Definitheit und Nichtnegativität ist klar, da das Skalarprodukt durch eine Norm definiert ist. Es bleibt die Bilinearität
        \[\dotproduct{\alpha{x}}{y} = \alpha\dotproduct{x}{y},\]
        \[\dotproduct{x_1 + x_2}{y} = \dotproduct{x_1}{y} + \dotproduct{x_2}{y}\]
        zu zeigen. Es gilt unter Verwendung der Parallelogrammgleichung:
        \begin{align*}
            &\norm*{x_1 + x_2 + y}^2 - \norm*{x_1 + x_2 - y}^2 - \norm*{x_1 + y}^2 - \norm*{x_2 + y}^2 + \norm*{x_1 - y}^2 + \norm*{x_2 - y}^2\\
            &= \norm*{x_1 + x_2 + y}^2 - \norm*{x_1 + x_2 - y}^2 - \norm*{x_1 + y}^2 - \norm*{x_2}^2 - \norm*{x_2 + y}^2 - \norm*{x_1}^2\\
            &\quad\, + \norm*{x_1 - y}^2 + \norm*{x_2}^2 + \norm*{x_2 - y}^2 + \norm*{x_1}^2\\
            &= \norm*{x_1 + x_2 + y}^2 - \norm*{x_1 + x_2 - y}^2\\
            &\quad\, - \frac{1}{2}\enbrace*{\norm*{x_1 + x_2 + y}^2 - \norm*{x_1 - x_2 + y}^2} - \frac{1}{2}\enbrace*{\norm*{x_1 + x_2 + y}^2 - \norm*{x_2 - x_1 + y}^2}\\
            &\quad\, + \frac{1}{2}\enbrace*{\norm*{x_1 + x_2 - y}^2 - \norm*{x_1 - x_2 + y}^2} - \frac{1}{2}\enbrace*{\norm*{x_1 + x_2 - y}^2 - \norm*{x_2 - x_1 + y}^2}\\
            &= 0
        \end{align*}
        Damit haben wir die Additivität gezeigt. Es bleibt die Aussage
        \[\norm*{\alpha{x} + y}^2 - \norm*{\alpha{x} - y}^2 = \alpha\enbrace*{\norm*{x + y}^2 - \norm*{x - y}^2}\]
        zu beweisen. Wir setzen in der vorherigen Rechnung \(x = x_1 = x_2\), somit haben wir
        \[\norm*{2x + y}^2 - \norm*{2x - y}^2 = 2\enbrace*{\norm*{x + y}^2 - \norm*{x - y}^2}\]
        bewiesen. Dann können wir die obige Aussage induktiv für alle \(\alpha \in \NN\) beweisen. Wegen
        \[\norm*{x + y}^2 - \norm*{x - y}^2 = n\enbrace*{\norm*{n^{-1}x + y}^2 - \norm*{n^{-1}x - y}^2}, \quad n \in \NN\]
        können wir die Aussage für alle \(q \in \QQ_{> 0}\) folgern. Dank der Additivität gilt die Aussage auf \(q \in \QQ\). Da die rationalen Zahlen dicht in der Menge der reellen Zahlen liegen, folgt die Behauptung.
    \end{enumerate}
\end{document}
