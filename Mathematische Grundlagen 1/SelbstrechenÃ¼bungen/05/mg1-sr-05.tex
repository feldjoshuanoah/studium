\documentclass[german,12pt]{homework}

\usepackage[ngerman]{babel}
\usepackage[utf8]{inputenc}
\usepackage[T1]{fontenc}

\newcommand{\NN}{\mathbb{N}}
\newcommand{\RR}{\mathbb{R}}

\newcommand{\dotproduct}[2]{\left\langle#1, #2\right\rangle}

\DeclarePairedDelimiter{\absolute}{\lvert}{\rvert}
\DeclarePairedDelimiter{\norm}{\lVert}{\rVert}
\DeclarePairedDelimiter{\enbrace}{(}{)}
\DeclarePairedDelimiter{\penbrace}{\{}{\}}

\title{Selbstrechenübung 5}
\author{Joshua Feld, 406718}
\institute{RWTH Aachen University\\Center for Computational Engineering Science}
\class{Mathematische Grundlagen I}
\professor{Prof. Dr. Torrilhon \& Prof. Dr. Stamm}

\begin{document}
    \maketitle

    \section*{Aufgabe 1. (Konvergenz von Folgen)}

    \begin{problem}
        Untersuchen Sie das Grenzwertverhalten folgender Folgen für \(n \to \infty\) und begründen Sie Ihre Antwort:
        \begin{enumerate}
            \item \(a_n := \frac{\enbrace*{2n + x}^4}{xn^4 - 5n^3 + 6}\), für \(x = 0\) und \(x \ne 0\)
            \item \(b_n := \enbrace*{\frac{2n}{2n - 2}}^n\)
        \end{enumerate}
    \end{problem}

    \subsection*{Lösung}
    \begin{enumerate}
        \item Wir formulieren zunächst die Folgenglieder um:
        \[a_n = \frac{\enbrace*{2n + x}^4}{xn^4 - 5n^3 + 6} = \frac{2n^4 \cdot \enbrace*{1 + \frac{x}{2n}}}{n^4 \cdot \enbrace*{x - \frac{5}{n} + \frac{6}{n^4}}} = \frac{16\enbrace*{1 + \frac{1}{2n}}^4}{x - \frac{5}{n} + \frac{6}{n^4}}.\]
        Für \(x = 0\) divergiert die Folge:
        \(\lim_{n \to \infty}a_n = -\infty.\)
        Für alle anderen \(x \in \RR \setminus \penbrace*{0}\) gilt \(\lim_{n \to \infty}a_n = \frac{16}{x}\), denn
        \[\lim_{n \to \infty}\enbrace*{1 + \frac{x}{2n}}^4 = 1 \quad \text{und} \quad \lim_{n \to \infty}\enbrace*{x - \frac{5}{n} + \frac{6}{n^4}} = x\]
        \item Bringt man den Ausdruck auf die Form
        \[b_n = \enbrace*{\frac{2n}{2n - 2}}^n = \enbrace*{\frac{n}{n - 1}}^n = \enbrace*{\frac{n + 1 - 1}{n - 1}}^n = \enbrace*{1 + \frac{1}{n - 1}}^n,\]
        so ergibt sich
        \begin{align*}
            \lim_{n \to \infty}b_n &= \lim_{n \to \infty}\enbrace*{1 + \frac{1}{n - 1}}^n = \lim_{n \to \infty}\enbrace*{1 + \frac{1}{n}}^{n + 1}\\
            &= \lim_{n \to \infty}\enbrace*{1 + \frac{1}{n}}^n \cdot \enbrace*{1 + \frac{1}{n}} = \lim_{n \to \infty}\enbrace*{1 + \frac{1}{n}}^n = e.
        \end{align*}
    \end{enumerate}

    \section*{Aufgabe 2. (\(\varepsilon\)-\(N\)-Kriterium)}

    \begin{problem}
        \begin{enumerate}
            \item Geben Sie die Definition der Konvergenz einer Folge durch das \(\varepsilon\)-\(N\)-Kriterium an.
            \item Seien \(\enbrace*{a_n}_{n \ge 1}\) und \(\enbrace*{b_n}_{n \ge 1}\) zwei konvergente reelle Folgen mit den entsprechenden Grenzwerten \(\lim_{n \to \infty}a_n = a \in \RR\) und \(\lim_{n \to \infty}b_n = b \in \RR\). Sei \(\lambda \in \RR\). Zeigen Sie anhand des \(\varepsilon\)-\(N\)-Kriteriums, dass die Folge \(\enbrace*{c_n}_{n \ge 1}\) mit \(c_n = \lambda\enbrace*{a_n + b_n}\) für \(n \to \infty\) gegen den Grenzwert \(\lambda\enbrace*{a + b}\) konvergiert.
        \end{enumerate}
    \end{problem}

    \subsection*{Lösung}
    \begin{enumerate}
        \item Eine reelle Folge \(\enbrace*{a_n}_{n \ge 1}\) heißt konvergent, wenn ein \(a \in \RR\) existiert, so dass zu jedem \(\varepsilon > 0\) ein \(N = N\enbrace*{\varepsilon} \in \NN\) existiert mit der Eigenschaft
        \[\absolute*{a_n - a} < \varepsilon \quad \text{für alle }n \ge N.\]
        Man nennt \(a\) Limes oder Grenzwert der Folge \(\enbrace*{a_n}_{n \ge 1}\) und schreibt \(\lim_{n \to \infty}a_n = a\) oder \(a_n \xrightarrow{n \to \infty} a\). Man sagt auch, dass \(\enbrace*{a_n}_{n \ge 1}\) gegen \(a\) konvergiert. Eine Folge die nicht konvergent ist, heißt divergent.
        \item Sei \(\varepsilon > 0\) beliebig. Dann exisitieren \(N_1, N_2 \in \NN\) für die gilt
        \[\absolute*{a_n - a} < \frac{\varepsilon}{2}\ \forall{n \ge N_1} \quad \text{und} \quad \absolute*{b_n - b} < \frac{\varepsilon}{2}\ \forall{n \ge N_2}.\]
        Wählen wir nun \(N = \max\penbrace*{N_1, N_2}\) und \(n \ge N\) beliebig. Daraus folgt
        \[\absolute*{\enbrace*{a_n + b_n} - \enbrace*{a + b}} = \absolute*{\enbrace*{a_n - a} + \enbrace*{b_n - b}} \le \absolute*{a_n - a} + \absolute*{b_n - b} < \frac{\varepsilon}{2} + \frac{\varepsilon}{2} = \varepsilon,\]
        wobei wir beim Schritt \(\le\) die aus der Vorlesung bekannte Dreiecksungleichung genutzt haben. Sei nun noch \(\lambda \in \RR\). Wir wollen zeigen, dass \(\lambda\enbrace*{a_n + b_n}\) gegen \(\lambda\enbrace*{a + b}\) konvergiert, d.h. wir müssen zeigen, dass \(\absolute*{\lambda\enbrace*{a_n + b_n} - \lambda\enbrace*{a + b}} < \varepsilon\) für alle \(n \ge N\) (\(N = N\enbrace*{\varepsilon} \in \NN\)) gilt:
        \begin{align*}
            \absolute*{\lambda\enbrace*{a_n + b_n} - \lambda\enbrace*{a + b}} < \varepsilon &\iff \absolute*{\lambda\enbrace*{\enbrace*{a_n + b_n} - \enbrace*{a + b}}} < \varepsilon\\
            &\iff \absolute*{\lambda} \cdot \absolute*{\enbrace*{a_n + b_n} - \enbrace*{a + b}} < \varepsilon
        \end{align*}
        Sei nun zunächst \(\lambda \ne 0\) (den Fall sehen wir uns danach an). Dann können wir weiter umformen zu \(\absolute*{\enbrace*{a_n + b_n} - \enbrace*{a + b}} < \frac{\varepsilon}{\absolute*{\lambda}}\). Da \(\absolute*{\enbrace*{a_n + b_n} - \enbrace*{a + b}}\) gegen \(0\) konvergiert, gibt es ein \(N \in \NN\), so dass \(\absolute*{\enbrace*{a_n + b_n} - \enbrace*{a + b}} < \frac{\varepsilon}{\absolute*{\lambda}}\) für alle \(n \ge N\) ist. Für den Fall \(\lambda = 0\) ist \(\lambda\enbrace*{a_n + b_n} = 0 \cdot \enbrace*{a_n + b_n} = 0\), also auch \(\lim_{n \to \infty}\lambda\enbrace*{a_n + b_n} = \lim_{n \to \infty}0 = 0\). Insgesamt konvergiert also die Folge \(\enbrace*{c_n}_{n \ge 1}\) gegen den Grenzwert \(\lambda\enbrace*{a + b}\).
    \end{enumerate}

    \section*{Aufgabe 3. (Lineare Anhängigkeit)}

    \begin{problem}
        Seien \(a, b \in \RR\) und \(v_1, v_2, v_3 \in \RR^4\) mit
        \[v_1 = \begin{pmatrix}1\\1\\0\\0\end{pmatrix}, \quad v_2 = \begin{pmatrix}0\\-1\\2\\0\end{pmatrix}, \quad v_3 = \begin{pmatrix}1\\0\\2a\\b\end{pmatrix}\]
        \begin{enumerate}
          \item Zeigen Sie, dass für \(b \ne 0\) die Vektoren \(v_1, v_2, v_3\) immer linear unabhängig sind.
          \item Finden Sie für \(b = 0\) ein \(a \in \RR\), so dass \(v_1, v_2, v_3\) linear abhängig sind.
        \end{enumerate}
    \end{problem}

    \subsection*{Lösung}

    \section*{Aufgabe 4. (Vorbereitung zur Gram-Schmidt Orthogonalisierung)}

    \begin{problem}
        Wir betrachten den Vektorraum \(V = \RR^3\) mit euklidischem Skalarprodukt \(\dotproduct{\cdot}{\cdot}\) und induzierter Norm \(\norm*{\cdot} = \sqrt{\dotproduct{\cdot}{\cdot}}\). Gegeben seien außerdem die drei Vektoren
        \[v_1 = \begin{pmatrix}3\\4\\0\end{pmatrix}, \quad v_2 = \begin{pmatrix}1\\2\\2\end{pmatrix}, \quad v_3 = \begin{pmatrix}-2\\0\\2\end{pmatrix}\]
        \begin{enumerate}
            \item Berechnen Sie zuerst die Projektion \(\tilde{v_2}\) von \(v_2\) auf \(v_1\). Berechnen Sie nun die zu \(v_1\) orthogonale Komponente von \(v_2\) und bezeichnen Sie diese als \(u_2\). Überprüfen Sie anschließend, ob \(u_2\) tatsächlich orthogonal zu \(v_1\) ist.
            \item Berechnen Sie die Projektion von \(v_3\) auf die Vektoren \(v_1\), \(v_2\) sowie \(u_2\). Berechnen Sie nun die zu \(v_1\) und \(u_2\) orthogonale Komponente von \(v_3\) und bezeichnen Sie diese als \(u_3\). Überprüfen Sie wiederum, ob \(u_3\) tatsächlich orthogonal zu \(v_1\) und \(u_2\) ist.
        \end{enumerate}
    \end{problem}

    \subsection*{Lösung}
    \begin{enumerate}
        \item Bestimme zunächst die Projektion von \(v_2\) auf \(v_1\):
        \[\tilde{v_2} = \frac{\dotproduct{v_1}{v_2}}{\dotproduct{v_1}{v_1}}v_1 = \frac{\dotproduct{\begin{pmatrix}3\\4\\0\end{pmatrix}}{\begin{pmatrix}1\\2\\2\end{pmatrix}}}{\dotproduct{\begin{pmatrix}3\\4\\0\end{pmatrix}}{\begin{pmatrix}3\\4\\0\end{pmatrix}}} \cdot \begin{pmatrix}3\\4\\0\end{pmatrix} = \frac{11}{25} \cdot \begin{pmatrix}3\\4\\0\end{pmatrix}.\]
        Die orthogonale Komponente \(u_2\) ist gegeben durch die Differenz von \(v_2\) und \(\tilde{v_2}\):
        \[u_2 = v_2 - \tilde{v_2} = \begin{pmatrix}1\\2\\2\end{pmatrix} - \frac{11}{25} \cdot \begin{pmatrix}3\\4\\0\end{pmatrix} = \frac{1}{25} \cdot \begin{pmatrix}-8\\6\\50\end{pmatrix}.\]
        Durch Nachrechnen sehen wir, dass
        \[\dotproduct{u_2}{v_1} = \dotproduct{\frac{1}{25} \cdot \begin{pmatrix}-8\\6\\50\end{pmatrix}}{\begin{pmatrix}3\\4\\0\end{pmatrix}} = \frac{1}{25}\enbrace*{-24 + 24 + 0} = 0.\]
        Somit ist \(u_2\) orthogonal zu \(v_1\).
        \item Projektion von \(v_3\) auf \(v_1\):
        \[\tilde{v_3} = \frac{\dotproduct{v_1}{v_3}}{\dotproduct{v_1}{v_1}}v_1 = \frac{\dotproduct{\begin{pmatrix}3\\4\\0\end{pmatrix}}{\begin{pmatrix}-2\\0\\2\end{pmatrix}}}{\dotproduct{\begin{pmatrix}3\\4\\0\end{pmatrix}}{\begin{pmatrix}3\\4\\0\end{pmatrix}}} \cdot \begin{pmatrix}3\\4\\0\end{pmatrix} = \frac{1}{25} \cdot \begin{pmatrix}-18\\-24\\0\end{pmatrix}.\]
        Projektion von \(v_3\) auf \(v_2\):
        \[\hat{v_3} = \frac{\dotproduct{v_2}{v_3}}{\dotproduct{v_2}{v_2}}v_2 = \frac{\dotproduct{\begin{pmatrix}1\\2\\2\end{pmatrix}}{\begin{pmatrix}-2\\0\\2\end{pmatrix}}}{\dotproduct{\begin{pmatrix}1\\2\\2\end{pmatrix}}{\begin{pmatrix}1\\2\\2\end{pmatrix}}} \cdot \begin{pmatrix}1\\2\\2\end{pmatrix} = \frac{1}{9} \cdot \begin{pmatrix}2\\4\\4\end{pmatrix}.\]
        Projektion von \(v_3\) auf \(u_2\):
        \begin{align*}
            \tilde{u_2} = \frac{\dotproduct{u_2}{v_3}}{\dotproduct{u_2}{u_2}}u_2 &= \frac{\dotproduct{\frac{1}{25} \cdot \begin{pmatrix}-8\\6\\50\end{pmatrix}}{\begin{pmatrix}-2\\0\\2\end{pmatrix}}}{\dotproduct{\frac{1}{25} \cdot \begin{pmatrix}-8\\6\\50\end{pmatrix}}{\frac{1}{25} \cdot \begin{pmatrix}-8\\6\\50\end{pmatrix}}} \cdot \frac{1}{25} \cdot \begin{pmatrix}-8\\6\\50\end{pmatrix}\\
            &= \frac{\dotproduct{\begin{pmatrix}-8\\6\\50\end{pmatrix}}{\begin{pmatrix}-2\\0\\2\end{pmatrix}}}{\dotproduct{\begin{pmatrix}-8\\6\\50\end{pmatrix}}{\begin{pmatrix}-8\\6\\50\end{pmatrix}}} \cdot \begin{pmatrix}-8\\6\\50\end{pmatrix} = \frac{29}{325} \cdot \begin{pmatrix}-4\\3\\25\end{pmatrix}.
        \end{align*}
        Orthogonale Komponente \(u_3\):
        \[u_3 = v_3 - \tilde{u_2} - \tilde{v_3} = \begin{pmatrix}-2\\0\\2\end{pmatrix} - \frac{29}{325} \cdot \begin{pmatrix}-4\\3\\25\end{pmatrix} - \frac{1}{25} \cdot \begin{pmatrix}-18\\-24\\0\end{pmatrix} = \frac{1}{13} \cdot \begin{pmatrix}-12\\9\\-3\end{pmatrix}.\]
        Nachrechnen zeigt, dass \(u_3\) orthogonal zu \(v_1\) und \(u_2\) ist:
        \begin{align*}
            \dotproduct{u_2}{v_1} &= \dotproduct{\frac{1}{13} \cdot \begin{pmatrix}-12\\9\\-3\end{pmatrix}}{\frac{1}{25} \cdot \begin{pmatrix}-8\\6\\50\end{pmatrix}} = \frac{1}{13} \cdot \frac{1}{25} \cdot \enbrace*{96 + 54 - 150} = 0,\\
            \dotproduct{u_3}{v_1} &= \dotproduct{\frac{1}{13} \cdot \begin{pmatrix}-12\\9\\-3\end{pmatrix}}{\begin{pmatrix}3\\4\\0\end{pmatrix}} = \frac{1}{13}\enbrace*{-36 + 36 + 0} = 0.
        \end{align*}
    \end{enumerate}
\end{document}
