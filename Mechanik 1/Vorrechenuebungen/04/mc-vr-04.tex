\documentclass[german,12pt]{homework}

\usepackage[ngerman]{babel}
\usepackage[utf8]{inputenc}
\usepackage[T1]{fontenc}

\usepackage{tikz}

\usepackage{amsmath}
\usepackage{mathtools}
\usepackage{amssymb}
\usepackage{icomma}

\DeclarePairedDelimiter{\enbrace}{(}{)}
\DeclarePairedDelimiter{\benbrace}{[}{]}

\usepackage{chemformula}
\usepackage{siunitx}
\newcommand{\sis}[1]{\,\si{#1}}
\newcommand{\degC}{\si{\degreeCelsius}}

\sisetup{per-mode=fraction,sticky-per}

\title{Vorrechenübung 4}
\author{Joshua Feld, 406718}
\institute{RWTH Aachen University\\Lehrstuhl für computergestützte Analyse
technischer Systeme}
\class{Mechanik I}
\professor{Prof. Dr. Behr}

\begin{document}
    \maketitle

    \section*{Aufgabe 1. (Stationäre Wärmeleitung)}

    \begin{problem}
        \begin{center}
            \begin{tikzpicture}
                \draw[-] (0,0) -- (-2.5,5);
                \draw[->] (0,0) -- (-.75,1.5) node[below left] {\(P\)};
                \draw[-] (0,0) -- (3,-1.5);
                \draw[->] (0,0) -- (1,-0.5) node[below left] {\(Q\)};
                \draw[dashed] (0,0) -- (-2.5,0) -- (-2.5,5) -- (0,5) -- cycle;
                \draw[dashed] (0,0) -- (3,0) -- (3,-1.5) -- (0,-1.5) -- cycle;
                \draw node at (-3.1,2.5) {\(32\sis{\meter}\)};
                \draw node at (-1.25,5.3) {\(16\sis{\meter}\)};
                \draw node at (1.5,-1.8) {\(24\sis{\meter}\)};
                \draw node at (3.55,-.75) {\(12\sis{\meter}\)};
                \draw node at (2,4) {\(P = 76\sis{\kilo\newton}\)};
                \draw node at (2,3.45) {\(Q = 52\sis{\kilo\newton}\)};
                \draw node at (2.5,2.5) {Berechne \(P + Q\)};

            \end{tikzpicture}
        \end{center}
    \end{problem}

    \subsection*{Lösung}

    \section*{Aufgabe 2. (Wärmeausdehnung I)}

    \begin{problem}
        Ein Würfel aus einem unbekannten Stoff ist komplett mit
        \(1\sis{\liter}\) bei \(3,98\degC\) gefüllt. Die Seiten des Würfels
        haben eine Dicke von \(1\sis{\centi\meter}\). Welchen Wert muss der
        Linearausdehnungskoeffizient haben, damit der Würfel bei \(0\degC\)
        immer noch komplett gefüllt ist? Der Raumausdehnungskoeffizient von
        Wasser in diesem Temperaturbereich beträgt ungefähr \(-0,5 \cdot 10^{-4}
        \sis{\per\kelvin}\). Halten Sie den errechneten Wert für sinnvoll?
    \end{problem}

    \subsection*{Lösung}

    \section*{Aufgabe 3. (Wärmeausdehnung II)}

    \begin{problem}
        Aufgrund von Temperaturschwankungen zwischen Sommer und Winter und der
        damit verbundenen Längenänderung von metallischen Bauteilen, werden
        beim Bau von Brücken üblicherweise Dehnungsfugen eingeplant. Die
        Ingenieure von Impractical Ideas, Inc. entwickeln eine alternative
        Methode: Die Hauptkonstruktion einer Versuchsbrücke der Länge
        \(100\sis{\meter}\) besteht aus Baustahl mit einem
        Linearausdehnungskoeffizienten \(\alpha_S = 1,2 \cdot 10^{-5}
        \sis{\per\kelvin}\). An einem Brückenkopf wird ein Konstruktionselement
        der Länge \(1\sis{\meter}\) aus einer speziellen Legierung verbaut. Die
        Legierung besitzt einen besonders hohen Linearausdehnungskoeffizienten
        \(\alpha_L = 4,2 \cdot 10^{-4}\sis{\per\kelvin}\). Die Brücke wird im
        Winter bei einer Temperatur von \(-20\degC\) spannungsfrei montiert und
        das spezielle Konstruktionselement soll im Sommer so gekühlt werden,
        dass die Brücke bis zu einer Temperatur von \(30\degC\) spannungsfrei
        bleibt. Berechnen Sie die minimale Temperatur, die das spezielle
        Konstruktionselement erreichen können muss.
    \end{problem}

    \subsection*{Lösung}
\end{document}
