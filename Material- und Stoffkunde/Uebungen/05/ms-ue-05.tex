\documentclass[german,12pt]{homework}

\usepackage[ngerman]{babel}
\usepackage[utf8]{inputenc}
\usepackage[T1]{fontenc}

\usepackage{amsmath}
\usepackage{mathtools}
\usepackage{amssymb}
\usepackage{icomma}

\newcommand{\dd}{\,\differ}
\DeclareMathOperator{\differ}{d}

\DeclarePairedDelimiter{\enbrace}{(}{)}
\DeclarePairedDelimiter{\benbrace}{[}{]}

\usepackage{chemformula}
\usepackage{siunitx}
\newcommand{\sis}[1]{\,\si{#1}}
\newcommand{\degC}{\si{\degreeCelsius}}

\sisetup{per-mode=fraction,sticky-per}

\DeclareSIUnit{\calorie}{cal}

\title{Übung 5}
\author{Joshua Feld, 406718}
\institute{RWTH Aachen University\\Aachener Verfahrenstechnik}
\class{Material- und Stoffkunde}
\professor{Prof. Dr. Gebhardt}

\begin{document}
    \maketitle

    \section*{Aufgabe 1. (Newtonsches Gesetz des Abkühlens)}

    \begin{problem}
        Eine zylindrische Tasse mit Höhe \(12\,\si{cm}\) und Durchmesser
        \(10\,\si{cm}\) ist randvoll mit Kaffee bei einer Temperatur von
        \(90\si{\degreeCelsius}\). Um eine Trinktemperatur von
        \(60\si{\degreeCelsius}\) zu erreichen wird die Tasse in einer Umgebung
        mit einer Temperatur von \(20\si{\degreeCelsius}\) abgestellt. Nach
        \(5\,\si{\minute}\) hat der Kaffee eine Temperatur von
        \(70\si{\degreeCelsius}\).
        \begin{enumerate}
            \item Berechnen Sie die Kühlrate \(k_0\) nach dem Newtonschen
            Gesetz des Abkühlens.
            \item Bestimmen Sie die Zeit, nach der der Kaffee die gewünschte
            Temperatur hat.
            \item Nehmen Sie an, die Kühlrate ließe sich durch Anpusten der
            Tasse mit eine Strömungsgeschwindigkeit \(v\) nach dem Ansatz
            \[k = k_0\enbrace*{1 + \frac{v}{1\,\si{\metre\per\second}}}\]
            erhöhen. Welche Strömungsgeschwindigkeit muss erreicht werden,
            damit der Kaffee bereits nach \(5\,\si{\minute}\) die gewünschte
            Temperatur hat?
        \end{enumerate}
    \end{problem}

    \subsection*{Lösung}
    \begin{enumerate}
        \item Gegeben sind die Umgebungstemperatur \(\Theta_a = 20\degC\), die
        Anfangstemperatur \(\Theta_0 = 90\degC\) und die Temperatur
        \(\Theta\enbrace*{t_1} = 70\degC\) nach \(t_1 = 5\,\si{\min}\). Mit dem
        Newtonschen Gesetz des Abkühlens ergibt sich
        \[T\enbrace*{t_1} = T_a + \enbrace*{T_0 - T_a}\exp\enbrace*{-k_0t_1}\]
        und nach Umformen
        \[k_0 = -\frac{1}{t_1}\ln\enbrace*{\frac{T\enbrace*{t_1} - T_a}{T_0 -
        T_a}} = -\frac{1}{5\sis{\minute}}\ln\enbrace*{\frac{70\degC - 20\degC}
        {90\degC - 20\degC}} = 0,0673\sis{\per\minute}.\]
        \item  Gegeben sind die Umgebungstemperatur \(\Theta_a = 20\degC\), die
        Anfangstemperatur \(\Theta_0 = 90\degC\) und die gewünschte Temperatur
        \(\Theta\enbrace*{t_2} = 60\degC\). Mit dem Newtonschen Gesetz des
        Abkühlens ergibt sich
        \[T\enbrace*{t_2} = T_a + \enbrace*{T_0 - T_a}\exp\enbrace*{-k_0t_2}\]
        und nach Umformen
        \[t_2 = -\frac{1}{k_0}\ln\enbrace*{\frac{T\enbrace*{t_2} - T_a}{T_0 -
        T_a}} = -\frac{1}{0,0673\sis{\per\minute}}\ln\enbrace*{\frac{60\degC -
        20\degC}{90\degC - 20\degC}} = 8,315\sis{\minute}.\]
        \item Gegeben sind die Umgebungstemperatur \(\Theta_a = 20\degC\), die
        Anfangstemperatur \(\Theta_0 = 90\degC\) und die Temperatur
        \(\Theta\enbrace*{t_3} = 60\degC\) nach \(t_3 = 5\,\si{\min}\). Analog
        zur ersten Teilaufgabe kann die Kühlrate bestimmt werden zu
        \[k = -\frac{1}{t_3}\ln\enbrace*{\frac{T\enbrace*{t_3} - T_a}{T_0 -
        T_a}} = -\frac{1}{5\sis{\minute}}\ln\enbrace*{\frac{60\degC - 20\degC}
        {90\degC - 20\degC}} = 0,1192\sis{\per\minute}.\]
        Es folgt für die Strömungsgeschwindigkeit
        \[v = 1\sis{\meter\per\second}\enbrace*{\frac{k}{k_0} - 1} =
        1\sis{\meter\per\second}\enbrace*{\frac{0,1192\sis{\per\minute}}{0,0673\sis{\per\minute}} - 1} = 0,663\sis{\per\minute}.\]
    \end{enumerate}

    \section*{Aufgabe 2. (Revision: Erwärmung von Brokkoli)}

    \begin{problem}
        Im Betrieb produziert ein Computerprozessor eine Wärmeleistung von
        \(35\,\si{\watt}\). Die Oberfläche des Prozessors beträgt
        \(16\,\si{\centi\meter\squared}\). Im Folgenden soll die
        Prozessortemperatur berechnet werden, die sich einstellt, wenn keine
        Kühllösung zum Einsatz kommt.
        \begin{enumerate}
            \item Berechnen Sie die Prozessortemperatur, wenn die
            Umgebungstemperatur \(20\si{\degreeCelsius}\) beträgt und der
            Wärmeübergang zwischen der Prozessoroberfläche und Luft durch einen
            Wärmeübergangskoeffizienten von
            \(8\,\si{\watt\per\meter\squared\kelvin}\) gegeben ist.
            Vernachlässigen Sie dabei den Einfluss von Wärmestrahlung.
            \item Berechnen Sie die Prozessortemperatur unter der
            ausschließlichen Betrachtung von Wärmestrahlung und einem
            Emissionsgrad \(\varepsilon = 0,9\) der Prozessoroberfläche.
            Vernachlässigen Sie den Einfluss von eingehender Wärmestrahlung.
        \end{enumerate}
    \end{problem}

    \subsection*{Lösung}
    \begin{enumerate}
        \item Gegeben sind der Wärmestrom \(\dot{Q} = 35\sis{\watt}\), die
        Umgebungstemperatur \(\Theta_a = 20\degC\), der
        Wärmeübergangskoeffizient \(h = 8\sis{\watt\per\meter\squared\kelvin}\)
        und die Prozessoroberfläche \(A = 16\sis{\centi\meter\squared}\). Da
        ausschließlich Wärmeleistung betrachtet wird, gilt für den Wärmestrom
        \[\dot{Q} = hA\enbrace*{\Theta - \Theta_a}.\]
        Aufgelöst nach der Prozessortemperatur \(\Theta\) ergibt sich
        \[T = \frac{\dot{Q}}{hA} + \Theta_a = \frac{35\sis{\watt}}{8
        \sis{\watt\per\meter\squared\kelvin} \cdot 16
        \sis{\centi\meter\squared}} + 20\degC = 2\,764,38\degC.\]
        \item Gegeben sind der Wärmestrom \(\dot{Q} = 35\sis{\watt}\), die
        Umgebungstemperatur \(\Theta_a = 20\degC\), der Emissionsgrad
        \(\varepsilon = 0,9\) und die Prozessoroberfläche \(A = 16
        \sis{\centi\meter\squared}\). Da ausschließlich Wärmestrahlung
        betrachtet wird, gilt für den Wärmestrom
        \[\dot{Q} = \varepsilon\sigma{A}T^4.\]
        Aufgelöst nach der Prozessortemperatur \(T\) ergibt sich
        \[T = \enbrace*{\frac{\dot{Q}}{\varepsilon\sigma{A}}}^\frac{1}{4}
        = \enbrace*{\frac{35\sis{\watt}}{0,9 \cdot 5,67 \cdot 10^{-8}
        \sis{\watt\per\meter\squared\kelvin\tothe{4}} \cdot 16
        \sis{\centi\meter\squared}}}^\frac{1}{4} = 809,15\sis{\kelvin}.\]
    \end{enumerate}

    \section*{Aufgabe 3. (Wärmekapazität von Gasen)}

    \begin{problem}
        Um die Temperatur von Computerprozessoren im Betrieb gering zu halten
        (unter \(80\si{\degreeCelsius}\)) wird häufig passive Luftkühlung
        eingesetzt. Dabei wird ein Kühlkörper mit großer Oberfläche in Form von
        Kühlrippen auf den Prozessor aufgesetzt um die effektive Kühloberfläche
        zu erhöhen. Im Folgenden wird ein Kupferkühlkörper der Masse
        \(1\,\si{\kilogram}\) betrachtet. Die Umgebungstemperatur beträgt
        \(20\si{\degreeCelsius}\). Nehmen Sie an, dass die Wärmeleistung in
        Kupfer ausreichend schnell abläuft und dass die Prozessortemperatur
        näherungsweise der Oberflächentemperatur des Kühlkörpers entspricht.
        Gehen Sie wie in Aufgabe 2 von einem Wärmeübergangskoeffizienten von
        \(8\,\si{\watt\per\meter\squared\kelvin}\) aus.
        \begin{enumerate}
            \item Berechnen Sie für den Prozessor aus Aufgabe 2 die benötigte
            Kühlkörperoberfläche damit eine Temperatur von
            \(80\si{\degreeCelsius}\) nicht überschritten wird.
            \item Berechnen Sie die Kühlrate des Kühlkörpers nach dem
            Newtonschen Gesetz des Abkühlens, die das Abkühlen nach dem Betrieb
            beschreibt.
            \item Wie lange dauert es bis der Kühlkörper nach dem Betrieb bei
            \(80\si{\degreeCelsius}\) auf \(40\si{\degreeCelsius}\) abgekühlt
            ist?
        \end{enumerate}
    \end{problem}

    \subsection*{Lösung}
    \begin{enumerate}
        \item Gegeben sind der Wärmestrom \(\dot{Q} = 35\sis{\watt}\), die
        Umgebungstemperatur \(T_a = 20\degC\), die Prozessortemperatur
        \(T = 80\degC\) und der Wärmeübergangskoeffizient
        \(h = 8\sis{\watt\per\meter\squared\kelvin}\). Wir erwarten, dass
        die Wärmeübertragung aufgrund des Kühlkörpers von Wärmeleistung
        dominiert wird und dass der Einfluss von Wärmestrahlung zu
        vernachlässigen ist. Dann ergibt sich für den Wärmestrom
        \[\dot{Q} = hA\enbrace*{T - T_a}\]
        und somit für die Kühloberfläche
        \[A = \frac{\dot{Q}}{h\enbrace*{T - T_a}} =
        \frac{35\sis{\watt}}{8\sis{\watt\per\meter\squared\kelvin} \cdot
        60\sis{\kelvin}} = 0,0729\sis{\meter\squared}.\]
        \item Nach dem Newtonschen Gesetz des Abkühlens gilt
        \begin{equation}\label{eq:1}
            \frac{\dd{T}}{\dd{t}} = -k\enbrace*{T - T_a}.
        \end{equation}
        Wir nehmen an, dass der Abkühlvorgang von Wärmeleistung dominiert wird.
        Damit gilt
        \begin{equation}\label{eq:2}
            \frac{\dd{Q}}{\dd{t}} = -hA\enbrace*{T - T_a}.
        \end{equation}
        Die Wärmekapazität von Kupfer bei \(298\sis{\kelvin}\) kann der
        Tabelle aus dem Vorlesungsumdruch mit dem Wert
        \[c_p = 0,383\sis{\joule\per\gram\kelvin}\]
        entnommen werden. Nehmen wir an, dass diese konstant mit der Temperatur
        ist, folgt für die innere Energie des Kühlkörpers
        \[Q = c_pmT.\]
        Somit lässt sich Gleichung \eqref{eq:2} umformen zu
        \[\frac{\dd{T}}{\dd{t}} = -\frac{hA}{c_pm}\enbrace*{T - T_a}.\]
        Durch Vergleich mit Gleichung \eqref{eq:1} folgt
        \[k = \frac{hA}{mc_p} = \frac{8\sis{\watt\per\meter\squared\kelvin}
        \cdot 0,0729\sis{\meter\squared}}{1\sis{\kilogram} \cdot 0,383
        \sis{\joule\per\gram\kelvin}} = 0,00152\sis{\per\second}.\]
        \item Nach dem Newtonschen Gesetz des Abkühlens ergibt sich
        \[T\enbrace*{t} = T_a + \enbrace*{T_0 + T_a}\exp\enbrace*{-kt}\]
        mit der Kühlrate aus der vorherigen Teilaufgabe. Durch Umformen nach der
        Kühlzeit ergibt sich
        \[t = -\frac{1}{k}\ln\enbrace*{\frac{T\enbrace*{t} - T_a}{T_0 - T_a}}
        = -\frac{1}{0,00152\sis{\per\second}}\ln\enbrace*{\frac{40\degC -
        20\degC}{80\degC - 20\degC}} = 723\sis{\second}.\]
    \end{enumerate}
\end{document}
