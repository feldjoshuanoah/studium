\documentclass[german,12pt]{homework}

\usepackage[ngerman]{babel}
\usepackage[utf8]{inputenc}
\usepackage[T1]{fontenc}

\usepackage{amsmath}
\usepackage{mathtools}
\usepackage{amssymb}
\usepackage{icomma}

\newcommand{\dd}{\,\differ}
\DeclareMathOperator{\differ}{d}

\DeclarePairedDelimiter{\enbrace}{(}{)}
\DeclarePairedDelimiter{\benbrace}{[}{]}

\usepackage{chemformula}
\usepackage{siunitx}

\DeclareSIUnit{\calorie}{cal}

\title{Übung 4}
\author{Joshua Noah Feld, 406718}
\institute{RWTH Aachen University\\Aachener Verfahrenstechnik}
\class{Material- und Stoffkunde}
\professor{Prof. Dr. Gebhardt}

\begin{document}
    \maketitle

    \section*{Aufgabe 1. (Dampfgenerierung für einen Autoklav)}

    \begin{problem}
        Zur Sterilisierung von medizinischer Ausrüstung wird eine Wassermenge
        von \(200\,\si{\liter}\) von einer Temperatur von
        \(20\si{\degreeCelsius}\) auf \(100\si{\degreeCelsius}\) erwärmt,
        verdampft und auf \(121\si{\degreeCelsius}\) überhitzt. Schätzen Sie
        die notwendige Wärmemenge ab. Die Verdampfungsenthalpie von Wasser
        beträgt \(\Delta_\text{vap}H = 2\,260\,\si{\joule\per{\g}}\). Beachten
        Sie, dass die Wärmekapazität des Wasserdampfes über den gegebenen
        Temperaturbereich \emph{nicht} als konstant angenommen werden kann.
    \end{problem}

    \subsection*{Lösung} Wir machen folgende Annahmen:
    \begin{itemize}
        \item Die Dichte von Wasser bei \(20\si{\degreeCelsius}\) ist \(\rho =
        1\,\si{\kg\per{\liter}}\).
        \item Es treten keine Wärmeverluste während des Prozesses auf.
        \item Die Wärmekapazität bei konstantem Druch von flüssigem Wasser ist
        konstant.
        \item Der Prozess findet bei Standarddruck statt.
    \end{itemize}
    Der Prozess lässt sich in drei Unterprozesse unterteilen.
    \begin{enumerate}
        \item Zunächst wird das Wasser von \(20\si{\degreeCelsius}\) auf
        \(100\si{\degreeCelsius}\) erwärmt. Dazu ist die Wärmemenge
        \[Q_1 = c_pm{\Delta}T_1 = c_p\rho{V}{\Delta}T_1\]
        mit \({\Delta}T_1 = T_2 - T_1 = 80\,\si{\kelvin}\) notwendig. Dabei
        ergibt sich die spezifische Wärmekapazität \(c_p\) aus der Tabelle im
        Vorlesungsumdruck zu \(c_p = 4,181\,\si{\joule\per{\g\kelvin}}\). Es
        folgt
        \[Q_1 = c_p\rho{V}{\Delta}T_1 = 4,181\,\si{\joule\per{\g\kelvin}} \cdot
        1\,\si{\kg\per{\liter}} \cdot 200\,\si{\liter} \cdot 80\,\si{\kelvin} =
        66,9\,\si{\mega\joule}.\]
        \item Anschließend wird das Wasser bei \(100\si{\degreeCelsius}\)
        verdapft. Die benötigte Wärmemenge \(Q_2\) ergibt sich mit der
        Verdampfungsenthalpie \(\Delta_\text{vap}H\) zu
        \[Q_2 = m\Delta_\text{vap}H = \rho{V}\Delta_\text{vap}H =
        1\,\si{\kg\per{\liter}} \cdot 200\,\si{\liter} \cdot
        2\,260\,\si{\joule\per{\g}} = 452\,\si{\mega\joule}.\]
        \item Schließlich wird der Wasserdampf von \(100\si{\degreeCelsius}\)
        (\(T_2 = 373\,\si{\kelvin}\)) auf \(121\si{\degreeCelsius}\) (\(T_3 =
        394\,\si{\kelvin}\)) überhitzt. Mit der temperaturabhängigen molaren
        Wärmekapazität bei konstantem Druck \(c_{m, p}\enbrace*{T}\) ergibt
        sich die benötigte Wärmemenge \(Q_3\) zu
        \[Q_3 = \int_{T_2}^{T_3}c_{m, p}\enbrace*{T}n{\dd}T =
        \int_{T_2}^{T_3}c_{m, p}\enbrace*{T}\frac{\rho{V}}{M}{\dd}T.\]
        Entsprechend dem polynomiellen Ansatz aus dem Vorlesungsumdruck ergibt
        sich \(c_{m, p}\enbrace*{T}\) zu
        \[c_{m, p}\enbrace*{T} = R\enbrace*{C_0 + C_1T + C_2T^2 + C_3T^3 +
        C_4T^4}\]
        mit den Koeffizientenwerten \(C_0 = 4,07\), \(C_1 = -1,108 \cdot 10^{-3}
        \,\si{\per\kelvin}\), \(C_2 = -4,152 \cdot 10^{-6}
        \,\si{\per\kelvin\squared}\), \(C_3 = -2,964 \cdot 10^{-9}
        \,\si{\per\kelvin\cubed}\) und \(C_4 = 0,807 \cdot 10^{-12}
        \,\si{\per\kelvin\tothe{4}}\). Durch Auflösung des Integrals ergibt sich
        \begin{align*}
            Q_3 &= \frac{R\rho{V}}{M}\int_{T_2}^{T_3}C_0 + C_1T + C_2T^2 +
            C_3T^3 + C_4T^4{\dd}T\\
            &= \frac{R\rho{V}}{M}\benbrace*{C_0T + \frac{1}{2}C_1T^2 +
            \frac{1}{3}C_2T^3 + \frac{1}{4}C_3T^4 +
            \frac{1}{5}C_4T^5}_{T_2}^{T_3}\\
            &= \frac{8,314\,\si{\joule\per{\mole\kelvin}} \cdot
            1\,\si{\kg\per{\liter}} \cdot
            200\,\si{\liter}}{18,02\,\si{\g\per{\mole}}}\left[4,07\,T + \frac{-
            1,108 \cdot 10^{-3}\,\si{\per\kelvin}}{2}T^2\right.\\
            &\ \ \ + \left.\frac{4,152 \cdot 10^{-
            6}\,\si{\per\kelvin\squared}}{3}T^3 + \frac{-2,964 \cdot 10^{-
            9}\,\si{\per\kelvin\cubed}}{4}T^4 + \frac{0,807 \cdot 10^{-
            12}\,\si{\per\kelvin\tothe{4}}}{5}T^5\right]_{373\,\si{\kelvin}}^{39
            4\,\si{\kelvin}}\\
            & = 7,96\,\si{\mega\joule}.
        \end{align*}
    \end{enumerate}
    Somit ergibt sich die gesamte Wärmemenge zu
    \[Q = Q_1 + Q_2 + Q_3 = 66,9\,\si{\mega\joule} + 452\,\si{\mega\joule} +
    7,96\,\si{\mega\joule} = 526,86\,\si{\mega\joule}.\]

    \section*{Aufgabe 2. (Revision: Erwärmung von Brokkoli)}

    \begin{problem}
        In Übung 3 haben Sie folgende Aufgabe gelöst: Bei vielen Lebensmitteln
        unterscheidet sich die Wärmekapazität im gefrorenen Zustand von der im
        aufgetauten Zustand. Brokkoli zum Beispiel hat eine Wärmekapazität von
        \(3,85\,\si{\kilo\joule\per{\kg\kelvin}}\) unter der Schmelztemperatur
        von Wasser und nur \(1,84\,\si{\kilo\joule\per{\kg\kelvin}}\) oberhalb
        der Schmelztemperatur. Wie viel Energie muss man aufwenden, um die
        Temperatur von \(500\,\si{\g}\) Brokkoli von \(-18\si{\degreeCelsius}\)
        auf \(50\si{\degreeCelsius}\) zu erhöhen? Ignorieren Sie bei der
        Berechnung den Einfluss der Schmelzwärme von Wasser.

        Betrachten Sie den selben Fall erneut ohne die Annahme
        vernachlässigbarer Schmelzwärme des Wassers. Nehmen Sie an, dass
        Brokkoli ausschließlich aus Wasser (Index \(W\)) und Biomasse (Index
        \(B\)) in einem Massenverhältnis \(\frac{m_B}{m_W} = 0,25\) besteht.
        Die Biomasse liegt im gesamten betrachteten Temperaturbereich fest vor.
    \end{problem}

    \subsection*{Lösung} Die Lösung der ursprünglichen Aufgabe ergab eine
    benötigte Wärmemenge von \(Q_\text{alt} = 80,65\,\si{\kilo\joule}\).
    Zusätzlich muss nun das Schmelzen des Wasseranteils betrachtet werden. Die
    Wassermasse \(m_W\) beträgt
    \[m_W = m\frac{m_W}{m_W + m_B} = m\enbrace*{\frac{m_W + m_B}{m_W}}^{-1} =
    m\enbrace*{1 + \frac{m_B}{m_W}}^{-1} = 500\,\si{\g}\enbrace*{1 + 0,25}^{-1}
    = 400\,\si{\g}.\]
    Mit der Schmelzwärme von Wasser \(\Delta_\text{fus}H =
    333,55\,\si{\joule\per{\g}}\) ergibt sich die zusätzlich erforderliche
    Wärmemenge \(Q_\text{fus}\) zu
    \[Q_\text{fus} = \Delta_\text{fus}Hm_W = 333,55\,\si{\joule\per{\g}} \cdot
    400\,\si{\g} = 133,42\,\si{\kilo\joule}\]
    und die gesamte benötigte Wärmemenge zu
    \[Q = Q_\text{alt} + Q_\text{fus} = 80,65\,\si{\kilo\joule} +
    133,42\,\si{\kilo\joule} = 214,07\,\si{\kilo\joule}.\]

    \section*{Aufgabe 3. (Wärmekapazität von Gasen)}

    \begin{problem}
        Ein Mol Kohlenstoffdioxid (\ch{CO2}) befindet sich in einem isolierten
        Zylinder. Durch einen beschwerten Kolben wird der Druck konstant bei
        \(2\,\si{\bar}\) gehalten. Dem Zylinder wird eine geringe, präzise
        dosierte Wärmemenge zugeführt, sodass sich der Kolben hebt. Das gleiche
        Experiment mit der gleichen Wärmemenge wird mit einem Mol Helium
        (\ch{He}) durchgeführt. Erwarten Sie, dass sich der Kolben um den
        gleichen, einen geringeren oder einen höheren Betrag hebt? Begründen
        Sie ihre Antwort, eine Rechnung ist nicht notwendig.
    \end{problem}

    \subsection*{Lösung} Aufgrund des geringen Drucks nehmen wir an, dass sich
    \ch{CO2} und \ch{He} als ideale Gase verhalten. Damit nehmen beide Gase vor
    der Wärmezufuhr das gleiche Volumen ein. Darüber hinaus ist das Volumen der
    Gase proportional zur absoluten Temperatur. Wie wir in der Vorlesung
    gelernt haben, besitzt \ch{CO2} als dreiatomiges Gas mehr Freiheitsgrade
    der Bewegung als das einatomige \ch{He}. Daraus ergibt sich, dass \ch{He}
    eine geringere Wärmekapazität als \ch{CO2} besitzt. Es folgt, dass bei
    Zufabe der gleichen Wärmemenge die Temperatur von \ch{He} um einen größeren
    Betrag steigt als die Temperatur von \ch{CO2}. Daraus folgt eine größere
    Volumenausdehnung und das Anheben des Kolbens um einen größeren Betrag.

    \section*{Aufgabe 4. (Kalorie)}

    \begin{problem}
        Die Kalorie (\(\si{\calorie}\)) ist eine Maßeinheit der Energie. Eine
        von mehreren Definitionen der Kalorie ist die Wärmemenge, die benötigt
        wird, um \(1\,\si{\g}\) Wasser von einer Temperatur von
        \(14,5\si{\degreeCelsius}\) auf \(15,5\si{\degreeCelsius}\) zu
        erwärmen. Bestimmen Sie den Umrechnungsfaktor, mit dem Sie eine
        Energiemenge von Kalorien in Joule umrechnen können.
    \end{problem}

    \subsection*{Lösung} Wir nehmen an, dass der Wert für die spezifische
    Wärmekapazität bei konstantem Druck von Wasser bei Standardbedingungen
    \(c_p = 4,181\,\si{\joule\per{\g\kelvin}}\) hinreichend genau ist. Dann
    ergibt sich eine Kalorie zu
    \[1\,\si{\calorie} = c_pm{\Delta}T = 4,181\,\si{\joule\per{\g\kelvin}}
    \cdot 1\,\si{\g} \cdot 1\,\si{\kelvin} = 4,181\,\si{\joule}\]
    und der Umrechnungsfaktor zu
    \[\alpha = 1 = 4,181\,\si{\joule\per{\calorie}}.\]
\end{document}
