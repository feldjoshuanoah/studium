\documentclass[german,12pt]{homework}

\usepackage[ngerman]{babel}
\usepackage[utf8]{inputenc}
\usepackage[T1]{fontenc}

\usepackage{booktabs}

\usepackage{amsmath}
\usepackage{mathtools}
\usepackage{amssymb}
\usepackage{icomma}

\DeclarePairedDelimiter{\enbrace}{(}{)}
\DeclarePairedDelimiter{\benbrace}{[}{]}

\usepackage{chemformula}
\usepackage{siunitx}
\newcommand{\sis}[1]{\,\si{#1}}
\newcommand{\degC}{\si{\degreeCelsius}}

\sisetup{per-mode=fraction,sticky-per}

\title{Übung 7}
\author{Joshua Feld, 406718}
\institute{RWTH Aachen University\\Aachener Verfahrenstechnik}
\class{Material- und Stoffkunde}
\professor{Prof. Dr. Gebhardt}

\begin{document}
    \maketitle

    \section*{Aufgabe 1. (Arrhenius-Modell)}

    \begin{problem}
        Der Diffusionskoeffizient eines einatomigen, idealen Gases wurde bei verschiedenen Temperaturen gemessen. Die Messwerte sind der folgenden Tabelle zu entnehmen.
        \begin{center}
            \begin{tabular}{ll}
                \toprule
                Temperatur \(T\) & Diffusionskoeffizient \(D\)\\
                \midrule
                \(300\sis{\kelvin}\) & \(1,8 \cdot 10^{-5}\sis{\meter\squared\per\second}\)\\
                \(350\sis{\kelvin}\) & \(3,2 \cdot 10^{-5}\sis{\meter\squared\per\second}\)\\
                \(400\sis{\kelvin}\) & \(5 \cdot 10^{-5}\sis{\meter\squared\per\second}\)\\
                \(450\sis{\kelvin}\) & \(7 \cdot 10^{-5}\sis{\meter\squared\per\second}\)\\
                \bottomrule
            \end{tabular}
        \end{center}
        \begin{enumerate}
            \item Verifizieren Sie rechnerisch, dass die Messdaten durch das Arrhenius-Modell
            \[D = D_0\exp\enbrace*{-\frac{E_a}{k_BT}}\]
            angenähert werden können. Bestimmen Sie dazu die Parameter \(D_0\) und \(E_a\). Zeigen Sie anschließend, dass die Messwerte mit den errechneten Parametern hinreichend genau reproduziert werden können.
            \item Bestimmen Sie rechnerisch die Wärmemenge, die einem Mol des Gases bei \(300\sis{\kelvin}\) zugeführt werden muss, um einen Diffusionskoeffizienten von \(D = 1,5 \cdot 10^{-4}\sis{\meter\squared\per\second}\) zu erreichen.
            \item Lösen Sie die obigen Aufgabenteile graphisch durch eine geeignete Auftragung der Messwerte.
        \end{enumerate}
    \end{problem}

    \subsection*{Lösung}

    \section*{Aufgabe 2. (Kohlensäure)}

    \begin{problem}
        Cola wird bei der Herstellung mit einer Massenkonzentration \(\gamma = 7\sis{\gram\per\liter}\) an Kohlensäure (\ch{CO2}) angereichert. Durch die PET-Hülle der Flasche entweicht das gelöste \ch{CO2} durch Diffusion. Die Löslichkeit von \ch{CO2} in PET ist abhängig von der Konzentration von \ch{CO2} in der Flüssigkeit. Sie kann mit Hilfe eines Löslichkeitskoeffizienten \(H\) berechnet werden:
        \[\gamma_{\ch{CO2},\text{PET}} = H \cdot \gamma_{\ch{CO2},\text{Cola}}.\]
        \begin{enumerate}
            \item Bestimmen Sie die anfängliche \ch{CO2}-Konzentration in der Flaschenwand an der Stelle höchster Konzentration. Auf der Außenseite wird \ch{CO2} vollständig abtransportiert, sodass die \ch{CO2}-Konzentration dort zu jeder Zeit Null ist.
            \item Berechnen Sie den anfänglichen Massenstrom an \ch{CO2}, der über die Oberfläche verloren geht. Nehmen Sie dazu an, dass sich bereits ein lineares Konzentrationsgefälle über der Dicke des PET-Mantels eingestellt hat.
            \item (Achtung: Schwierig!) Bestimmen Sie die Zeit nach der \(5\%\) des gelösten \ch{CO2} die Flasche durch Diffusion verlassen hat. Stellen Sie dazu zunächst eine Gleichung für die zeitliche Massenänderung auf.
        \end{enumerate}
        Treffen Sie zur Berechnung die folgenden Annahmen:
        \begin{itemize}
            \item Die Löslichkeit \(H\) ist konstant.
            \item Das Konzentrationsprofil über der Manteldicke ist linear.
            \item Das Volumen der Flüssigkeit bleibt konstant.
            \item Die Diffusion im Inneren der Flasche ist viel schneller als die Diffusion durch den PET-Mantel, sodass das Innere als ideal durchgemischt angenommen werden kann, d.h. es gibt kein \ch{CO2}-Konzentrationsgefälle in der Flüssigkeit.
        \end{itemize}
        \textbf{Gegeben:}

        \begin{center}
            \begin{tabular}{ll}
                \toprule
                Diffusionskoeffizient & \(D = 2 \cdot 10^{-13}\sis{\meter\squared\per\second}\)\\
                Oberfläche der Flasche & \(A = 0,07\sis{\meter\squared}\)\\
                Dicke des PET-Mantels & \(d = 0,27\sis{\milli\meter}\)\\
                Flüssigkeitsvolumen & \(V = 1,5\sis{\liter}\)\\
                Löslichkeitskoeffizient & \(H = 1,43\)\\
                \bottomrule
            \end{tabular}
        \end{center}
    \end{problem}

    \subsection*{Lösung}
\end{document}
