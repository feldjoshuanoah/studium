\documentclass[german,12pt]{homework}

\usepackage[ngerman]{babel}
\usepackage[utf8]{inputenc}
\usepackage[T1]{fontenc}

\usepackage{amsmath}
\usepackage{mathtools}
\usepackage{amssymb}
\usepackage{icomma}

\newcommand{\dd}{\,\differ}
\DeclareMathOperator{\differ}{d}

\DeclarePairedDelimiter{\enbrace}{(}{)}
\DeclarePairedDelimiter{\benbrace}{[}{]}

\usepackage{chemformula}
\usepackage{siunitx}

\newcommand{\sis}[1]{\,\si{#1}}
\newcommand{\degC}{\si{\degreeCelsius}}

\sisetup{per-mode=fraction,sticky-per}

\DeclareSIUnit{\calorie}{cal}
\DeclareSIUnit{\atmosphere}{atm}

\title{Übung 1}
\author{Joshua Feld, 406718}
\institute{RWTH Aachen University\\Aachener Verfahrenstechnik}
\class{Material- und Stoffkunde}
\professor{Prof. Dr. Gebhardt}

\begin{document}
    \maketitle

    \section*{Aufgabe 1. (Gültigkeit des idealen Gasgesetzes)}

    \begin{problem}
        Eine Annahme des idealen Gasgesetzes ist, dass der Druck nicht zu hoch
        ist. Ist der Druch größer als \(1\sis{\mega\pascal}\), können wir das
        ideale Gasgesetz eigentlich nicht anwenden. Betrachten Sie ein Mol
        Sauerstoff-Gas (\(\ch{O2}\)) bei einer Temperatur von
        \(273\sis{\kelvin}\) und einem Druck von \(100\sis{\bar}\).
        \begin{enumerate}
            \item Berechnen Sie das Volumen dieses Gases unter der Annahme, das
            ideale Gasgesetz sei gültig.
            \item Berechnen Sie das Volumen dieses Gases mit der
            \emph{Viralgleichung}:
            \[\frac{p\bar{V}}{RT} = 1 + \frac{B\enbrace*{T}p}{RT},\]
            wobei \(\bar{V}\) das molare Volumen und \(B\enbrace*{T}\) eine
            Funktion der Temperatur ist:
            \[B\enbrace*{T} = \enbrace*{0,039\,5 - \frac{10\sis{\kelvin}}{T}
            - \frac{1,084 \cdot 10^3\sis{\kelvin\squared}}{T^2}}
            \sis{\liter\per\mole}.\]
        \end{enumerate}
    \end{problem}

    \subsection*{Lösung}
    \begin{enumerate}
        \item Mit dem idealen Gasgesetz gilt \(V = \frac{nRT}{p}\) und somit:
        \[V = \frac{nRT}{p} = \frac{1\sis{\mole} \cdot 8,315\sis{\joule\per\mole\kelvin} \cdot 273\sis{\kelvin}}{100 \cdot 10^5\sis{\pascal}} = 2,27 \cdot 10^{-4}\sis{\meter\cubed}.\]
        \item Wir können die Viralgleichung umformen zu
        \[\bar{V} = \frac{RT}{p} + B\enbrace*{T}.\]
        Bei \(273\sis{\kelvin}\) ist \(B\enbrace*{T}\):
        \[B\enbrace*{273\sis{\kelvin}} = \enbrace*{0,039\,5 -
        \frac{10\sis{\kelvin}}{273\sis{\kelvin}} - \frac{1,084 \cdot 10^3
        \sis{\kelvin\squared}}{\enbrace*{273\sis{\kelvin}}^2}}
        \sis{\liter\per\mole} = -0,011\,7\sis{\liter\per\mole}.\]
        Damit folgt für ein Mol Sauerstoff-Gas:
        \begin{align*}
            V = n\bar{V} &= n\enbrace*{\frac{RT}{p} + B\enbrace*{T}}\\
            &= 1\sis{\mole}\enbrace*{2,27 \cdot 10^{-4}
            \sis{\meter\cubed\per\mole} - 0,011\,7\sis{\liter\per\mole}}\\
            &= 2,15 \cdot 10^{-4}\sis{\meter\cubed}.
        \end{align*}
    \end{enumerate}

    \section*{Aufgabe 2. (Van der Waals-Gleichung)}

    \begin{problem}
        Der tiefste Punkt der Erdoberfläche liegt im Meerestief ``Challanger
        Deep'' \(11\,034\sis{\meter}\) unter dem Meeresspiegel im Marianengraben
        im westlichen Pazifischen Ozean. Betrachten Sie zwei Blasen
        Kohlendioxid (\ch{CO2}), jede mit einem Volumen von
        \(1\sis{\centi\meter\cubed}\). Eine Blase befindet sich an der
        Meeresoberfläche, die andere befindet sich im Challanger Deep.
        Bestimmen Sie das Verhältnis der Mole Gas in den zwei Blasen.

        Beachten Sie auch die folgenden Informationen:
        \begin{itemize}
            \item In solchen Tiefen ist das Idealgasgesetz ungültig.
            Stattdessen muss ein anderes Gesetz genutzt werden. Verwenden Sie
            die \emph{van der Waals-Gleichung}:
            \[p = \frac{nRT}{V - nb} - \frac{an^2}{V^2}.\]
            Darin sind \(a\) und \(b\) gasspezifische Parameter. Nehmen Sie für
            Kohlendioxid folgende van der Waals-Parameter an:
            \[a = 0,036\,4\sis{\liter\squared\atmosphere\per\mole\squared},
            \quad b = 0\sis{\liter\per\mole}.\]
            (Für diese Einheiten ist die Gaskonstante \(R = 0,082\,057
            \sis{\liter\atmosphere\per\mole\kelvin}\).)
            \item Die Temperatur im Ozean ist nicht konstant. Für Tiefen
            jenseits \(2000\sis{\meter}\) unter dem Meeresspiegel entspricht die
            Temperatur der folgenden Formel:
            \[T\enbrace*{x} = 269\sis{\kelvin} + \enbrace*{0,8\sis{\kelvin}}
            \frac{2\,000\sis{\meter} - x}{1\,000\sis{\meter}},\]
            wobei \(x\) die Tiefe ist.
            \item Der Druck in \(11\,034\sis{\meter}\) Tiefe beträgt ungefähr
            \(1\,094\sis{\atmosphere}\).
            \item Die Temperatur an der Meeresoberfläche beträgt
            \(298\sis{\kelvin}\).
        \end{itemize}
    \end{problem}

    \subsection*{Lösung} Die Temperatur in \(11\,034\sis{\meter}\) Tiefe ist
    \[T = 269\sis{\kelvin} + \enbrace*{0,8\sis{\kelvin}}
    \frac{2\,000\sis{\meter} - 11\,034\sis{\meter}}{1\,000\sis{\meter}}
    = 261,8\sis{\kelvin}.\]
    Das Volumen ist gegeben (\(1 \cdot 10^{-3}\sis{\liter}\)) und somit können
    wir die Molanzahl berechnen. Mit der Annahme \(b = 0\sis{\liter\per\mole}\)
    lässt sich die kubische van der Waals-Gleichung zu der quadratischen
    Gleichung
    \[p - \frac{nRT}{V} + \frac{an^2}{V^2} = 0\]
    vereinfachen. Die Lösung dieser Gleichung ist
    \[n = \frac{\frac{RT}{V} \pm \sqrt{\enbrace*{\frac{RT}{V}}^2 -
    4\frac{a}{V^2}p}}{2\enbrace*{\frac{a}{V^2}}}.\]
    Die verschiedenen Terme sind:
    \[\frac{RT}{V} = \frac{0,082\,057\sis{\liter\atmosphere\per\mole\kelvin}
    \cdot 261,8\sis{\kelvin}}{0,001\sis{\liter}} = 21\,483
    \sis{\atmosphere\per\mole}\]
    und
    \[\frac{a}{V^2} = \frac{0,0364
    \sis{\liter\squared\atmosphere\per\mole\squared}}{1 \cdot 10^{-6}
    \sis{\liter\squared}} = 3,64 \cdot 10^4\sis{\atmosphere\per\mole\squared}\]
    und damit haben wir
    \[n_1 = \frac{21\,474 \pm \sqrt{21\,474^2 - 4 \cdot 1\,094 \cdot 3,64 \cdot 10^4}}{2 \cdot 3,64 \cdot 10^4}\sis{\mole} = 0,056\,3\sis{\mole} \lor 0,534\sis{\mole}.\]
    Die größere Molzahl gehört zu einer flüssigen Phase. Da wir an einer
    Gasblase interessiert sind, wählen wir \(n_1= 0,056\,3\sis{\mole}\).

    Am Meeresspiegel haben wir bei Standarddruck und -temperatur:
    \[n_2 = \frac{pV}{RT} = \frac{1\sis{\atmosphere} \cdot 0,001\sis{\liter}}
    {0,082\,057\sis{\liter\atmosphere\per\mole\kelvin} \cdot 298\sis{\kelvin}}
    = 4,1 \cdot 10^{-5}\sis{\mole}.\]
    Damit ist das gesuchte Verhältnis
    \[\frac{n_1}{n_2} = \frac{0,056\,3\sis{\mole}}{4,1 \cdot 10^{-5}\sis{\mole}}
    \approx 1\,373.\]
\end{document}
