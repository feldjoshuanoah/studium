\documentclass[german,12pt]{homework}

\usepackage[ngerman]{babel}
\usepackage[utf8]{inputenc}
\usepackage[T1]{fontenc}

\usepackage{amsmath}
\usepackage{mathtools}
\usepackage{amssymb}
\usepackage{icomma}

\newcommand{\dd}{\,\differ}
\DeclareMathOperator{\differ}{d}

\DeclarePairedDelimiter{\enbrace}{(}{)}
\DeclarePairedDelimiter{\benbrace}{[}{]}

\usepackage{chemformula}
\usepackage{siunitx}

\newcommand{\sis}[1]{\,\si{#1}}
\newcommand{\degC}{\si{\degreeCelsius}}

\sisetup{per-mode=fraction,sticky-per}

\DeclareSIUnit{\calorie}{cal}
\DeclareSIUnit{\atmosphere}{atm}

\title{Übung 3}
\author{Joshua Feld, 406718}
\institute{RWTH Aachen University\\Aachener Verfahrenstechnik}
\class{Material- und Stoffkunde}
\professor{Prof. Dr. Gebhardt}

\begin{document}
    \maketitle

    \section*{Aufgabe 1. (Wärmekapazität)}

    \begin{problem}
        Bei vielen Lebensmitteln unterscheidet sich die Wärmekapazität im
        gefrorenen Zustand von der im aufgetauten Zustand. Brokkoli zum
        Beispiel hat eine Wärmekapazität von \(3,85
        \sis{\kilo\joule\per\kilogram\kelvin}\) unter der Schmelztemperatur von
        Wasser und nur \(1,84\sis{\kilo\joule\per\kilogram\kelvin}\) oberhalb
        der Schmelztemperatur. Wie viel Energie muss man aufwenden, um die
        Temperatur von \(500\sis{\gram}\) Brokkoli von \(-18\degC\) auf
        \(50\degC\) zu erhöhen? Ignorieren sie bei der Berechnung den Einfuss
        der Schmelzwärme von Wasser.
    \end{problem}

    \subsection*{Lösung}

    \section*{Aufgabe 2. (Erwärmung von Eisen)}

    \begin{problem}
        Eisen der Masse \(300\sis{\gram}\) wird durch eine elektrische Heizung
        mit einer konstanten Leistungsaufnahme von \(200\sis{\watt}\) in einem
        Isoliermantel aufgeheizt. Berechnen Sie die Temperatur des
        Eisenblockes nach \(t = 60\sis{\second}\), wenn man davon ausgeht, dass
        er zum Zeitpunkt \(t = 0\sis{\second}\) eine Temperatur von
        \(T_0 = 298\sis{\kelvin}\) hatte und dass durch die Isolierung keine
        Wärme über die Oberfäche entweicht.

        \textbf{Hinweis:} \quad \emph{Elektrische Arbeit kann vollständig in
        Wärme umgewandelt werden.}
    \end{problem}

    \subsection*{Lösung}

    \section*{Aufgabe 3. (Quentschen von Zink)}

    \begin{problem}
        Zink der Masse \(20\sis{\kilogram}\) wird in einem Wasserbad der Masse
        \(200\sis{\kilogram}\) gequentscht. Das Zink hat eine Anfangstemperatur
        von \(400\degC\) und die Temperatur des Wasserbades beträgt zu Beginn
        \(25\degC\). Welche Temperatur hat das Wasser nach Beendigung des
        Prozesses? Die Wärmekapazität des Wasserbehälters und dessen
        Wärmeabgabe an die Umgebung sei vernachlässigbar klein.
    \end{problem}

    \subsection*{Lösung}

    \section*{Aufgabe 4. (Schmelzen von Eis)}

    \begin{problem}
        Sie möchten Getränke möglichst schnell kühlen ohne sie zu verwässern. Sie haben \(2\sis{\kilogram}\) Eis bei einer Temperatur von
        \(-20\degC\) zur Verfügung. Ihr Dozent in Material- und Stoffkunde erzählt Ihnen, dass der Wärmeübergang in einem Wasserbad wesentlich schneller abläuft als in einem Eisbett. Daher entscheiden Sie sich, Leitungswasser mit einer Temperatur von \(15\degC\) zu Ihrem Eis zu mischen bis das entstehende Wasserbad eine Temperatur von \(0\degC\) hat. Wie viel Wasser müssen Sie hinzufügen? Nehmen Sie für Eis eine Wärmekapazität von \(2\,060\sis{\joule\per\kilogram\kelvin}\) an.
    \end{problem}

    \subsection*{Lösung}

    \section*{Aufgabe 5. (Erwärmung von Wasser)}

    \begin{problem}
        Ein großer Topf, der mit \(10\sis{\liter}\) Wasser gefüllt ist wird mit
        einer konstanten Leistung von \(4\sis{\kilo\watt}\) erhitzt. Die
        Anfangstemperatur des Wassers beträgt \(5\degC\), die Endtemperatur
        beträgt \(70\degC\). Der Topf ist nicht isoliert und gibt nachdem das
        Wasser eine Temperatur von \(25\degC\) erreicht hat, eine konstante
        Wärmeleistung von \(500\sis{\watt}\) an die Umgebung ab.

        Folgende Stoffdaten sind gegeben und können über den gesamten
        Temperaturbereich als konstant angenommen werden:
        \begin{itemize}
            \item Dichte von Wasser: \(\rho_W = 1\,000
            \sis{\kilogram\per\meter\cubed}\)
            \item spezifische Wärmekapazität von Wasser: \(c_{p, W} =
            4\,181\sis{\joule\per\kilogram\kelvin}\)
        \end{itemize}
        \begin{enumerate}
            \item Wie viele Minuten dauert der Heizvorgang?
            \item Wegen eines Herdausfalls von \(10\sis{\minute}\) sinkt die
            Temperatur des Wassers wieder. Welche Temperatur wird nach
            \(10\sis{\minute}\) erreicht?
        \end{enumerate}
    \end{problem}
\end{document}
