\documentclass[german,12pt]{homework}

\usepackage[ngerman]{babel}
\usepackage[utf8]{inputenc}
\usepackage[T1]{fontenc}

\usepackage{amsmath}
\usepackage{mathtools}
\usepackage{amssymb}
\usepackage{icomma}

\DeclarePairedDelimiter{\enbrace}{(}{)}
\DeclarePairedDelimiter{\benbrace}{[}{]}

\usepackage{chemformula}
\usepackage{siunitx}
\newcommand{\sis}[1]{\,\si{#1}}
\newcommand{\degC}{\si{\degreeCelsius}}

\sisetup{per-mode=fraction,sticky-per}

\title{Übung 6}
\author{Joshua Feld, 406718}
\institute{RWTH Aachen University\\Aachener Verfahrenstechnik}
\class{Material- und Stoffkunde}
\professor{Prof. Dr. Gebhardt}

\begin{document}
    \maketitle

    \section*{Aufgabe 1. (Stationäre Wärmeleitung)}

    \begin{problem}
        Die eindimensionale Wärmeleitung wird durch das Fouriersche Gesetz
        beschrieben:
        \[\dot{Q} = -\kappa{A}\frac{\partial{T\enbrace*{x}}}{\partial{x}}.\]
        Betrachten wir eine Platte der Dicke \(d\), so stellt sich nach
        Ablaufen aller zeitlich veränderlicher Prozesse ein lineares
        Temperaturprofil über die Dicke der Platte ein. Es gilt
        \[T\enbrace*{x} = T_A = \frac{T_B - T_A}{d}x,\]
        mit den Temperaturen \(T_A\) bei \(x = 0\) und \(T_B\) bei \(x = d\) an
        den Oberflächen der Platte. Mit
        \[\frac{\partial{T\enbrace*{x}}}{\partial{x}} = \frac{T_B - T_A}{d}\]
        ergibt sich der Wärmestrom durch die Platte zu
        \[\dot{Q} = -\kappa{A}\frac{T_B - T_A}{d}.\]
        Der Zustand in dem alle zeitlich veränderlichen Prozesse abgelaufen
        sind wird als stationärer Zustand bezeichnet.

        Betrachten Sie zwei Platten, eine Kupferplatte (Wärmeleitfähigkeit
        \(\kappa_1 = 320\sis{\watt\per\meter\kelvin}\)) der Dicke \(d_1 =
        5\sis{\centi\meter}\) und eine Eisenplatte (Wärmeleitfähigkeit
        \(\kappa_2 = 80\sis{\watt\per\meter\kelvin}\)) der Dicke \(d_2 =
        10\sis{\centi\meter}\). Die Platten stehen in direktem Kontakt und es
        existiert kein Temperatursprung über die Kontaktstelle. Über den
        Plattenverbund herrscht eine Temperaturdifferenz \(\Delta{T} = 80
        \sis{\kelvin}\). Bestimmen Sie die Wärmestromdichte
        \(\frac{\dot{Q}}{A}\) über den Plattenverbund im stationären Zustand.
        Gehen Sie dabei wie folgt vor:
        \begin{enumerate}
            \item Stellen Sie die Ausdrücke für die Wärmestromdichten
            \(\frac{\dot{Q_1}}{A}\) und \(\frac{\dot{Q_2}}{A}\) in Abhängigkeit
            der Temperaturdifferenzen \(\Delta{T_1}\) und \(\Delta{T_2}\) auf.
            \item Ermitteln Sie anhand einer Inzidenzmatrix die Bestimmtheit
            des entstehenden Gleichungssystems.
            \item Fügen Sie gegebenenfalls Gleichungen hinzu, die das Lösen der
            Gleichungssystems erlauben.
            \item Lösen Sie das Gleichungssystem.
        \end{enumerate}
    \end{problem}

    \subsection*{Lösung}

    \section*{Aufgabe 2. (Wärmeausdehnung I)}

    \begin{problem}
        Ein WÜrfel aus einem unbekannten Stoff ist komplett mit
        \(1\sis{\liter}\) bei \(3,98\degC\) gefüllt. Die Seiten des Würfels
        haben eine Dicke von \(1\sis{\centi\meter}\). Welchen Wert muss der
        Linearausdehnungskoeffizient haben, damit der Würfel bei \(0\degC\)
        immer noch komplett gefüllt ist? Der Raumausdehnungskoeffizient von
        Wasser in diesem Temperaturbereich beträgt ungefähr \(-0,5 \cdot 10^{-4}
        \sis{\per\kelvin}\). Halten Sie den errechneten Wert für sinnvoll?
    \end{problem}

    \subsection*{Lösung}

    \section*{Aufgabe 3. (Wärmeausdehnung II)}

    \begin{problem}
        Aufgrund von Temperaturschwankungen zwischen Sommer und Winter und der
        damit verbundenen Längenänderung von metallischen Bauteilen, werden
        beim Bau von Brücken üblicherweise Dehnungsfugen eingeplant. Die
        Ingenieure von Impractical Ideas, Inc. entwickeln eine alternative
        Methode: Die Hauptkonstruktion einer Versuchsbrücke der Länge
        \(100\sis{\meter}\) besteht aus Baustahl mit einem
        Linearausdehnungskoeffizienten \(\alpha_S = 1,2 \cdot 10^{-5}
        \sis{\per\kelvin}\). An einem Brückenkopf wird ein Konstruktionselement
        der Länge \(1\sis{\meter}\) aus einer speziellen Legierung verbaut. Die
        Legierung besitzt einen besonders hohen Linearausdehnungskoeffizienten
        \(\alpha_L = 4,2 \cdot 10^{-4}\sis{\per\kelvin}\). Die Brücke wird im
        Winter bei einer Temperatur von \(-20\degC\) spannungsfrei montiert und
        das spezielle Konstruktionselement soll im Sommer so gekühlt werden,
        dass die Brücke bis zu einer Temperatur von \(30\degC\) spannungsfrei
        bleibt. Berechnen Sie die minimale Temperatur, die das spezielle
        Konstruktionselement erreichen können muss.
    \end{problem}

    \subsection*{Lösung}
\end{document}
