\documentclass[german,12pt]{homework}

\usepackage[ngerman]{babel}
\usepackage[utf8]{inputenc}
\usepackage[T1]{fontenc}

\usepackage{booktabs}

\newcommand{\ZZ}{\mathbb{Z}}

\DeclarePairedDelimiter{\enbrace}{(}{)}
\DeclarePairedDelimiter{\benbrace}{[}{]}
\DeclarePairedDelimiter{\penbrace}{\{}{\}}

\title{Exercise 3}
\author{Joshua Feld, 406718}
\institute{RWTH Aachen University\\Chair of Process Control Engineering}
\class{Process Measurement}
\professor{Prof Dr. Kleinert}

\begin{document}
    \maketitle

    \section*{Task 1. (Strain Gauges)}

    \begin{problem}
        \begin{enumerate}
            \item Describe the working principle of a strain gauge.
            \item Deduce the linear correlation between resistance and length
            variation.
            \item The figure depicts the circuit diagram of a full bridge
            circuit. Specify the linearized equation that describes the
            correlation between measured voltage \(U_m\), feeding voltage
            \(U_s\) and the resistors \(R_i\) (\(i = 1, \ldots, 4\)).
            \item For the purpose of this task, assume that the depicted full
            bridge circuit is attached to a carrier for bending measurements.
            Determine the impact of temperature changes within the resistors on
            the measured voltage \(U_m\).
        \end{enumerate}
    \end{problem}

    \subsection*{Lösung}

    \section*{Task 2. (Capacitive Displacement Measurement)}
    \begin{problem}
        The structural setup of a capacitive displacement measurement, which
        includes three capacitor plates, is depicted in the figure below.
        Specify the formula for the determination of \(x\). Assume that the
        capacity is given.
    \end{problem}

    \subsection*{Lösung}
\end{document}
